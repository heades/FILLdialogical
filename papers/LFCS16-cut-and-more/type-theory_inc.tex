% Theorems
\newtheorem{theorem}{Theorem}
\newtheorem{lemma}[theorem]{Lemma}
\newtheorem{fact}[theorem]{Fact}
\newtheorem{corollary}[theorem]{Corollary}
\newtheorem{definition}[theorem]{Definition}
\newtheorem{remark}[theorem]{Remark}
\newtheorem{proposition}[theorem]{Proposition}
\newtheorem{notn}[theorem]{Notation}
\newtheorem{observation}[theorem]{Observation}

% Commands that are useful for writing about type theory and programming language design.
%% \newcommand{\case}[4]{\text{case}\ #1\ \text{of}\ #2\text{.}#3\text{,}#2\text{.}#4}
\newcommand{\interp}[1]{\llbracket #1 \rrbracket}
\newcommand{\normto}[0]{\rightsquigarrow^{!}}
\newcommand{\join}[0]{\downarrow}
\newcommand{\redto}[0]{\rightsquigarrow}
\newcommand{\nat}[0]{\mathbb{N}}
\newcommand{\fun}[2]{\lambda #1.#2}
\newcommand{\CRI}[0]{\text{CR-Norm}}
\newcommand{\CRII}[0]{\text{CR-Pres}}
\newcommand{\CRIII}[0]{\text{CR-Prog}}
\newcommand{\subexp}[0]{\sqsubseteq}
%% Must include \usepackage{mathrsfs} for this to work.
\newcommand{\powerset}[0]{\mathscr{P}}
