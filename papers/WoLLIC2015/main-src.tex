\usepackage[utf8]{inputenc}
\usepackage{amssymb,amsmath,amsthm}
\usepackage{cmll}
\usepackage{stmaryrd}
\usepackage{todonotes}
\usepackage{mathpartir}
\usepackage{fullpage}
\usepackage{hyperref}

% Theorems
\newtheorem{theorem}{Theorem}
\newtheorem{lemma}[theorem]{Lemma}
\newtheorem{fact}[theorem]{Fact}
\newtheorem{corollary}[theorem]{Corollary}
\newtheorem{definition}[theorem]{Definition}
\newtheorem{remark}[theorem]{Remark}
\newtheorem{proposition}[theorem]{Proposition}
\newtheorem{notn}[theorem]{Notation}
\newtheorem{observation}[theorem]{Observation}

% Commands that are useful for writing about type theory and programming language design.
%% \newcommand{\case}[4]{\text{case}\ #1\ \text{of}\ #2\text{.}#3\text{,}#2\text{.}#4}
\newcommand{\interp}[1]{\llbracket #1 \rrbracket}
\newcommand{\normto}[0]{\rightsquigarrow^{!}}
\newcommand{\join}[0]{\downarrow}
\newcommand{\redto}[0]{\rightsquigarrow}
\newcommand{\nat}[0]{\mathbb{N}}
\newcommand{\fun}[2]{\lambda #1.#2}
\newcommand{\CRI}[0]{\text{CR-Norm}}
\newcommand{\CRII}[0]{\text{CR-Pres}}
\newcommand{\CRIII}[0]{\text{CR-Prog}}
\newcommand{\subexp}[0]{\sqsubseteq}
%% Must include \usepackage{mathrsfs} for this to work.
\newcommand{\powerset}[0]{\mathscr{P}}


\title{Yet Another Short Note on Full Intuitionistic Linear Logic}
\author{Valeria de Paiva and Harley Eades III}

% Ott includes.
\usepackage{color}
\usepackage{supertabular}
\input{FILL-ott}

% Renewing some Ott commands to shrink some of the labels.
\renewcommand{\FILLdrulename}[1]{\scriptsize \textsc{#1}}

\begin{document}

\maketitle

\begin{abstract}
  Full Intuitionistic Linear Logic (FILL) was first introduced by
  Hyland and de Paiva as one of the results of their investigation
  into a categorical understanding of G\"odel's Dialectica
  interpretation. FILL went against current beliefs that it was not
  possible to incorporate all of the linear connectives, e.g. tensor,
  par, and implication, into an intuitionistic linear logic.  However,
  they showed that it is natural to support all of the connectives
  given that sequents have multiple hypothesis and multiple
  conclusions.  To enforce intuitionism de Paiva's original
  formalization of FILL used the well-known Dragalin restriction
  forcing the implication right rule to have only a single conclusion
  in its premise, but Schellinx later showed that this results in a
  failure of cut-elimination.  To overcome this failure Hyland and de
  Paiva introduced a term assignment to FILL that eliminated the need
  for the restriction.  The main idea was to lift the restriction,
  assign variables to each hypothesis and terms to each conclusion,
  and then add the property that the variable annotating the
  hypothesis being discharged when applying the implication right rule
  can only be free in the term annotating the conclusion of the
  implication being introduced.  Unfortunately, Bierman was able to
  show in his short note that this formalization of FILL still did not
  enjoy cut-elimination, because of a flaw in the left rule for par.
  However, Bellin proposed an alternate left rule for par and
  conjectured that by adopting his rule cut-elimination is restored.
  In this note we show that by adopting Bellin's proposed rule one
  obtains cut-elimination for FILL.  Additionally, we show that this
  new formalization can be modeled by a new form of dialectica
  category called order-enriched dialectica categories of de Paiva,
  and discuss future work giving FILL a semantics in Lorenzen games.
\end{abstract}

\section{Introduction}
\label{sec:introduction}
\cite{dePaiva:2005}
In \cite{Hyland:1993} Martin Hyland and Veleria de Paiva give a term
formalization of Full Intuitionistic Linear Logic (FILL), but later
Bierman was able to give a counterexample to cut-elimination
\cite{Bierman:1996}.  As Bierman explains the problem was that the
left rule for par introduced a fresh variable into to many terms on
the right-side of the conclusion.  This resulted in a counterexample
where this fresh variable became bound in one term, but is left free
in another. This resulted from first doing a commuting conversion on
cut, and then $\lambda$-binding the fresh variable.  Thus,
cut-elmination failed.  In the conclusion of Bierman's paper he gives
an alternate left-par rule which he attributes to Bellin, and states
that this alternate rule should fix the problem with cut-elimination
\cite{Bierman:1996}.  In this note we adopt Bellin's rule, and then
show cut-elimination in Section~\ref{sec:cut-elimination}.
% section introduction (end)

\section{Full Intuitionistic Linear Logic (FILL)}
\label{sec:full_intuitionistic_linear_logic_(fill)}

In this section we give a brief description of Full Intuitionistic
Linear Logic (FILL) in the style found in \cite{Hyland:1993}.
However, we use a slightly different presentation that we feel
provides a more elegant description of the logic.  We first give the
syntax of formulas, patterns, terms, and contexts.  Following the syntax we
define several meta-functions that will be used when defining the
inference rules of the logic.

\begin{definition}
  \label{def:syntax}
  The syntax for FILL is as follows:
  \begin{center}
    \begin{math}
      \begin{array}{cll}
        \text{(Formulas)}       & [[A]], [[B]], [[C]], [[D]], [[E]] ::= [[I]] \mid [[_|_]]
        \mid [[A -o B]] \mid [[A (x) B]] \mid [[A (+) B]] \\
        \text{(Patterns)} & [[p]] ::= [[stp]] \mid [[-]] \mid [[x]] \mid [[p1 (x)
        p2]] \mid [[p1 (+) p2]]\\
        \text{(Terms)}          & [[t]], [[e]] ::= [[x]] \mid [[*]] \mid [[o]] \mid
        [[t1 (x) t2]] \mid [[t1 (+) t2]] \mid [[\x.t]] \mid [[let t be p in e]] \mid [[t1 t2]]\\
        \text{(Left Contexts)}  & [[G]] ::= [[.]] \mid [[x : A]] \mid [[G1,G2]]\\
        \text{(Right Contexts)} & [[L]] ::= [[.]] \mid [[t : A]] \mid [[L1,L2]]\\
      \end{array}
    \end{math}
  \end{center}
\end{definition}

At this point we introduce some basic syntax and definitions to
facilitate readability, and presentation of the inference rules.
First, we will often write $[[L1 | L2]]$ as syntactic sugar for
$[[L1,L2]]$.  The former syntax should be read as ``$[[L1]]$ or
$[[L2]]$.'' This will help readability of the sequent we will
introduce below.  We denote the usual capture-avoiding substitution by
$[[ [t/x]t']]$, and its extension to right contexts as $[[ [t/x]L]]$.

% \begin{definition}
%   \label{def:delta-sub}
%   We extend the capture-avoiding substitution function to right
%   contexts as follows:
%   \begin{center}
%     \begin{math}
%       \begin{array}{lll}
%         [[ [t/x] .]] = [[.]]\\
%         [[ [t/x] (t' : A)]] = [[{[t/x]t'} : A]]\\
%         [[ [t/x] (L1 | L2)]] = [[([t/x]L1) | ([t/x]L2)]]\\
%       \end{array}
%     \end{math}
%   \end{center}
% \end{definition}
The previous extension will make conducting substitutions across a
sequence of terms in an inference rule easier.  Similarly, we find
it convenient to be able to do this style of extension for the
let-binding as well.
\begin{definition}
  \label{def:delta-let}
  We extend let-binding terms to right contexts as follows:
  \begin{center}
    \begin{math}
      \begin{array}{lll}
        [[ let t be p in .]] = [[.]]\\
        [[ let t be p in (t' : A)]] = [[{let t be p in t'} : A]]\\
        [[ let t be p in (L1 | L2)]] = [[(let t be p in L1) | (let t be p in L2)]]\\
      \end{array}
    \end{math}
  \end{center}
\end{definition}
We denote the usual function that computes the set of free variables
in a term by $\mathsf{FV}([[t]])$, and its straightforward extension to
right contexts as $\mathsf{FV}([[L]])$.  
% \begin{definition}
%   \label{def:delta-FV}
%   We extend the free-variable function on terms to right contexts as
%   follows:
%   \begin{center}
%     \begin{math}
%       \begin{array}{lll}
%         [[FV(.)]] = \emptyset\\
%         [[FV(t : A)]] = [[FV(t)]]\\
%         [[FV(L1 | L2)]] = [[FV(L1)]] \cup [[FV(L2)]]\\
%       \end{array}
%     \end{math}
%   \end{center}
% \end{definition}
Finally, we arrive at the inference rules of FILL.
\begin{definition}
  \label{def:infr}
  The inference rules for derivability in FILL are as follows:
  \vspace{-10px}
  \begin{center}
    \scriptsize
      \begin{mathpar}
        \FILLdruleAx{}    \and 
        \FILLdruleCut{}     \and 
        \FILLdruleIl{}            \and 
        \FILLdruleIr{}    \and 
        \FILLdruleTl{}    \and 
        \FILLdruleTr{}    \and 
        \FILLdrulePl{}    \and 
        \FILLdrulePr{}    \and 
        \FILLdruleParl{}    \and 
        \FILLdruleParr{}    \and 
        \FILLdruleImpl{}    \and 
        \FILLdruleImpr{}    \and 
        \FILLdruleExl{}    \and 
        \FILLdruleExr{}    
    \end{mathpar}
  \end{center}
\end{definition}

The $\FILLdrulename{Parl}$ rule depends on a function $[[let-pat z p
L]]$.  We define this function next.
\begin{definition}
  \label{def:let-pat-term}
  The function $[[let-pat z p t]]$ is defined as follows:
  \begin{center}
    \begin{math}
      \begin{array}{lll}      
        \begin{array}{lll}
          [[let-pat z (x (+) -) t]] = [[t]]\\
          \,\,\,\,\,\,\text{where } [[x]] \not\in \mathsf{FV}([[t]])\\
        \end{array}
        & 
          \begin{array}{lll}
            [[let-pat z (- (+) y) t]] = [[t]]\\
        \,\,\,\,\,\,\text{where } [[y]] \not\in \mathsf{FV}([[t]])\\
          \end{array}
        & 
          \begin{array}{lll}
            [[let-pat z p t]] = [[let z be p in t]]\\
            & \\
          \end{array}
      \end{array}
    \end{math}
  \end{center}
  It is straightforward to extend the previous definition to
  right-contexts, and we denote this extension by $[[let-pat z p L]]$.
  % \begin{center}
  %   \begin{math}
  %     \begin{array}{lll}      
  %       [[let-pat z p .]] = [[.]]\\                
  %       [[let-pat z p (t : A)]] = [[{let-pat z p t} : A]]\\
  %       [[let-pat z p (L1 | L2)]] = [[(let-pat z p L1) | (let-pat z p L2)]]\\
  %     \end{array}
  %   \end{math}
  % \end{center}
\end{definition}
The motivation behind this function is that it only binds the pattern
variables in $[[p]]$ in a term if and only if those pattern variables
are free in the term.  This over comes the counterexample given by
Bierman in \cite{Bierman:1996}.  Throughout the sequel we will denote
derivations of the previous rules by $\pi$.
% section full_intuitionistic_linear_logic_(fill) (end)

\section{Cut-elimination}
\label{sec:cut-elimination}
The usual proof of cut-elimination for intuitionistic and classical
linear logic should suffice for FILL.  Thus, in this section we simply
give the cut-elimination procedure for FILL following the development
in \cite{Mellies:2009}.  However, there is one invariant that must be
verified across each derivation transformation.  The invariant is that
if a derivation $\pi$ is transformed into a derivation $\pi'$, then
the terms in the conclusion of the final rule applied in $\pi$ must be
equivalent to the terms in the conclusion of the final rule applied in
$\pi'$, but using what notion of equivalence?

\begin{definition}
  \label{def:FILL-eq}
  Equivalence on terms is defined as follows:
  \begin{center}
    \footnotesize
    \begin{mathpar}
      \FILLdruleAlpha{} \and
      \FILLdruleEtaFun{} \and
      \FILLdruleBetaFun{} \and
      \FILLdruleEtaOneI{} \and
\begin{report}
  \FILLdruleEtaTwoI{} \and
\end{report}
      \FILLdruleBetaI{} \and
      \FILLdruleNatI{} \and
      \begin{report}
        \FILLdruleEtaTen{} \and
      \end{report}
      \FILLdruleBetaOneTen{} \and
      \FILLdruleBetaTwoTen{} \and
      \FILLdruleNatTen{} \and
      \FILLdruleEtaParU{} \and
      \FILLdruleEtaPar{} \and
      \FILLdruleBetaOnePar{} \and
      \FILLdruleBetaTwoPar{} \and
      \FILLdruleNatOnePar{} \and
      \FILLdruleNatTwoPar{} 
      \begin{report}
        \and
      \FILLdruleLam{} \and
      \FILLdruleAppOne{} \and
      \FILLdruleAppTwo{} \and
      \FILLdruleTenOne{} \and
      \FILLdruleTenTwo{} \and
      \FILLdruleParOne{} \and
      \FILLdruleParTwo{} \and
      \FILLdruleLetOne{} \and
      \FILLdruleLetTwo{} \and
      \FILLdruleRefl{} \and
      \FILLdruleSym{} \and
      \FILLdruleTrans{}
      \end{report}
    \end{mathpar}
  \end{center}
\end{definition}

The cut elimination procedure requires the following two basic
results:
\begin{lemma}[Substitution Distribution]
  \label{lemma:substitution_distribution}
  For any terms $[[t]]$, $[[t1]]$, and $[[t2]]$, $[[ [t1/x][t2/y]t]] = [[ [ [t1/x]t2/y][t2/x]t]]$.
\end{lemma}
\begin{proof}
  This proof holds by straightforward induction on the form of $t$.
\end{proof}

\begin{lemma}[Let-pat Distribution]
  \label{lemma:let-pat_distribution}
  For any terms $[[t]]$, $[[t1]]$, and $[[t2]]$, and pattern p, \\
  $[[ let-pat t p [t1/y]t2]] = [[ [ let-pat t p t1/y]t2]]$.
\end{lemma}
\begin{proof}
  This proof holds by case splitting over $p$, and then using the
  naturality equations for the respective pattern.
\end{proof}

Throughout the remainder of this section we present a particular step
in the cut-elimination procedure, and then give a short proof that
equality of terms are preserved across the particular transformation
on derivations. \begin{paper}Many of the transformations are trivial,
  and follow directly from the traditional proof.  Thus, we only
  present here the most interesting cases.  The full
  proof can be found in the companion report
  \cite{comprepo}. \end{paper}

\begin{report}
\subsection{Commuting conversion cut vs cut (first case)}
\label{subsec:commuting_conversion_cut_vs_cut_(first_case)}
The following proof
\begin{center}
  \begin{math}
    $$\mprset{flushleft}
    \inferrule* [right=Cut] {
      $$\mprset{flushleft}
      \inferrule* [right=] {
        \inferrule* [right=,vdots=1.5em,fraction=\,] {
          \,
        }{\pi_1}
      }{[[G |- t : A | L]]}
      \\
      $$\mprset{flushleft}
      \inferrule* [right=Cut] {
        $$\mprset{flushleft}
        \inferrule* [right=] {
          \inferrule* [right=,vdots=1.5em,fraction=\,] {
            \,
          }{\pi_2}
        }{[[G2,x : A,G3 |- t1 : B | L1]]}
        \\
        $$\mprset{flushleft}
        \inferrule* [right=] {
          \inferrule* [right=,vdots=1.5em,fraction=\,] {
            \,
          }{\pi_3}
        }{[[G1,y : B,G4 |- L2]]}
      }{[[G1,G2,x : A,G3,G4 |- h(L1 | [t1/y]L2)]]}
    }{[[G1,G2,G,G3,G4 |- h(L | [t/x]L1) | [t/x]h([t1/y]L2)]]}
  \end{math}
\end{center}
is transformed into the proof
\begin{center}
  \begin{math}
    $$\mprset{flushleft}
    \inferrule* [right=Cut] {
      $$\mprset{flushleft}
      \inferrule* [right=] {
        $$\mprset{flushleft}
      \inferrule* [right=] {
        \inferrule* [right=,vdots=1.5em,fraction=\,] {
          \,
        }{\pi_1}               
      }{[[G |- t : A | L]]}
      \\
      $$\mprset{flushleft}
        \inferrule* [right=] {
          \inferrule* [right=,vdots=1.5em,fraction=\,] {
            \,
          }{\pi_2}
        }{[[G2,x : A,G3 |- t1 : B | L1]]}
      }{[[G2,G,G3 |- h{[t/x]t1} : B | [t/x]L1]]}
      \\
      $$\mprset{flushleft}
        \inferrule* [right=] {
          \inferrule* [right=,vdots=1.5em,fraction=\,] {
            \,
          }{\pi_3}
        }{[[G1,y : B,G4 |- L2]]}
    }{[[G1,G2,G,G3,G4 |- h(L | [t/x]L1) | [ [t/x]t1/y]L2]]}
  \end{math}
\end{center}
First, if $[[L2]]$ is empty, then all the terms in the conclusion of
the previous two derivations are equivalent.  
So suppose $[[L2]] = [[t2 : C | L2']]$.  Then we know that the term
$[[ [t/x][t1/y]t2]]$ in the first derivation above is equivalent to
$[[ [ [t/x] t1/y][t/x] t2]]$ by
Lemma~\ref{lemma:substitution_distribution}.  Furthermore, by
inspecting the first derivation we can see that $[[x nin FV(t2)]]$,
and thus, $[[ [ [t/x] t1/y][t/x] t2 = [ [t/x] t1/y] t2]]$.  This
argument may be repeated for any term in $[[L2']]$, and thus, we know
$[[ [t/x][t1/y]L2 = [ [t/x]t1/y]L2]]$.
% subsection commuting_conversion_cut_vs_cut_(first_case) (end)

\subsection{Commuting conversion cut vs. cut (second case)}
\label{subsec:commuting_conversion_cut_vs._cut_(second_case)}
The second commuting conversion on cut begins with the proof
\begin{center}
  \begin{math}
    $$\mprset{flushleft}
    \inferrule* [right=Cut] {
      $$\mprset{flushleft}
      \inferrule* [right=] {
        \inferrule* [right=,vdots=1.5em,fraction=\,] {
          \,
        }{\pi_1}
      }{[[G |- t : A | L]]}
      \\
      $$\mprset{flushleft}
      \inferrule* [right=Cut] {
        $$\mprset{flushleft}
        \inferrule* [right=] {
          \inferrule* [right=,vdots=1.5em,fraction=\,] {
            \,
          }{\pi_2}
        }{[[G' |- t' : B | L']]}
        \\
        $$\mprset{flushleft}
        \inferrule* [right=] {
          \inferrule* [right=,vdots=1.5em,fraction=\,] {
            \,
          }{\pi_3}
        }{[[G1,x : A,G2,y:B,G3 |- L1]]}
      }{[[G1,x : A,G2,G',G3 |- L' | [t'/y]L1]]}
    }{[[G1,G,G2,G',G3 |- h(L | [t/x]L') | [t/x]h([t'/y]L1)]]}
  \end{math}
\end{center}
is transformed into the following proof:
\begin{center}
  \begin{math}
    $$\mprset{flushleft}
    \inferrule* [right=\text{\scriptsize Series of Exchanges}] {
      $$\mprset{flushleft}
    \inferrule* [right=Cut] {
      $$\mprset{flushleft}
      \inferrule* [right=] {
        \inferrule* [right=,vdots=1.5em,fraction=\,] {
          \,
        }{\pi_2}
      }{[[G' |- t' : B | L']]}
      \\
      $$\mprset{flushleft}
      \inferrule* [right=Cut] {
        $$\mprset{flushleft}
        \inferrule* [right=] {
          \inferrule* [right=,vdots=1.5em,fraction=\,] {
            \,
          }{\pi_1}
        }{[[G |- t : A | L]]}      
        \\
        $$\mprset{flushleft}
        \inferrule* [right=] {
          \inferrule* [right=,vdots=1.5em,fraction=\,] {
            \,
          }{\pi_3}
        }{[[G1,x : A,G2,y:B,G3 |- L1]]}
      }{[[G1,G,G2,y:B,G3 |- L | [t/x]L1]]}
    }{[[G1,G,G2,G',G3 |- h(L' | [t'/y]L) | [t'/y]h([t/x]L1)]]}
    }{[[G1,G,G2,G',G3 |- h(h([t'/y]L) | L') | [t'/y]h([t/x]L1)]]}
  \end{math}
\end{center}
We know $[[x,y
nin
FV(L)]]$ by inspection of the first derivation, and so we know that
$[[L
= [t'/y]L]]$ and $[[L' =
[t/x]L']]$.  Without loss of generality suppose $[[L1]] = [[t1 : C |
L'1]]$. Then we know that $[[x,y nin FV(t)]]$ and $[[x,y nin
FV(t')]]$.  Thus, by this fact and
Lemma~\ref{lemma:substitution_distribution}, we know that $[[
[t/x][t'/y]t1 = [ [t/x]t'/y][t/x]t1]] = [[ [t'/y][t/x]t1
]]$.  This argument can be repeated for any term in
$[[L'1]]$, hence, $[[ [t/x][t'/y]L1 = [t'/y][t/x]L1]]$.
% subsection commuting_conversion_cut_vs._cut_(second_case) (end)
\end{report}

\begin{report}
\subsection{The $\eta$-expansion cases}
\label{subsec:eta-expansion_cases}

\subsubsection{Tensor}
\label{subsec:tensor}
The proof
\begin{center}
  \begin{math}
    $$\mprset{flushleft}
    \inferrule* [right=Ax] {
      \,
    }{[[x : A (x) B |- x : A (x) B]]}
  \end{math}
\end{center}
is transformed into the proof
\begin{center}
  \begin{math}
    $$\mprset{flushleft}
    \inferrule* [right=Tl] {
      $$\mprset{flushleft}
      \inferrule* [right=Tr] {
        $$\mprset{flushleft}
        \inferrule* [right=Ax] {
          \,
        }{[[y : A |- y : A]]}
        \\
        $$\mprset{flushleft}
        \inferrule* [right=Ax] {
          \,
        }{[[z : B |- z : B]]}
      }{[[y : A, z : B |- y (x) z : A (x) B]]}
    }{[[x : A (x) B |- h{let x be y (x) z in {y (x) z}} : A (x) B]]}
  \end{math}
\end{center}
By the rule $\FILLdrulename{EQ\_EtaTensor}$
we know $[[let x be y (x) z in {y (x) z} = x]]$.
% subsubsection tensor (end)

\subsubsection{Par}
\label{subsec:par}
The proof
\begin{center}
  \begin{math}
    $$\mprset{flushleft}
    \inferrule* [right=Ax] {
      \,
    }{[[x : A (+) B |- x : A (+) B]]}
  \end{math}
\end{center}
is transformed into the proof
\begin{center}
  \begin{math}
    $$\mprset{flushleft}
    \inferrule* [right=Parr] {
      $$\mprset{flushleft}
      \inferrule* [right=Parl] {
        $$\mprset{flushleft}
        \inferrule* [right=Ax] {
          \,
        }{[[y : A |- y : A]]}
        \\
        $$\mprset{flushleft}
        \inferrule* [right=Ax] {
          \,
        }{[[z : B |- z : B]]}
      }{[[x : A (+) B |- h{let x be (y (+) -) in y} : A | h{let x be (- (+) z) in z} : B]]}
    }{[[x : A (+) B |- {let x be (y (+) -) in y} (+) {let x be (- (+) z) in z} : A (+) B]]}
  \end{math}
\end{center}
By rule $\FILLdrulename{Eq\_EtaPar}$ we know
$[[{{let x be (y (+) -) in y} (+) {let x be (- (+) z) in z}} = x]]$.
% subsubsection par (end)

\subsubsection{Implication}
\label{subsec:implication}
The proof
\begin{center}
  \begin{math}
    $$\mprset{flushleft}
    \inferrule* [right=Ax] {
      \,
    }{[[x : A -o B |- x : A -o B]]}
  \end{math}
\end{center}
transforms into the proof 
\begin{center}
  \begin{math}
    $$\mprset{flushleft}
    \inferrule* [right=ImpR] {
      $$\mprset{flushleft}
      \inferrule* [right=ImpL] {
        $$\mprset{flushleft}
        \inferrule* [right=Ax] {
          \,
        }{[[y : A |- y : A]]}
        \\
        $$\mprset{flushleft}
        \inferrule* [right=Ax] {
          \,
        }{[[z : B |- z : B]]}
      }{[[y : A, x : A -o B |- x y : B]]}
    }{[[x : A -o B |- \y.h{x y} : A -o B]]}
  \end{math}  
\end{center}
All terms in the two derivations are equivalent, because
$[[{\y.h{x y}} = x]]$ by the $\FILLdrulename{Eq\_EtaFun}$ rule.
% subsubsection implication (end)

\subsubsection{Tensor unit}
\label{subsec:tensor_unit}
The proof
\begin{center}
  \begin{math}
    $$\mprset{flushleft}
    \inferrule* [right=Ax] {
      \,
    }{[[x : I |- x : I]]}
  \end{math}
\end{center}
transforms into the proof
\begin{center}
  \begin{math}
    $$\mprset{flushleft}
    \inferrule* [right=Il] {
      $$\mprset{flushleft}
      \inferrule* [right=Ir] {
        \,
      }{[[. |- * : I]]}
    }{[[x : I |- h{let x be stp in *} : I]]}
  \end{math}
\end{center}
We know $[[x = let x be stp in *]]$ by
$\FILLdrulename{Eq\_EtaI}$.
% subsubsection tensor_unit (end)

\subsubsection{Par unit}
\label{subsec:tensor_unit}
The proof
\begin{center}
  \begin{math}
    $$\mprset{flushleft}
    \inferrule* [right=Ax] {
      \,
    }{[[x : _|_ |- x : _|_]]}
  \end{math}
\end{center}
transforms into the proof
\begin{center}
  \begin{math}
    $$\mprset{flushleft}
    \inferrule* [right=Pr] {
      $$\mprset{flushleft}
      \inferrule* [right=Pl] {
        \,
      }{[[x : _|_ |- .]]}
    }{[[x : _|_ |- o : _|_]]}
  \end{math}
\end{center}
We know $[[x = o]]$ by $\FILLdrulename{Eq\_EtaParU}$.
% subsubsection tensor_unit (end)
% subsection eta-expansion_cases (end)
\end{report}

\begin{report}
\subsection{The axiom steps}
\label{sec:the_axiom_steps}
\subsubsection{The axiom step}
\label{subsec:the_axiom_step}
The proof 
\begin{center}
  \begin{math}
    $$\mprset{flushleft}
    \inferrule* [right=Cut] {
      $$\mprset{flushleft}
      \inferrule* [right=Ax] {
        \,
      }{[[x : A |- x : A]]}
      \\
      $$\mprset{flushleft}
      \inferrule* [right=] {
        \inferrule* [right=,vdots=1.5em,fraction=\,] {
            \,
          }{\pi}          
      }{[[G1,y : A, G2 |- L]]}
    }{[[G1,x : A, G2 |- [x/y]L]]}
  \end{math}
\end{center}
transforms into the proof
\begin{center}
  \begin{math}
    $$\mprset{flushleft}
      \inferrule* [right=] {
        \inferrule* [right=,vdots=1.5em,fraction=\,] {
            \,
          }{\pi}          
      }{[[G1,y : A, G2 |- L]]}
  \end{math}
\end{center}
By $\FILLdrulename{Eq\_Alpha}$, we know, for any $[[t]]$ in
$[[L]]$, $[[t = [x/y]t]]$, and hence $[[L = [x/y]L]]$.
% subsection the_axiom_step (end)
\end{report}

\begin{report}  
\subsubsection{Conclusion vs. axom}
\label{subsec:conclusion_vs._axom}
The proof 
\begin{center}
  \begin{math}
    $$\mprset{flushleft}
    \inferrule* [right=Cut] {
      $$\mprset{flushleft}
      \inferrule* [right=] {
        \inferrule* [right=,vdots=1.5em,fraction=\,] {
            \,
          }{\pi}          
      }{[[G |- t : A | L]]}
      \,
      $$\mprset{flushleft}
      \inferrule* [right=Ax] {
        \,
      }{[[x : A |- x : A]]}
    }{[[G |- L | h{[t/x]x} : A]]}
  \end{math}
\end{center}
transforms into 
\begin{center}
  \begin{math}
    $$\mprset{flushleft}
    \inferrule* [right=\text{\tiny Series of Exchanges}] {
      \inferrule* [right=] {
        \inferrule* [right=,vdots=1.5em,fraction=\,] {
            \,
          }{\pi}          
      }{[[G |- t : A | L]]}
    }{[[G |- L | t : A ]]}
  \end{math}
\end{center}
By the definition of the substitution function we know $[[t =
[t/x]x]]$.
% subsection conclusion_vs._axom (end)
% section the_axiom_steps (end)
\end{report}

\begin{report}
\subsection{The exchange steps}
\label{subsec:the_exchange_steps}
\subsubsection{Conclusion vs. left-exchange (the first case)}
\label{subsec:conclusion_vs._exchange_(the_first_case)}
The proof
\begin{center}
  \begin{math}
    $$\mprset{flushleft}
    \inferrule* [right=Cut] {
      \inferrule* [right=] {
        \inferrule* [right=,vdots=1.5em,fraction=\,] {
            \,
          }{\pi_1}          
      }{[[G |- t : A | L]]}
      \\
      $$\mprset{flushleft}
      \inferrule* [right=Exl] {        
        $$\mprset{flushleft}
        \inferrule* [right=] {
          \inferrule* [right=,vdots=1.5em,fraction=\,] {
            \,
          }{\pi_2}          
        }{[[G1,x : A, y : B, G2 |- L']]}        
      }{[[G1,y : B,x : A, G2 |- L']]}
    }{[[G1,y : B,G, G2 |- L | [t/x]L']]}
  \end{math}
\end{center}
transforms into the proof
\begin{center}
  \begin{math}
    $$\mprset{flushleft}
    \inferrule* [right=\text{Series of Exchanges}] {
      $$\mprset{flushleft}
      \inferrule* [right=Cut] {
        \inferrule* [right=] {
        \inferrule* [right=,vdots=1.5em,fraction=\,] {
            \,
          }{\pi_1}          
      }{[[G |- t : A | L]]}
      \\
        $$\mprset{flushleft}
        \inferrule* [right=] {
          \inferrule* [right=,vdots=1.5em,fraction=\,] {
            \,
          }{\pi_2}          
        }{[[G1,x : A, y : B, G2 |- L']]}        
      }{[[G1,G,y : B, G2 |- L | [t/x]L']]}
    }{[[G1,y : B,G, G2 |- L | [t/x]L']]}
  \end{math}
\end{center}
Clearly, all terms are equivalent.
% subsubsection conclusion_vs._exchange_(the_first_case) (end)

\subsubsection{Conclusion vs. left-exchange (the second case)}
\label{subsec:conclusion_vs._exchange_(the_second_case)}
The proof
\begin{center}
  \begin{math}
    $$\mprset{flushleft}
    \inferrule* [right=Cut] {
      \inferrule* [right=] {
        \inferrule* [right=,vdots=1.5em,fraction=\,] {
            \,
          }{\pi_1}          
      }{[[G |- t : B | L]]}
      \\
      $$\mprset{flushleft}
      \inferrule* [right=Exl] {        
        $$\mprset{flushleft}
        \inferrule* [right=] {
          \inferrule* [right=,vdots=1.5em,fraction=\,] {
            \,
          }{\pi_2}          
        }{[[G1,x : A, y : B, G2 |- L']]}        
      }{[[G1,y : B,x : A, G2 |- L']]}
    }{[[G1,G, x : A,G2 |- L | [t/y]L']]}
  \end{math}
\end{center}
transforms into the proof
\begin{center}
  \begin{math}
    $$\mprset{flushleft}
    \inferrule* [right=\text{Series of Exchanges}] {
      $$\mprset{flushleft}
      \inferrule* [right=Cut] {
        \inferrule* [right=] {
        \inferrule* [right=,vdots=1.5em,fraction=\,] {
            \,
          }{\pi_1}          
      }{[[G |- t : B | L]]}
      \\
        $$\mprset{flushleft}
        \inferrule* [right=] {
          \inferrule* [right=,vdots=1.5em,fraction=\,] {
            \,
          }{\pi_2}          
        }{[[G1,x : A, y : B, G2 |- L']]}        
      }{[[G1,x : A,G,G2 |- L | [t/y]L']]}
    }{[[G1,G,x : A,G2 |- L | [t/y]L']]}
  \end{math}
\end{center}
Clearly, all terms are equivalent.
% subsubsection conclusion_vs._exchange_(the_second_case) (end)

\subsubsection{Conclusion vs. right-exchange}
\label{subsec:conclusion_vs._right-exchange}
The proof
\begin{center}
  \begin{math}
    $$\mprset{flushleft}
    \inferrule* [right=Cut] {
        \inferrule* [right=] {
        \inferrule* [right=,vdots=1.5em,fraction=\,] {
            \,
          }{\pi_1}          
      }{[[G |- t : A | L]]}
      \\
      $$\mprset{flushleft}
      \inferrule* [right=Exr] {
        $$\mprset{flushleft}
        \inferrule* [right=] {
          \inferrule* [right=,vdots=1.5em,fraction=\,] {
            \,
          }{\pi_2}          
        }{[[G1,x : A, G2 |- h(h(L1 | t1 : B) | t2 : C) | L']]}        
      }{[[G1,x : A, G2 |- h(h(L1 | t2 : C) | t1 : B) | L']]}
    }{[[G1,G,G2 |- L | h(h(h(h([t/x]L1) | h{[t/x]t2} : C) | h{[t/x]t1} : B) | [t/x]L')]]}
  \end{math}
\end{center}
transforms into this proof
\begin{center}
  \begin{math}
    $$\mprset{flushleft}
    \inferrule* [right=Exr] {
      $$\mprset{flushleft}
      \inferrule* [right=Cut] {
        \inferrule* [right=] {
        \inferrule* [right=,vdots=1.5em,fraction=\,] {
            \,
          }{\pi_1}          
      }{[[G |- t : A | L]]}
      \\
      $$\mprset{flushleft}
        \inferrule* [right=] {
          \inferrule* [right=,vdots=1.5em,fraction=\,] {
            \,
          }{\pi_2}          
        }{[[G1,x : A, G2 |- h(h(L1 | t1 : B) | t2 : C) | L']]}        
      }{[[G1,G, G2 |- L | h(h(h(h([t/x]L1) | h{[t/x]t1} : B) | h{[t/x]t2} : C) | [t/x]L')]]}
    }{[[G1,G, G2 |- h(h(h([t/x]L1) | h{[t/x]t2} : C) | h{[t/x]t1} : B) | [t/x]L']]}
  \end{math}
\end{center}
Clearly, all terms are equivalent.
% subsubsection conclusion_vs._right-exchange_(the_first_case) (end)
% subsection the_exchange_steps (end)
\end{report}

\subsection{Principle formula vs. principle formula}
\label{subsec:principle_formula_vs._principle_formula}

\begin{report} 
\subsubsection{Tensor}
\label{subsec:tensor}
The proof 
\begin{center}

  \begin{math}
    $$\mprset{flushleft}
    \inferrule* [right=\footnotesize Cut] {
      $$\mprset{flushleft}
      \inferrule* [right=\footnotesize Tr] {
        \inferrule* [right=] {
        \inferrule* [right=,vdots=1.5em,fraction=\,] {
            \,
          }{\pi_1}          
      }{[[G1 |- t1 : A | L1]]}
      \\
      \inferrule* [right=] {
        \inferrule* [right=,vdots=1.5em,fraction=\,] {
            \,
          }{\pi_2}          
      }{[[G2 |- t2 : B | L2]]}
      }{[[G1,G2 |- t1 (x) t2 : A (x) B | h(L1 | L2)]]}
      \\
      $$\mprset{flushleft}
      \inferrule* [right=\footnotesize Tl] {
        \inferrule* [right=] {
          \inferrule* [right=,vdots=1.5em,fraction=\,] {
            \,
          }{\pi_3}          
        }{[[G3,x : A, y : B,G4 |- L3]]}
      }{[[G3,z : A (x) B,G4 |- h(let z be x (x) y in L3)]]}
    }{[[G3,G1,G2,G4 |- h(L1 | L2) | [t1 (x) t2/z](h(let z be x (x) y in L3))]]}
  \end{math}
\end{center}
is transformed into the proof
\begin{center}
  \begin{math}
    $$\mprset{flushleft}
    \inferrule* [right=Cut] {
      \inferrule* [right=] {
        \inferrule* [right=,vdots=1.5em,fraction=\,] {
            \,
          }{\pi_1}          
      }{[[G1 |- t1 : A | L1]]}
      \\
      $$\mprset{flushleft}
      \inferrule* [right=Cut] {
        \inferrule* [right=] {
        \inferrule* [right=,vdots=1.5em,fraction=\,] {
            \,
          }{\pi_2}          
      }{[[G2 |- t2 : B | L2]]}
      \\
      \inferrule* [right=] {
          \inferrule* [right=,vdots=1.5em,fraction=\,] {
            \,
          }{\pi_3}          
        }{[[G3,x : A, y : B,G4 |- L3]]}
      }{[[G3,x : A, G2,G4 |- L2 | [t2/y]L3]]}
    }{[[G3,G1, G2,G4 |- h(L1 | L2) | [t1/x][t2/y]L3]]}
  \end{math}
\end{center}
Without loss of generality suppose $[[L3 = t3
: C, L'3]]$.  We can see that $[[h{[t1 (x) t2/z]{let z be x (x) y in
    t3}} = let t1 (x) t2 be x (x) y in t3]]$ by the definition of
substitution, and by using the $\FILLdrulename{Eq\_Beta1Tensor}$ rule we obtain
$[[let t1 (x) t2 be x (x) y in t3 = [t1/x][t2/y]t3]]$.  This argument
can be repeated for any term in $[[ [t1 (x) t2/z](h(let z be x (x) y
in L'3))]]$, and thus, $[[ [t1 (x) t2/z](h(let z be x (x) y in L3)) =
[t1/x][t2/y]L3]]$.

Note that in the second derivation of the above transformation we
first cut on $[[B]]$, and then $[[A]]$, but we could have cut on
$[[A]]$ first, and then $[[B]]$, but this would yeild equivalent
derivations as above by using
Lemma~\ref{lemma:substitution_distribution}.
% subsubsection tensor (end)
\end{report}
\subsubsection{Par}
\label{subsec:par}
The proof
\begin{center}
  \scriptsize
  \begin{math}
    $$\mprset{flushleft}
\inferrule* [right=\scriptsize Cut] {
  $$\mprset{flushleft}
  \inferrule* [right=\tiny Parr] {
    \inferrule* [right=] {
        \inferrule* [right=,vdots=1.5em,fraction=\,] {
            \,
          }{\pi_1}          
      }{[[G1 |- h(L1 | t1 : A) | h(t2 : B | L2)]]}
    }{[[G1 |- h(L1 | t1 (+) t2 : A (+) B) | L2]]}
  \\
  $$\mprset{flushleft}
  \inferrule* [right=\tiny Parl] {
    \inferrule* [right=] {
        \inferrule* [right=,vdots=1.5em,fraction=\,] {
            \,
          }{\pi_2}          
      }{[[G2,x : A |- L3]]}
      \\
      \inferrule* [right=] {
        \inferrule* [right=,vdots=1.5em,fraction=\,] {
            \,
          }{\pi_3}          
      }{[[G3, y : B |- L4]]}
  }{[[G2,G3,z : A (+) B |- h(let-pat z (x (+) -) L3) | h(let-pat z (- (+) y) L4)]]}
}{[[G2,G3,G1 |- h(h(L1 | L2) | h([t1 (+) t2/z](let-pat z (x (+) -) L3))) | h([t1 (+) t2/z](let-pat z (- (+) y) L4))]]}
  \end{math}
\end{center}
is transformed into the proof
\begin{center}
  \scriptsize
  \begin{math}    
    $$\mprset{flushleft}
    \inferrule* [right=  \scriptsize \tiny Series of Exchanges] {
      $$\mprset{flushleft}
    \inferrule* [right=  \scriptsize Cut] {
      $$\mprset{flushleft}
      \inferrule* [right=  \scriptsize Cut] {
        \inferrule* [right=] {
          \inferrule* [right=,vdots=1.5em,fraction=\,] {
            \,
          }{\pi_1}          
        }{[[G1 |- h(L1 | t1 : A) | h(t2 : B | L2)]]}
        \\
        \inferrule* [right=] {
          \inferrule* [right=,vdots=1.5em,fraction=\,] {
            \,
          }{\pi_3}          
        }{[[G3, y : B |- L4]]}
      }{[[G3, G1 |- h(h(L1 | t1 : A) | L2) | [t2/y]L4]]}
      \\
      \inferrule* [right=] {
        \inferrule* [right=,vdots=1.5em,fraction=\,] {
          \,
        }{\pi_2}          
      }{[[G2,x : A |- L3]]}
    }{[[G2,G3,G1 |- h(h(L1 | L2) | [t2/y]L4) | [t1/x]L3]]}
    }{[[G2,G3,G1 |- h(L1 | L2) | h(h([t1/x]L3) | [t2/y]L4)]]}
  \end{math}
\end{center}
Without loss of generality consider the case when
$[[L3]] = [[t3 : C_1 | L'3]]$ and $[[L4]] = [[t4 : C_2 | L'4]]$.  First,
$[[ [t1 (+) t2/z]{let-pat z (x (+) -) t3} = let-pat {t1 (+) t2} (x (+)
-) t3]]$,
and by $\FILLdrulename{Eq\_Beta1Par}$ we know
$[[let-pat {t1 (+) t2} (x (+) -) t3 = [t1/x]t3]]$ if $[[x in FV(t3)]]$
or $[[let-pat {t1 (+) t2} (x (+) -) t3 = t3]]$ otherwise.  In the
latter case we can see that $[[t3 = [t1/x]t3]]$, thus, in both cases
$[[let-pat {t1 (+) t2} (x (+) -) t3 = [t1/x]t3]]$.  This argument can
be repeated for any terms in $[[L'3]]$, and hence
$[[ [t1 (+) t2/z](let-pat z (x (+) -) L3)]] = [[let-pat {t1 (+) t2} (x
(+) -) L3 = [t1/x]L3]]$.
We can apply a similar argument for
$[[ [t1 (+) t2/z]{let-pat z (- (+) y) t4}]]$ and
$[[ [t1 (+) t2/z](let-pat z (- (+) y) L4)]]$.

\begin{paper}
  Note that we could have first cut on $[[A]]$, and then on $[[B]]$ in
  the second derivation, but we would have arrived at the same result
  just with potentially more exchanges on the right.
\end{paper}
\begin{report}
  Note that just as we mentioned about tensor we could have first cut on
$[[A]]$, and then on $[[B]]$ in the second derivation, but we would
have arrived at the same result just with potentially more exchanges
on the right.
\end{report}
% subsubsection par (end)

\subsubsection{Implication}
\label{subsubsec:implication}
The proof
\begin{center}
  \scriptsize
  \begin{math}
    $$\mprset{flushleft}
    \inferrule* [right=\scriptsize Cut] {
      $$\mprset{flushleft}
      \inferrule* [right=\scriptsize Impr] {
        \inferrule* [right=] {
          \inferrule* [right=,vdots=1.5em,fraction=\,] {
            \,
          }{\pi_1}          
        }{[[G, x : A |- t : B | L]]}
        \\
        [[x nin FV(L)]]
      }{[[G |- \x.t : A -o B | L]]}
      \\
      $$\mprset{flushleft}
      \inferrule* [right=\scriptsize Impl] {
        \inferrule* [right=] {
          \inferrule* [right=,vdots=1.5em,fraction=\,] {
            \,
          }{\pi_2}          
        }{[[G1 |- t1 : A | L1]]}
        \\
        \inferrule* [right=] {
          \inferrule* [right=,vdots=1.5em,fraction=\,] {
            \,
          }{\pi_3}          
        }{[[G2, y : B |- L2]]}
      }{[[G1, z : A -o B,G2 |- L1 | [z t1/y]L2]]}
    }{[[G1,G,G2 |- L | h(h([\x.t/z]L1) | h([\x.t/z][z t1/y]L2))]]}
  \end{math}
\end{center}
transforms into the proof
\begin{center}
  \scriptsize
  \begin{math}
    $$\mprset{flushleft}
    \inferrule* [right=\tiny Series of Exchanges] {
      $$\mprset{flushleft}
    \inferrule* [right=\scriptsize Cut] {
      $$\mprset{flushleft}
      \inferrule* [right=\scriptsize Cut] {
        \inferrule* [right=] {
          \inferrule* [right=,vdots=1.5em,fraction=\,] {
            \,
          }{\pi_2}          
        }{[[G1 |- t1 : A | L1]]}
        \\
        \inferrule* [right=] {
          \inferrule* [right=,vdots=1.5em,fraction=\,] {
            \,
          }{\pi_1}          
        }{[[G, x : A |- t : B | L]]}
        \\
        [[x nin FV(L)]]
      }{[[G, G1 |- L1 | h(h{[t1/x]t} : B | [t1/x]L)]]}
      \\
      \inferrule* [right=] {
          \inferrule* [right=,vdots=1.5em,fraction=\,] {
            \,
          }{\pi_3}          
        }{[[G2, y : B |- L2]]}
      }{[[G2, G,G1 |- h(L1 | [t1/x]L) | [ [t1/x]t/y]L2]]}
    }{[[G1, G,G2 |- h([t1/x]L) | h(L1 | h([ [t1/x]t/y]L2))]]}
  \end{math}
\end{center}
Without loss of generality consider the case when $[[L2]] = [[t2 : C | L'2]]$. 
First, by hypothesis we know $[[x nin FV(L)]]$, and so we know $[[L = [t1/x]L]]$.  We can see
that $[[ [\x.t/z][z t1/y]t2 = [{\x.t} t1/y]t2]] = [[ [ [t1/x]t/y]t2]]$ by using the congruence rules
of equality and the rule $\FILLdrulename{Eq\_BetaFun}$.  This argument can be repeated for any term
in $[[ [\x.t/z][z t1/y]L'2]]$, and so $[[ [\x.t/z][z t1/y]L2 = [ [t1/x]t/y]L2]]$.  Finally, by inspecting the previous
derivations we can see that $[[z nin FV(L1)]]$, and thus, $[[L1 = [\x.t/z]L1]]$.  
% subsubsection implication (end)

\begin{report}
  
\subsubsection{Tensors Unit}
\label{subsubsec:ten-unit}
The proof
\begin{center}
  \begin{math}
    $$\mprset{flushleft}
    \inferrule* [right=Cut] {
      $$\mprset{flushleft}
      \inferrule* [right=Ir] {
        \,
      }{[[. |- * : I]]}
      \\
      $$\mprset{flushleft}
      \inferrule* [right=Il] {
        \inferrule* [right=] {
          \inferrule* [right=,vdots=1.5em,fraction=\,] {
            \,
          }{\pi}          
        }{[[G |- L]]}
      }{[[G, x : I |- let x be stp in L]]}
    }{[[G |- [*/x](let x be stp in L)]]}
  \end{math}
\end{center}
is transformed into the proof
\begin{center}
  \begin{math}
    \inferrule* [right=] {
      \inferrule* [right=,vdots=1.5em,fraction=\,] {
        \,
      }{\pi}          
    }{[[G |- L]]}
  \end{math}
\end{center}

Without loss of generality suppose $[[L]] = [[t : A | L']]$. We can see that 
$[[ [*/x]{let x be stp in t} = let * be stp in t]] = [[t]]$ by the definition of 
substitution and the $\FILLdrulename{Eq\_EtaI}$ rule.  This argument can be repeated for any
term in $[[ [*/x](let x be stp in L')]]$, and hence, $[[ [*/x](let x be
stp in L) = L]]$.
% subsubsection ten-unit (end)

\subsubsection{Pars Unit}
\label{subsec:pars_unit}
The proof
\begin{center}
  \begin{math}
    $$\mprset{flushleft}
    \inferrule* [right=Cut] {
      $$\mprset{flushleft}
      \inferrule* [right=Pr] {
        \inferrule* [right=] {
          \inferrule* [right=,vdots=1.5em,fraction=\,] {
            \,
          }{\pi}          
        }{[[G |- L]]}
      }{[[G |- o : _|_ | L]]}
      \\
      $$\mprset{flushleft}
      \inferrule* [right=Pl] {
        \,
      }{[[x : _|_ |- .]]}
    }{[[G |- L | [o/x].]]}
  \end{math}
\end{center}
transforms into the proof
\begin{center}
  \begin{math}
    \inferrule* [right=] {
      \inferrule* [right=,vdots=1.5em,fraction=\,] {
        \,
      }{\pi}          
    }{[[G |- L]]}
  \end{math}
\end{center}
Clearly, $[[ [o/x]. = .]]$.
% subsubsection pars_unit (end)
\end{report}
% subsection principle_formula_vs._principle_formula (end)
\begin{report}  
\subsection{Secondary conclusion}
\label{subsec:secondary_conclusion}


\subsubsection{Left introduction of implication}
\label{subsec:left_introduction_of_implication}
The proof 
\begin{center}
  \begin{math}
    $$\mprset{flushleft}
    \inferrule* [right=Cut] {
      $$\mprset{flushleft}
      \inferrule* [right=Impl] {
        \inferrule* [right=] {
          \inferrule* [right=,vdots=1.5em,fraction=\,] {
            \,
          }{\pi_1}          
        }{[[G |- t1 : A | L]]}
        \\
        \inferrule* [right=] {
          \inferrule* [right=,vdots=1.5em,fraction=\,] {
            \,
          }{\pi_2}          
        }{[[G1, x : B,G2 |- t2 : C | L2]]}
      }{[[G,y : A -o B,G1,G2 |- h(L | h([y t1/x]t2 : C)) | [y t1/x]L2]]}
      \\      
      \inferrule* [right=] {
          \inferrule* [right=,vdots=1.5em,fraction=\,] {
            \,
          }{\pi_3}          
        }{[[G3,z : C, G4 |- L3]]}
      }{[[G3,G,y : A -o B,G1,G2,G4 |- L | h(h([y t1/x]L2) | [ [y t1/x]t2/z]L3)]]}
  \end{math}
\end{center}
transforms into the proof
\begin{center}
  \begin{math}
    $$\mprset{flushleft}
    \inferrule* [right=\tiny Series of Exchanges] {
      $$\mprset{flushleft}
    \inferrule* [right=Impl] {
      \inferrule* [right=] {
          \inferrule* [right=,vdots=1.5em,fraction=\,] {
            \,
          }{\pi_1}          
        }{[[G |- t1 : A | L]]}
      \\
      $$\mprset{flushleft}
      \inferrule* [right=Cut] {
        \inferrule* [right=] {
          \inferrule* [right=,vdots=1.5em,fraction=\,] {
            \,
          }{\pi_2}          
        }{[[G1, x : B,G2 |- t2 : C | L2]]}
        \\
        \inferrule* [right=] {
          \inferrule* [right=,vdots=1.5em,fraction=\,] {
            \,
          }{\pi_3}          
        }{[[G3,z : C, G4 |- L3]]}
      }{[[G3,G1, x : B,G2,G4 |-L2 | [t2/z]L3]]}
    }{[[G, y : A -o B,G3,G1,G2,G4 |- h(L | h([y t1/x]L2)) | [y t1/x][t2/z]L3]]}
  }{[[G3, G,y : A -o B,G1,G2,G4 |- h(L | h([y t1/x]L2)) | [y t1/x][t2/z]L3]]}
  \end{math}
\end{center}
This case is similar to
Section~\ref{subsec:commuting_conversion_cut_vs_cut_(first_case)}.
Thus, we can prove that $[[ [y t1/x][t2/z]L3 = [ [y t1/x]t2/z]L3]]$ by
Lemma~\ref{lemma:substitution_distribution} and the fact that $[[x nin
FV(L3)]]$.
% subsubsection left_introduction_of_implication (end)

\subsubsection{Left introduction of exchange}
\label{subsec:exchange}
The proof
\begin{center}
  \begin{math}
    $$\mprset{flushleft}
    \inferrule* [right=Cut] {
      $$\mprset{flushleft}
      \inferrule* [right=Exl] {
        \inferrule* [right=] {
          \inferrule* [right=,vdots=1.5em,fraction=\,] {
            \,
          }{\pi_1}          
        }{[[G,y : B, x : A, G' |- t : C | L]]}        
      }{[[G, x : A, y : B, G' |- t : C | L]]}
      \\
      \inferrule* [right=] {
        \inferrule* [right=,vdots=1.5em,fraction=\,] {
          \,
        }{\pi_2}          
      }{[[G1,z : C, G2 |- L2]]}
    }{[[G1,G, x : A, y : B, G', G2 |- L | [t/z]L2]]}
  \end{math}
\end{center}
transforms into the proof
\begin{center}
  \begin{math}
    $$\mprset{flushleft}
    \inferrule* [right=Exl] {
      $$\mprset{flushleft}
      \inferrule* [right=Cut] {
        \inferrule* [right=] {
          \inferrule* [right=,vdots=1.5em,fraction=\,] {
            \,
          }{\pi_1}          
        }{[[G,y : B, x : A, G' |- t : C | L]]}        
        \\
        \inferrule* [right=] {
        \inferrule* [right=,vdots=1.5em,fraction=\,] {
          \,
        }{\pi_2}          
      }{[[G1,z : C, G2 |- L2]]}
    }{[[G1,G,y : B, x : A, G', G2 |- L | [t/z]L2]]}
  }{[[G1,G, x : A, y : B, G', G2 |- L | [t/z]L2]]}
  \end{math}
\end{center}
Clearly, all terms are equivalent.
% subsubsection exchange (end)

\subsubsection{Left introduction of tensor}
\label{subsec:left-intro-tensor}
The proof 
\begin{center}
  \begin{math}
    $$\mprset{flushleft}
    \inferrule* [right=Cut] {
      $$\mprset{flushleft}
      \inferrule* [right=Tl] {
        \inferrule* [right=] {
        \inferrule* [right=,vdots=1.5em,fraction=\,] {
          \,
        }{\pi_1}          
      }{[[G, x : A, y : B |- t : C | L]]}      
      }{[[G, z : A (x) B |- h{let z be x (x) y in t} : C | let z be x (x) y in L]]}
      \\
      \inferrule* [right=] {
        \inferrule* [right=,vdots=1.5em,fraction=\,] {
          \,
        }{\pi_2}          
      }{[[G1,w : C, G2 |- L2]]}
    }{[[G1,G, z : A (x) B, G2 |- h(let z be x (x) y in L) | [h{let z be x (x) y in t}/w]L2]]}
  \end{math}
\end{center}
transforms into the proof
\begin{center}
  \begin{math}
    $$\mprset{flushleft}
    \inferrule* [right=Tl] {
      $$\mprset{flushleft}
      \inferrule* [right=Cut] {
        \inferrule* [right=] {
        \inferrule* [right=,vdots=1.5em,fraction=\,] {
          \,
        }{\pi_1}          
      }{[[G, x : A, y : B |- t : C | L]]}      
      \\
      \inferrule* [right=] {
        \inferrule* [right=,vdots=1.5em,fraction=\,] {
          \,
        }{\pi_2}          
      }{[[G1,w : C, G2 |- L2]]}
    }{[[G1,G, x : A, y : B, G2 |- L | [t/w]L2]]}      
  }{[[G1,G, z : A (x) B, G2 |- h(let z be x (x) y in L) | let z be x (x) y in ([t/w]L2)]]}
  \end{math}
\end{center}
It suffices to show that $[[let z be x (x) y in ([t/w]L2) = [h{let z
  be x (x) y in t}/w]L2]]$.  This is a simple consequence of the rule \FILLdrulename{Eq\_NatTensor}.
% subsubsection tensor (end)


\subsubsection{Left introduction of Par}
\label{subsec:left_introduction_of_par}
The proof 
\begin{center}
  \scriptsize
  \begin{math}
    $$\mprset{flushleft}
    \inferrule* [right=\scriptsize Cut] {
      $$\mprset{flushleft}
      \inferrule* [right=\scriptsize Parl] {
        \inferrule* [right=] {
          \inferrule* [right=,vdots=1.5em,fraction=\,] {
            \,
          }{\pi_1}          
        }{[[G, x : A |- L]]}      
        \\
        \inferrule* [right=] {
          \inferrule* [right=,vdots=1.5em,fraction=\,] {
            \,
          }{\pi_2}          
        }{[[G', y : B |- t' : C | L']]}      
      }{[[G,G',z : A (+) B |- h(h(let-pat z (x (+) -) L) | h{let-pat z (- (+) y) t'} : C) | h(let-pat z (- (+) y) L')]]}
      \\
      \inferrule* [right=] {
        \inferrule* [right=,vdots=1.5em,fraction=\,] {
          \,
        }{\pi_3}          
      }{[[G1,w : C, G2 |- L2]]}
    }{[[G1,G,G',z : A (+) B, G2 |- h(h(let-pat z (x (+) -) L) | h(let-pat z (- (+) y) L')) | [h{let-pat z (- (+) y) t'}/w]L2 ]]}
  \end{math}
\end{center}
is transformed into the proof
\begin{center}
  \begin{math}
    $$\mprset{flushleft}
    \inferrule* [right=\tiny Series of Exchanges] {
      $$\mprset{flushleft}
    \inferrule* [right=Parl] {
      \inferrule* [right=] {
        \inferrule* [right=,vdots=1.5em,fraction=\,] {
          \,
        }{\pi_1}          
      }{[[G, x : A |- L]]}      
      \\
      $$\mprset{flushleft}
      \inferrule* [right=Cut] {
        \inferrule* [right=] {
          \inferrule* [right=,vdots=1.5em,fraction=\,] {
            \,
          }{\pi_2}          
        }{[[G', y : B |- t' : C | L']]}      
        \\
        \inferrule* [right=] {
          \inferrule* [right=,vdots=1.5em,fraction=\,] {
            \,
          }{\pi_3}          
        }{[[G1,w : C, G2 |- L2]]}
      }{[[G1,G', y : B, G2 |- L' | [t'/w]L2]]}
    }{[[G,G1,G', G2, z : A (+) B |- h(h(let-pat z (x (+) -) L) | h(let-pat z (- (+) y) L')) | h(let-pat z (- (+) y) [t'/w]L2)]]}
  }{[[G1,G,G',z : A (+) B,G2 |- h(h(let-pat z (x (+) -) L) | h(let-pat z (- (+) y) L')) | h(let-pat z (- (+) y) [t'/w]L2)]]}
  \end{math}
\end{center}
It suffices to show that $[[let-pat z (- (+) y) [t'/w]L2]] = [[
[h{let-pat z (- (+) y) t'}/w]L2]]$.  This follows from the rule $\FILLdrulename{Eq\_Nat2Par}$.
% subsubsection left_introduction_of_par (end)

\subsubsection{Left introduction of tensor unit}
\label{subsec:left_introduction_of_tensor_unit}
The proof
\begin{center}
  \begin{math}
    $$\mprset{flushleft}
    \inferrule* [right=Cut] {
      $$\mprset{flushleft}
      \inferrule* [right=Il] {
        \inferrule* [right=] {
        \inferrule* [right=,vdots=1.5em,fraction=\,] {
          \,
        }{\pi_1}          
      }{[[G |- t : C | L]]}      
      }{[[G, x : I |- t : C | L]]}
      \\
      \inferrule* [right=] {
          \inferrule* [right=,vdots=1.5em,fraction=\,] {
            \,
          }{\pi_2}          
        }{[[G1,w : C, G2 |- L1]]}
      }{[[G1,G,x : I,G2 |- L | [t/w]L1]]}
  \end{math}
\end{center}
is transformed into the following:
\begin{center}
  \begin{math}
    $$\mprset{flushleft}
    \inferrule* [right=\tiny Series of Exchanges] {
      $$\mprset{flushleft}
    \inferrule* [right=Il] {
      $$\mprset{flushleft}
      \inferrule* [right=Cut] {
        \inferrule* [right=] {
          \inferrule* [right=,vdots=1.5em,fraction=\,] {
            \,
          }{\pi_1}          
        }{[[G |- t : C | L]]}      
        \\
        \inferrule* [right=] {
          \inferrule* [right=,vdots=1.5em,fraction=\,] {
            \,
          }{\pi_2}          
        }{[[G1,w : C, G2 |- L1]]}  
      }{[[G1,G, G2 |- L | [t/w]L1]]}
    }{[[G1,G, G2, x : I |- L | [t/w]L1]]}
  }{[[G1,G, x : I,G2 |- L | [t/w]L1]]}
  \end{math}
\end{center}
Clearly, all terms are equivalent.  Note that we do not give a case
for secondary conclusion of the left introduction of par's unit,
because it can only be introduced given an empty right context, and
thus there is no cut formula.
% subsubsection left_introduction_of_tensor_unit (end)
\end{report}
% subsection secondary_conclusion (end)

\begin{report}
\subsection{Secondary hypothesis}
\label{subsec:secondary_hypothesis}

\subsubsection{Left introduction of tensor}
\label{subsec:left_introduction_of_tensor}
The proof
\begin{center}
  \begin{math}
    $$\mprset{flushleft}
    \inferrule* [right=Cut] {
      \inferrule* [right=] {
        \inferrule* [right=,vdots=1.5em,fraction=\,] {
          \,
        }{\pi_1}          
      }{[[G |- t : A | L]]}      
      \\
      $$\mprset{flushleft}
      \inferrule* [right=Tl] {
        \inferrule* [right=] {
        \inferrule* [right=,vdots=1.5em,fraction=\,] {
          \,
        }{\pi_2}          
      }{[[G1,x : A,G2,y : B,z : C,G3 |- t1 : D | L1]]}                  
    }{[[G1,x : A,G2,w : B (x) C,G3 |- h{let w be y (x) z in t1} : D | let w be y (x) z in L1]]}
  }{[[G1,G,G2,w : B (x) C,G3 |- L | h(h{[t/x]{let w be y (x) z in t1}} : D | [t/x](let w be y (x) z in L1))]]}
  \end{math}
\end{center}
transforms into the proof
\begin{center}
  \begin{math}
    $$\mprset{flushleft}
    \inferrule* [right=Tl] {
      $$\mprset{flushleft}
      \inferrule* [right=Cut] {
        \inferrule* [right=] {
        \inferrule* [right=,vdots=1.5em,fraction=\,] {
          \,
        }{\pi_1}          
      }{[[G |- t : A | L]]}      
      \\
      \inferrule* [right=] {
        \inferrule* [right=,vdots=1.5em,fraction=\,] {
          \,
        }{\pi_2}          
      }{[[G1,x : A,G2,y : B,z : C,G3 |- t1 : D | L1]]}                  
    }{[[G1,G,G2,y : B,z : C,G3 |- h(L | h{[t/x]t1} : D) | [t/x]L1]]}
  }{[[G1,G,G2,w : B (x) C,G3 |- h(h(let w be x (x) y in L) | h{let w be x (x) y in [t/x]t1} : D) | let w be x (x) y in [t/x]L1]]}
  \end{math}
\end{center}
First, we can see by inspection of the previous derivations that
$[[x,y nin FV(L)]]$, thus, by using similar reasoning as above we can
use the $\FILLdrulename{EtaTensor}$ rule to obtain $[[let w be x (x) y in
L = L]]$.  It is a well-known property of substitution that 
$[[ [t/x]{let w be x (x) y in t1}]] = [[let [t/x]w be x (x) y in [t/x]t1]] = [[let w be x (x) y in [t/x]t1]]$.
% subsubsection left_introduction_of_tensor (end)

\subsubsection{Right introduction of tensor (first case)}
\label{subsec:right_introduction_of_the_tensor_(first_case)}

The proof
\begin{center}
  \begin{math}
    $$\mprset{flushleft}
    \inferrule* [right=Cut] {
      \inferrule* [right=] {
        \inferrule* [right=,vdots=1.5em,fraction=\,] {
          \,
        }{\pi_1}          
      }{[[G |- t : A | L]]}      
      \\
      $$\mprset{flushleft}
      \inferrule* [right=Tr] {
        \inferrule* [right=] {
        \inferrule* [right=,vdots=1.5em,fraction=\,] {
          \,
        }{\pi_2}          
      }{[[G1,x : A, G2 |- t1 : B | L1]]}      
      \\
      \inferrule* [right=] {
        \inferrule* [right=,vdots=1.5em,fraction=\,] {
          \,
        }{\pi_3}          
      }{[[G3 |- t2 : C | L2]]}      
      }{[[G1,x :A,G2,G3 |- t1 (x) t2 : B (x) C | h(L1 | L2)]]}
    }{[[G1,G,G2,G3 |- L | h(h{[t/x]{t1 (x) t2}} : B (x) C | h(h([t/x]L1) | [t/x]L2))]]}
  \end{math}
\end{center}
transforms into the proof
\begin{center}
  \begin{math}
    $$\mprset{flushleft}
    \inferrule* [right=\tiny Series of Exchanges] {
      $$\mprset{flushleft}
    \inferrule* [right=Tr] {
      $$\mprset{flushleft}
      \inferrule* [right=Cut] {
        \inferrule* [right=] {
        \inferrule* [right=,vdots=1.5em,fraction=\,] {
          \,
        }{\pi_1}          
      }{[[G |- t : A | L]]}      
      \\
      \inferrule* [right=] {
        \inferrule* [right=,vdots=1.5em,fraction=\,] {
          \,
        }{\pi_2}          
      }{[[G1,x : A, G2 |- t1 : B | L1]]}      
    }{[[G1,G, G2 |- L | h(h{[t/x]t1} : B | [t/x]L1)]]}
    \\
    \inferrule* [right=] {
        \inferrule* [right=,vdots=1.5em,fraction=\,] {
          \,
        }{\pi_3}          
      }{[[G3 |- t2 : C | L2]]}      
    }{[[G1,G, G2,G3 |- h(h{[t/x]t1} (x) t2 : B (x) C) | h(L | h(h([t/x]L1) | L2))]]}
  }{[[G1,G, G2,G3 |- L | h(h({[t/x]t1} (x) t2 : B (x) C) | h(h([t/x]L1) | L2))]]}
  \end{math}
\end{center}
By inspection of the previous derivations we can see that $[[x nin
FV(t2)]]$ and $[[x nin FV(L2)]]$.  Thus, $[[ [t/x]L2 = L2]]$ and
$[[ [t/x]{t1 (x) t2} = {[t/x]t1} (x) {[t/x]t2}]] = [[{[t/x]t1} (x) t2]]$.
% subsubsection right_introduction_of_the_tensor_(first_case) (end)

\subsubsection{Right introduction of tensor (second case)}
\label{subsec:right_introduction_of_tensor_(second_case)}
The proof
\begin{center}
  \begin{math}
    $$\mprset{flushleft}
    \inferrule* [right=Cut] {
      \inferrule* [right=] {
        \inferrule* [right=,vdots=1.5em,fraction=\,] {
          \,
        }{\pi_1}          
      }{[[G |- t : A | L]]}      
      \\
      $$\mprset{flushleft}
      \inferrule* [right=Tr] {
        \inferrule* [right=] {
        \inferrule* [right=,vdots=1.5em,fraction=\,] {
          \,
        }{\pi_2}          
      }{[[G1 |- t1 : B | L1]]}      
      \\
      \inferrule* [right=] {
        \inferrule* [right=,vdots=1.5em,fraction=\,] {
          \,
        }{\pi_3}          
      }{[[G2,x : A, G3  |- t2 : C | L2]]}      
      }{[[G1,G2,x :A,G3 |- t1 (x) t2 : B (x) C | h(L1 | L2)]]}
    }{[[G1,G,G2,G3 |- L | h(h{[t/x]{t1 (x) t2}} : B (x) C | h(h([t/x]L1) | [t/x]L2))]]}
  \end{math}
\end{center}
transforms into the proof
\begin{center}
  \begin{math}
    $$\mprset{flushleft}
    \inferrule* [right=\tiny Series of Exchanges] {
      $$\mprset{flushleft}
      \inferrule* [right=Tr] {
        \inferrule* [right=] {
        \inferrule* [right=,vdots=1.5em,fraction=\,] {
          \,
        }{\pi_2}          
      }{[[G1 |- t1 : B | L1]]}      
      \\
      $$\mprset{flushleft}
      \inferrule* [right=Cut] {
        \inferrule* [right=] {
        \inferrule* [right=,vdots=1.5em,fraction=\,] {
          \,
        }{\pi_1}          
      }{[[G |- t : A | L]]}      
      \\
      \inferrule* [right=] {
        \inferrule* [right=,vdots=1.5em,fraction=\,] {
          \,
        }{\pi_3}          
      }{[[G2,x : A, G3 |- t2 : C | L2]]}      
    }{[[G2,G, G3 |- L | h(h{[t/x]t2} : C | [t/x]L2)]]}       
    }{[[G1,G2,G,G3 |- h(t1 (x) {[t/x]t2} : B (x) C) | h(L1 | h(L | [t/x]L2))]]}
  }{[[G1,G, G2,G3 |- L | h(h(t1 (x) {[t/x]t2} : B (x) C) | h(L1 | h([t/x]L2)))]]}
  \end{math}
\end{center}
This case is similar to the previous case.  
% subsubsection right_introduction_of_tensor_(second_case) (end)


\subsubsection{Right introduction of par}
\label{subsec:right_introduction_of_par_(first_case)}
The proof
\begin{center}
  \begin{math}
    $$\mprset{flushleft}
    \inferrule* [right=Cut] {
      \inferrule* [right=] {
        \inferrule* [right=,vdots=1.5em,fraction=\,] {
          \,
        }{\pi_1}          
      }{[[G |- t : A | L]]}      
      \\
      $$\mprset{flushleft}
      \inferrule* [right=Parr] {
        \inferrule* [right=] {
        \inferrule* [right=,vdots=1.5em,fraction=\,] {
          \,
        }{\pi_2}          
      }{[[G1,x : A,G2 |- h(L1 | h(t1 : B | t2 : C)) | L2]]}                  
    }{[[G1,x : A,G2 |- h(L1 | h(t1 (+) t2 : B (+) C)) | L2]]}
  }{[[G1,G,G2 |- L | h(h(h([t/x]L1) | h([t/x]{t1 (+) t2} : B (+) C)) | [t/x]L2)]]}
  \end{math}
\end{center}
transforms into the proof
\begin{center}
  \begin{math}
    $$\mprset{flushleft}
\inferrule* [right=Parl] {
  $$\mprset{flushleft}
  \inferrule* [right=Cut] {
    \inferrule* [right=] {
        \inferrule* [right=,vdots=1.5em,fraction=\,] {
          \,
        }{\pi_1}          
      }{[[G |- t : A | L]]}      
      \\
      \inferrule* [right=] {
        \inferrule* [right=,vdots=1.5em,fraction=\,] {
          \,
        }{\pi_2}          
      }{[[G1,x : A,G2 |- h(L1 | h(t1 : B | t2 : C)) | L2]]}                  
    }{[[G1,G,G2 |- L | h(h(h([t/x]L1) | h(h{[t/x]t1} : B | h{[t/x]t2} : C)) | [t/x]L2)]]}
  }{[[G1,G,G2 |- L | h(h(h([t/x]L1) | h{[t/x]t1} (+) h{[t/x]t2} : B (+) C) | [t/x]L2)]]}
  \end{math}
\end{center}
Clearly, $[[ [t/x]{t1 (+) t2} = {[t/x]t1} (+) [t/x]t2]]$.
% subsubsection right_introduction_of_par_(first_case) (end)

\subsubsection{Left introduction of par (first case)}
\label{subsec:left_introduction_of_par_(first_case)}
The proof
\begin{center}
  \begin{math}
    $$\mprset{flushleft}
    \inferrule* [right=Cut] {
      \inferrule* [right=] {
        \inferrule* [right=,vdots=1.5em,fraction=\,] {
          \,
        }{\pi_1}          
      }{[[G |- t : A | L]]}      
      \\
      $$\mprset{flushleft}
      \inferrule* [right=Parl] {
        \inferrule* [right=] {
        \inferrule* [right=,vdots=1.5em,fraction=\,] {
          \,
        }{\pi_2}          
      }{[[G1,x : A,G2,y : B |- L1]]}      
      \\
      \inferrule* [right=] {
        \inferrule* [right=,vdots=1.5em,fraction=\,] {
          \,
        }{\pi_3}          
      }{[[G3,z : C |- L2]]}      
      }{[[G1,x : A,G2,G3,w : B (+) C |- h(let-pat w (y (+) -) L1) | h(let-pat w (- (+) z) L2)]]}
    }{[[G1,G,G2,G3,w : B (+) C |- L | h(h([t/x](let-pat w (y (+) -) L1)) | h([t/x](let-pat w (- (+) z) L2)))]]}
  \end{math}
\end{center}
transforms into the proof
\begin{center}
  \begin{math}
    $$\mprset{flushleft}
    \inferrule* [right=ParL] {
      $$\mprset{flushleft}
      \inferrule* [right=Cut] {
        \inferrule* [right=] {
        \inferrule* [right=,vdots=1.5em,fraction=\,] {
          \,
        }{\pi_1}          
      }{[[G |- t : A | L]]}      
      \\
      \inferrule* [right=] {
        \inferrule* [right=,vdots=1.5em,fraction=\,] {
          \,
        }{\pi_2}          
      }{[[G1,x : A,G2,y : B |- L1]]}      
      }{[[G1,G,G2,y : B |- L | [t/x]L1]]}
      \\
      \inferrule* [right=] {
        \inferrule* [right=,vdots=1.5em,fraction=\,] {
          \,
        }{\pi_3}          
      }{[[G3,z : C |- L2]]}      
    }{[[G1,G,G2,G3,w : B (+) C |- h(h(let-pat w (y (+) -) L) | h(let-pat w (y (+) -) [t/x]L1)) | let-pat w (- (+) z) L2]]}
  \end{math}
\end{center}
First, by inspection of the previous proofs we can see that $[[x nin
FV(L)]]$ and $[[x nin FV(L2)]]$.  Thus, $[[let-pat w (y (+) -) L =
L]]$, and $[[ [t/x](let-pat w (- (+) z) L2) = let-pat w (- (+) z)
L2]]$. It suffices to show that $[[ [t/x](let-pat w (y (+) -) L1) =
let-pat w (y (+) -) [t/x]L1]]$ but this easily follows from a simple
distributing the substitution into the let-pat, and then
simplifying using the fact that $w \neq x$.
% subsubsection left_introduction_of_par_(first_case) (end)
\end{report}
\begin{report}
  
\subsubsection{Left introduction of par (second case)}
\label{subsec:left_introduction_of_par_(second_case)}
The proof
\begin{center}
  \begin{math}
    $$\mprset{flushleft}
    \inferrule* [right=Cut] {
      \inferrule* [right=] {
        \inferrule* [right=,vdots=1.5em,fraction=\,] {
          \,
        }{\pi_1}          
      }{[[G |- t : A | L]]}      
      \\
      $$\mprset{flushleft}
      \inferrule* [right=Parl] {
        \inferrule* [right=] {
        \inferrule* [right=,vdots=1.5em,fraction=\,] {
          \,
        }{\pi_2}          
      }{[[G1,y : B |- L1]]}      
      \\
      \inferrule* [right=] {
        \inferrule* [right=,vdots=1.5em,fraction=\,] {
          \,
        }{\pi_3}          
      }{[[G2,x : A,G3,z : C |- L2]]}      
      }{[[G1,G2,x : A,G3,w : B (+) C |- h(let-pat w (y (+) -) L1) | h(let-pat w (- (+) z) L2)]]}
    }{[[G1,G2,G,G3,w : B (+) C |- L | h(h([t/x](let-pat w (y (+) -) L1)) | h([t/x](let-pat w (- (+) z) L2)))]]}
  \end{math}
\end{center}
transforms into the proof
\begin{center}
  \begin{math}
    $$\mprset{flushleft}
    \inferrule* [right=Parl] {
      \inferrule* [right=] {
        \inferrule* [right=,vdots=1.5em,fraction=\,] {
          \,
        }{\pi_2}          
      }{[[G1,y : B |- L1]]}      
      \\
      $$\mprset{flushleft}
      \inferrule* [right=Cut] {
        \inferrule* [right=] {
        \inferrule* [right=,vdots=1.5em,fraction=\,] {
          \,
        }{\pi_1}          
      }{[[G |- t : A | L]]}      
      \\
      \inferrule* [right=] {
        \inferrule* [right=,vdots=1.5em,fraction=\,] {
          \,
        }{\pi_3}          
      }{[[G2,x : A,G3,z : C |- L2]]}      
    }{[[G2,G,G3,z : C |- L | [t/x]L2]]}
    }{[[G1,G2,G,G3,w : B (+) C |- h(let-pat w (y (+) -) L1) | h(h(let-pat w (- (+) z) L) | h(let-pat w (- (+) z) [t/x]L2))]]}
  \end{math}
\end{center}

Similar to the previous case.
% subsubsection left_introduction_of_par_(second_case) (end)
\end{report}
\begin{report}
\subsubsection{Left introduction of implication (first case)}
\label{subsec:left_introduction_of_implication_(first_case)}
The proof
\begin{center}
  \begin{math}
    $$\mprset{flushleft}
    \inferrule* [right=Cut] {
      \inferrule* [right=] {
        \inferrule* [right=,vdots=1.5em,fraction=\,] {
          \,
        }{\pi_1}          
      }{[[G |- t : A | L]]}      
      \\
      $$\mprset{flushleft}
      \inferrule* [right=Impl] {
        \inferrule* [right=] {
          \inferrule* [right=,vdots=1.5em,fraction=\,] {
            \,
          }{\pi_2}          
        }{[[G1,x : A,G2 |- t1 : B | L1]]}      
        \\
        \inferrule* [right=] {
          \inferrule* [right=,vdots=1.5em,fraction=\,] {
            \,
          }{\pi_3}          
        }{[[G3,y : C |- L2]]}      
      }{[[G1,x:A,G2,G3,z : B -o C |- L1 | [z t1/y]L2]]}
    }{[[G1,G,G2,G3,z : B -o C |- L | h(h([t/x]L1) | [t/x]h([z t1/y]L2))]]}
  \end{math}
\end{center}
transforms into the proof
\begin{center}
  \begin{math}
    $$\mprset{flushleft}
    \inferrule* [right=Impl] {
      $$\mprset{flushleft}
      \inferrule* [right=Cut] {
        \inferrule* [right=] {
        \inferrule* [right=,vdots=1.5em,fraction=\,] {
          \,
        }{\pi_1}          
      }{[[G |- t : A | L]]}      
      \\
      \inferrule* [right=] {
          \inferrule* [right=,vdots=1.5em,fraction=\,] {
            \,
          }{\pi_2}          
        }{[[G1,x : A,G2 |- t1 : B | L1]]}      
      }{[[G1,G,G2 |- h(L | h{[t/x]t1} : B) | [t/x]L1]]}
      \\
      \inferrule* [right=] {
          \inferrule* [right=,vdots=1.5em,fraction=\,] {
            \,
          }{\pi_3}          
        }{[[G3,y : C |- L2]]}      
    }{[[G1,G,G2,G3, z : B -o C |- h(L | h([t/x]L1)) | [z {[t/x]t1}/y]L2]]}
  \end{math}
\end{center}
By inspection of the above derivations we can see that $[[x nin
FV(L2)]]$, and hence, by this fact and substitution distribution
(Lemma~\ref{lemma:substitution_distribution}) we know 
$[[ [t/x]h([z t1/y]L2) = h([{[t/x]z} {[t/x]t1}/y][t/x]L2)]] = [[h([z {[t/x]t1}/y]L2)]]$.
% subsubsection left_introduction_of_implication_(first_case) (end)

\subsubsection{Left introduction of implication (second case)}
\label{subsec:left_introduction_of_implication_(second_case)}
The proof
\begin{center}
  \begin{math}
    $$\mprset{flushleft}
    \inferrule* [right=Cut] {
      \inferrule* [right=] {
        \inferrule* [right=,vdots=1.5em,fraction=\,] {
          \,
        }{\pi_1}          
      }{[[G |- t : A | L]]}      
      \\
      $$\mprset{flushleft}
      \inferrule* [right=Impl] {
        \inferrule* [right=] {
          \inferrule* [right=,vdots=1.5em,fraction=\,] {
            \,
          }{\pi_2}          
        }{[[G1 |- t1 : B | L1]]}      
        \\
        \inferrule* [right=] {
          \inferrule* [right=,vdots=1.5em,fraction=\,] {
            \,
          }{\pi_3}          
        }{[[G2,x : A,G3,y : C |- L2]]}      
      }{[[G1,G2,x:A,G3,z : B -o C |- L1 | [z t1/y]L2]]}
    }{[[G1,G2,G,G3,z : B -o C |- L | h(h([t/x]L1) | [t/x]h([z t1/y]L2))]]}
  \end{math}
\end{center}
transforms into the proof
\begin{center}
  \begin{math}
    $$\mprset{flushleft}
    \inferrule* [right=\tiny Series of Exchanges] {
      $$\mprset{flushleft}
    \inferrule* [right=Impl] {
      \inferrule* [right=] {
        \inferrule* [right=,vdots=1.5em,fraction=\,] {
          \,
        }{\pi_2}          
      }{[[G1 |- t1 : B | L1]]}      
      \\
      $$\mprset{flushleft}
      \inferrule* [right=Cut] {
        \inferrule* [right=] {
        \inferrule* [right=,vdots=1.5em,fraction=\,] {
          \,
        }{\pi_1}          
      }{[[G |- t : A | L]]}      
      \\
      \inferrule* [right=] {
          \inferrule* [right=,vdots=1.5em,fraction=\,] {
            \,
          }{\pi_3}          
        }{[[G2,x : A,G3,y : C |- L2]]}      
      }{[[G2,G,G3,y : C |- L | [t/x]L2]]}
    }{[[G1,G2,G,G3,z : B -o C |- L1 | h(h([z t1/y]L) | h([z t1/y][t/x]L2))]]}
  }{[[G1,G2,G,G3,z : B -o C |- h(h([z t1/y]L) | h(L1 | h([z t1/y][t/x]L2)))]]}
  \end{math}
\end{center}
By inspection of the above proofs we can see that $[[y nin
FV(L)]]$. Thus, $[[ [z t1/y]L = L]]$.  The same can be said for the
variable $[[x]]$ and context $[[L1]]$, and hence, $[[ [t/x]L1 = L1]]$.
Finally, by inspection of the above proofs $[[x nin FV(t1)]]$ and so
by substitution distribution
(Lemma~\ref{lemma:substitution_distribution}) we know $[[ [t/x]h([z
t1/y]L2) = [z t1/y][t/x]L2]]$.
% subsubsection left_introduction_of_implication_(second_case) (end)

\subsubsection{Left introduction of implication (second case)}
\label{subsec:left_introduction_of_implication_(third_case)}
The proof
\begin{center}
  \begin{math}
    $$\mprset{flushleft}
    \inferrule* [right=Cut] {
      \inferrule* [right=] {
        \inferrule* [right=,vdots=1.5em,fraction=\,] {
          \,
        }{\pi_1}          
      }{[[G |- t : A | L]]}      
      \\
      $$\mprset{flushleft}
      \inferrule* [right=Impl] {
        \inferrule* [right=] {
          \inferrule* [right=,vdots=1.5em,fraction=\,] {
            \,
          }{\pi_2}          
        }{[[G1 |- t1 : B | L1]]}      
        \\
        \inferrule* [right=] {
          \inferrule* [right=,vdots=1.5em,fraction=\,] {
            \,
          }{\pi_3}          
        }{[[G2,y : C,G3,x : A |- L2]]}      
      }{[[G1,G2,z : B -o C,G3,x:A |- L1 | [z t1/y]L2]]}
    }{[[G1,G2,z : B -o C,G3,G |- L | h(h([t/x]L1) | [t/x]h([z t1/y]L2))]]}
  \end{math}
\end{center}
transforms into the proof
\begin{center}
  \begin{math}
    $$\mprset{flushleft}
    \inferrule* [right=\tiny Series of Exchanges] {
      $$\mprset{flushleft}
    \inferrule* [right=Impl] {
      \inferrule* [right=] {
        \inferrule* [right=,vdots=1.5em,fraction=\,] {
          \,
        }{\pi_2}          
      }{[[G1 |- t1 : B | L1]]}      
      \\
      $$\mprset{flushleft}
      \inferrule* [right=Cut] {
        \inferrule* [right=] {
        \inferrule* [right=,vdots=1.5em,fraction=\,] {
          \,
        }{\pi_1}          
      }{[[G |- t : A | L]]}      
      \\
      \inferrule* [right=] {
          \inferrule* [right=,vdots=1.5em,fraction=\,] {
            \,
          }{\pi_3}          
        }{[[G2,y : C,G3,x : A |- L2]]}      
      }{[[G2,y : C,G3,G |- L | [t/x]L2]]}
    }{[[G1,G2,z : B -o C,G3,G |- L1 | h(h([z t1/y]L) | h([z t1/y][t/x]L2))]]}
  }{[[G1,G2,z : B -o C,G3,G |- h(h([z t1/y]L) | h(L1 | h([z t1/y][t/x]L2)))]]}
  \end{math}
\end{center}
Similar to the previous case.
% subsubsection left_introduction_of_implication_(third_case) (end)

\subsubsection{Right introduction of implication}
\label{subsec:right_introduction_of_implication}
The proof
\begin{center}
  \begin{math}
    $$\mprset{flushleft}
    \inferrule* [right=Cut] {
      \inferrule* [right=] {
        \inferrule* [right=,vdots=1.5em,fraction=\,] {
          \,
        }{\pi_1}          
      }{[[G |- t : A | L]]}      
      \\
      $$\mprset{flushleft}
      \inferrule* [right=Impr] {
        \inferrule* [right=] {
        \inferrule* [right=,vdots=1.5em,fraction=\,] {
          \,
        }{\pi_2}          
      }{[[G1,x : A,G2,y : B |- t1 : C | L1]]}      
      \\
      [[y nin FV(L1)]]
      }{[[G1,x : A,G2 |- \y.t1 : B -o C | L1]]}
    }{[[G1,G,G2 |- L | h(h{[t/x]{\y.t1}} : B -o C | [t/x]L1)]]}
  \end{math}
\end{center}
transforms into the proof
\begin{center}
  \begin{math}
    $$\mprset{flushleft}
    \inferrule* [right=Impr] {
      $$\mprset{flushleft}
      \inferrule* [right=Cut] {
        \inferrule* [right=] {
          \inferrule* [right=,vdots=1.5em,fraction=\,] {
            \,
          }{\pi_1}          
        }{[[G |- t : A | L]]}      
        \\        
          \inferrule* [right=] {
            \inferrule* [right=,vdots=1.5em,fraction=\,] {
              \,
            }{\pi_2}          
          }{[[G1,x : A,G2,y : B |- t1 : C | L1]]}      
        }{[[G1,G,G2,y : B |- L | h(h{[t/x]t1} : C | [t/x]L1)]]}
      }{[[G1,G,G2 |- L | h(h{\y.[t/x]t1} : B -o C | [t/x]L1)]]}
    \end{math}
  \end{center}
Clearly, $[[ [t/x]{\y.t1} = \y.[t/x]t1]]$.
% subsubsection right_introduction_of_implication (end)

\subsubsection{Left introduction of tensor unit}
\label{subsec:left_introduction_of_tensor_unit}
The proof 
\begin{center}
  \begin{math}
    $$\mprset{flushleft}
    \inferrule* [right=Cut] {
      \inferrule* [right=] {
          \inferrule* [right=,vdots=1.5em,fraction=\,] {
            \,
          }{\pi_1}          
        }{[[G |- t : A | L]]}      
        \\
        $$\mprset{flushleft}
        \inferrule* [right=Il] {
          \inferrule* [right=] {
          \inferrule* [right=,vdots=1.5em,fraction=\,] {
            \,
          }{\pi_2}          
        }{[[G1, x : A, G2 |- L1]]}      
      }{[[G1,x:A,G2,y : I |- let y be stp in L1]]}
    }{[[G1,G,G2,y : I |- L | [t/x](let y be stp in L1)]]}
  \end{math}
\end{center}
transforms into the proof
\begin{center}
  \begin{math}
    $$\mprset{flushleft}
    \inferrule* [right=Il] {
      $$\mprset{flushleft}
      \inferrule* [right=Cut] {
        \inferrule* [right=] {
          \inferrule* [right=,vdots=1.5em,fraction=\,] {
            \,
          }{\pi_1}          
        }{[[G |- t : A | L]]}      
        \\
        \inferrule* [right=] {
          \inferrule* [right=,vdots=1.5em,fraction=\,] {
            \,
          }{\pi_2}          
        }{[[G1, x : A, G2 |- L1]]}      
      }{[[G1, G, G2 |- L | [t/x]L1]]}
    }{[[G1, G, G2, y : I |- h(let y be stp in L) | let y be stp in [t/x]L1]]}
  \end{math}
\end{center}
It suffices to show that $[[L]] = [[let y be stp in L]]$ and
$[[ [t/x](let y be stp in L1)]] = [[let y be stp in [t/x]L1]]$.
Without loss of generality suppose $[[L]] = [[t : B, L']]$.  We know
that it must be the case that $[[y nin FV(t)]]$, and we know that
$[[ [y/z]t]] = [[t]]$ when $[[z nin FV(t)]]$.  Then by
$\FILLdrulename{Eq\_Eta2I}$ we have $[[t]] = [[let y be stp in t]]$.  This
argument can be repeated for any other term in $[[L']]$.  Thus,
$[[L]] = [[let y be stp in L]]$.  It is easy to see that
$[[ [t/x](let y be stp in L1)]] = [[let y be stp in [t/x]L1]]$ using
the rule $\FILLdrulename{Eq\_NatI}$.
% subsubsection left_introduction_of_tensor_unit (end)

\subsubsection{Right introduction of par unit}
\label{subsec:right_introduction_of_par_unit}
The proof
\begin{center}
  \begin{math}
    $$\mprset{flushleft}
    \inferrule* [right=Cut] {
      \inferrule* [right=] {
        \inferrule* [right=,vdots=1.5em,fraction=\,] {
            \,
          }{\pi_1}          
        }{[[G |- t : A | L]]}      
        \\
        $$\mprset{flushleft}
        \inferrule* [right=Pr] {
          \inferrule* [right=] {
          \inferrule* [right=,vdots=1.5em,fraction=\,] {
            \,
          }{\pi_2}          
        }{[[G1,x : A, G2 |- L1]]}      
        }{[[G1,x : A, G2 |- o : _|_ | L1]]}
      }{[[G1,G, G2 |- L | h(h{[t/x]o} : _|_ | [t/x]L1)]]}
  \end{math}
\end{center}
transforms into the proof
\begin{center}
  \begin{math}
    $$\mprset{flushleft}
    \inferrule* [right=\tiny Series of Exchanges] {
      $$\mprset{flushleft}
    \inferrule* [right=Pr] {
      $$\mprset{flushleft}
      \inferrule* [right=Cut] {
        \inferrule* [right=] {
        \inferrule* [right=,vdots=1.5em,fraction=\,] {
            \,
          }{\pi_1}          
        }{[[G |- t : A | L]]}      
        \\
        \inferrule* [right=] {
          \inferrule* [right=,vdots=1.5em,fraction=\,] {
            \,
          }{\pi_2}          
        }{[[G1,x : A, G2 |- L1]]}      
      }{[[G1,G, G2 |- L | [t/x]L1]]}
    }{[[G1,G, G2 |- o : _|_ | h(L | [t/x]L1)]]}
  }{[[G1,G, G2 |- L | h(o : _|_ | [t/x]L1)]]}
  \end{math}
\end{center}
Clearly, $[[ [t/x]o = o]]$.
% subsubsection right_introduction_of_par_unit (end)

\subsubsection{Left introduction of exchange}
\label{subsec:left_introduction_of_exchange}
The proof
\begin{center}
  \begin{math}
    $$\mprset{flushleft}
    \inferrule* [right=Cut] {
      \inferrule* [right=] {
        \inferrule* [right=,vdots=1.5em,fraction=\,] {
            \,
          }{\pi_1}          
        }{[[G |- t : A | L]]}      
        \\
        $$\mprset{flushleft}
        \inferrule* [right=Exl] {
          \inferrule* [right=] {
        \inferrule* [right=,vdots=1.5em,fraction=\,] {
            \,
          }{\pi_2}          
        }{[[G1,x : A,G2, w : B, y : C, G3 |- L1]]}      
      }{[[G1,x : A,G2, y : C, w : B, G3 |- L1]]}
    }{[[G1,G,G2, y : C, w : B, G3 |- L | [t/x]L1]]}
  \end{math}
\end{center}
tranforms into the proof
\begin{center}
  \begin{math}
    $$\mprset{flushleft}
    \inferrule* [right=Exl] {
      $$\mprset{flushleft}
      \inferrule* [right=Cut] {
        \inferrule* [right=] {
        \inferrule* [right=,vdots=1.5em,fraction=\,] {
            \,
          }{\pi_1}          
        }{[[G |- t : A | L]]}      
        \\
        \inferrule* [right=] {
        \inferrule* [right=,vdots=1.5em,fraction=\,] {
            \,
          }{\pi_2}          
        }{[[G1,x : A,G2, w : B, y : C, G3 |- L1]]}      
      }{[[G1,G,G2, w : B, y : C, G3 |- L | [t/x]L1]]}
    }{[[G1,G,G2, y : C, w : B,G3 |- L | [t/x]L1]]}
  \end{math}
\end{center}
Clearly, all terms are equivalent.
% subsubsection left_introduction_of_exchange (end)

\subsubsection{Right introduction of exchange}
\label{subsec:right_introduction_of_exchange}
The proof
\begin{center}
  \begin{math}
    $$\mprset{flushleft}
    \inferrule* [right=Cut] {
      \inferrule* [right=] {
        \inferrule* [right=,vdots=1.5em,fraction=\,] {
            \,
          }{\pi_1}          
        }{[[G |- t : A | L]]}      
        \\
        $$\mprset{flushleft}
        \inferrule* [right=Exr] {
          \inferrule* [right=] {
        \inferrule* [right=,vdots=1.5em,fraction=\,] {
            \,
          }{\pi_2}          
        }{[[G1, x : A, G2 |- L1 | h(h(t1 : B | t2 : C) | L2)]]}      
      }{[[G1, x : A, G2 |- L1 | h(h(t2 : C | t1 : B) | L2)]]}
    }{[[G1, G, G2 |- L | h(h([t/x]L1) | h(h(h{[t/x]t2} : C | h{[t/x]t1} : B) | [t/x]L2))]]}
  \end{math}
\end{center}
is transformed into 
\begin{center}
  \begin{math}
    $$\mprset{flushleft}
    \inferrule* [right=Exr] {
      $$\mprset{flushleft}
      \inferrule* [right=Cut] {
        \inferrule* [right=] {
        \inferrule* [right=,vdots=1.5em,fraction=\,] {
            \,
          }{\pi_1}          
        }{[[G |- t : A | L]]}      
        \\
        \inferrule* [right=] {
        \inferrule* [right=,vdots=1.5em,fraction=\,] {
            \,
          }{\pi_2}          
        }{[[G1, x : A, G2 |- L1 | h(h(t1 : B | t2 : C) | L2)]]}      
      }{[[G1, G, G2 |- L | h(h([t/x]L1) | h(h(h{[t/x]t1} : B | h{[t/x]t2} : C) | [t/x]L2))]]}
    }{[[G1, G, G2 |- L | h(h([t/x]L1) | h(h(h{[t/x]t2} : C | h{[t/x]t1} : B) | [t/x]L2))]]}
  \end{math}
\end{center}
Clearly, all terms are equivalent.
% subsubsection right_introduction_of_exchange (end)
\end{report}
% subsection secondary_hypothesis (end)
% section cut-elimination (end)


\bibliographystyle{plain}
\bibliography{ref}

\appendix

\begin{report}
  \section{The full specification of FILL}
\label{sec:fill_specification}
\FILLall{}
% section the_full_fill_specification (end)
\end{report}

%%% Local Variables: 
%%% mode: latex
%%% TeX-master: t
%%% End: 