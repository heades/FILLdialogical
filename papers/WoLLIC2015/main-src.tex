\usepackage[utf8]{inputenc}
\usepackage{amssymb,amsmath,amsthm}
\usepackage{cmll}
\usepackage{stmaryrd}
\usepackage{todonotes}
\usepackage{mathpartir}
\usepackage{hyperref}
\usepackage[barr]{xy}
\usepackage{fullpage}
% Theorems
\newtheorem{theorem}{Theorem}
\newtheorem{lemma}[theorem]{Lemma}
\newtheorem{fact}[theorem]{Fact}
\newtheorem{corollary}[theorem]{Corollary}
\newtheorem{definition}[theorem]{Definition}
\newtheorem{remark}[theorem]{Remark}
\newtheorem{proposition}[theorem]{Proposition}
\newtheorem{notn}[theorem]{Notation}
\newtheorem{observation}[theorem]{Observation}

% Commands that are useful for writing about type theory and programming language design.
%% \newcommand{\case}[4]{\text{case}\ #1\ \text{of}\ #2\text{.}#3\text{,}#2\text{.}#4}
\newcommand{\interp}[1]{\llbracket #1 \rrbracket}
\newcommand{\normto}[0]{\rightsquigarrow^{!}}
\newcommand{\join}[0]{\downarrow}
\newcommand{\redto}[0]{\rightsquigarrow}
\newcommand{\nat}[0]{\mathbb{N}}
\newcommand{\fun}[2]{\lambda #1.#2}
\newcommand{\CRI}[0]{\text{CR-Norm}}
\newcommand{\CRII}[0]{\text{CR-Pres}}
\newcommand{\CRIII}[0]{\text{CR-Prog}}
\newcommand{\subexp}[0]{\sqsubseteq}
%% Must include \usepackage{mathrsfs} for this to work.
\newcommand{\powerset}[0]{\mathscr{P}}


\title{Multiple Conclusion Intuitionistic Linear Logic and Cut Elimination}
\author{Harley Eades III and Valeria de Paiva}
\date{}

% Ott includes.
\usepackage{color}
\usepackage{supertabular}

\input{FILL-ott}

% Renewing some Ott commands to shrink some of the labels.
\renewcommand{\FILLdrulename}[1]{\scriptsize \textsc{#1}}

% Cat commands.
\newcommand{\cat}[1]{\mathcal{#1}}
\newcommand{\limp}[0]{\multimap}
\newcommand{\dial}[0]{\mathsf{Dial_2}(\mathsf{Sets})}
\newcommand{\sets}[0]{\mathsf{Sets}}
\newcommand{\obj}[1]{\mathsf{Obj}(#1)}
\newcommand{\mor}[1]{\mathsf{Mor(#1)}}
\newcommand{\id}[0]{\mathsf{id}}
\newcommand{\lett}[0]{\mathsf{let}\,}
\newcommand{\inn}[0]{\,\mathsf{in}\,}

\begin{document}

\maketitle

\begin{abstract}
  Full Intuitionistic Linear Logic (FILL) was first introduced by
  Hyland and de Paiva as one of the results of their investigation
  into a categorical understanding of G\"odel’s Dialectica
  interpretation. FILL went against current beliefs that it was not
  possible to incorporate all of the linear connectives, e.g. tensor,
  par, and implication, into an intuitionistic linear logic. They
  showed that it is natural to support all of the connectives given
  sequents that have multiple hypotheses and multiple conclusions. To
  enforce intuitionism de Paiva’s original formalization of FILL used
  the well-known Dragalin restriction, forcing the implication right
  rule to have only a single conclusion in its premise, but Schellinx
  showed that this results in a failure of cut-elimination. To
  overcome this failure Hyland and de Paiva introduced a term
  assignment for FILL that eliminated the need for the strong
  restriction. The main idea was to first relax the restriction by
  assigning variables to each hypothesis and terms to each conclusion.
  Then when introducing an implication on the right enforcing that the
  variable annotating the hypothesis being discharged is only free in
  the term annotating the conclusion of the implication. Bierman
  showed potentially that this formalization of FILL still did not
  enjoy cut-elimination, because of a flaw in the left rule for
  par. However, Bellin proposed an alternate left rule for par and
  conjectured that by adopting his rule cut-elimination is
  restored. In this note we show that adopting Bellin’s proposed rule
  one does obtain cut-elimination for FILL, as
  suggested. Additionally, we show that a categorical model of FILL
  in the basic dialectica category is also a LNL model of Benton.
\end{abstract}

\section{Introduction}
\label{sec:introduction}

A commonly held belief during the early history of linear logic was
that the linear-connective par could not be incorporated into an
intuitionistic linear logic.  This belief was challenged when de Paiva
gave a categorical understanding of G\"odel's Dialectica
interpretation in terms of dialectica categories
\cite{dePaiva:1987,dePaiva:1988}.  Upon setting out on her
investigation she initially believed that dialectica categories would
end up being a model of intuitionistic logic, but to her surprise they
are actually models of intuitionistic linear logic, containing the
linear connectives: tensor, implication,additives, and exponentials.
She then improved her models to capture both FILL and CLL.
Furthermore, unlike other models at that time the units did not
collapse into a single object.

Armed with this semantic insight de Paiva gave the first formalization
of Full Intuitionistic Linear Logic (FILL) \cite{dePaiva:1988}.  FILL
is a sequent calculus with multiple conclusions in addition to
multiple hypotheses.  Logics of this type go back to Gentzen's work on
the sequent calculi LK and LJ, and Maehara's work on LJ'
\cite{Maehara:1954,Takeuti:1975}.  The sequents in these types of
logics usually have the form $\Gamma \vdash \Delta$ where $\Gamma$ and
$\Delta$ are multisets of formulas.  Sequents such as these are read
as ``the conjunction of the formulas in $\Gamma$ imply the disjunction
of the formulas in $\Delta$''.  For a brief, but more complete history
of logics with multiple conclusions see the introduction to
\cite{dePaiva:2005}.

Gentzen showed that to obtain intuitionistic logic one could start
with the logic LK and then place a cardinality restriction on the
right-hand side of sequents, however, this is not the only means of
enforcing intuitionism.  Maehara showed that in the propositional case
one could simply place the cardinality restriction on the premise of
the implication right rule, and leave all of the other rules of LK
unrestricted.  This restriction is sometimes called the Dragalin
restriction, as it appeared in his AMS textbook \cite{Dragalin:1988}.
The classical implication right rule has the form:
\begin{center}
  \begin{math}
    $$\mprset{flushleft}
    \inferrule* [right=impR] {
      [[G]], [[A]] \vdash [[B]], [[L]]
    }{[[G]] \vdash [[A -o B]], [[L]]}
  \end{math}
\end{center}
By placing the Dragalin restriction on the previous rule we obtain:
\begin{center}
  \begin{math}
    $$\mprset{flushleft}
    \inferrule* [right=impR] {
      [[G]], [[A]] \vdash [[B]]
    }{[[G]] \vdash [[A -o B]]}
  \end{math}
\end{center}
de Paiva's first formalization of FILL used the Dragalin restriction,
see \cite{dePaiva:1988} p. 58, but Schellinx showed that this restriction has
the unfortunate consequence of breaking cut-elimination
\cite{Schellinx:1991}.

Later, Hyland and de Paiva gave an alternative formalization of FILL
in to regain cut-elimination \cite{Hyland:1993}.  This new
formalization lifted the Dragalin restriction by decorating sequents
with a term assignment.  Hypotheses were assigned variables, and the
conclusions were assigned terms.  Then using these terms one can track
the use of hypotheses throughout a derivation.  They proposed a new
implication right rule:
\begin{center}
  \begin{math}
    $$\mprset{flushleft}
    \inferrule* [right=impR] {
      [[G, x : A |- t : B,L]]
      \\
      [[x nin FV(L)]]
    }{[[G |- \x.t : A -o B, L]]}
  \end{math}
\end{center}
Intuitionism is enforced in this rule by requiring that the variable
being dischanged, $x$,
is potentially free in only one term annotating a conclusion.
Unfortunately, this formalization did not enjoy cut-elimination
either.

Bierman was able to give a counterexample to cut-elimination
\cite{Bierman:1996}.  As Bierman explains the problem was with the
left rule for par.  The original rule was as follows:
\begin{center}
  \begin{math}
    $$\mprset{flushleft}
    \inferrule* [right=parL] {
      [[G, x : A |- L]] 
      \\
      [[G', y : B |- L']]
    }{[[G,G',z : A (+) B |- h(let z be (x (+) -) in L) | h(let z be (- (+) y) in L')]]}
  \end{math}
\end{center}
In this rule the pattern variables $x$ and $y$ are bound in each term
of $[[L]]$ and $[[L']]$ respectively. Notice that the variable $z$
becomes free in every term in $[[L]]$ and $[[L']]$. Bierman showed
that this rule mixed with the restriction on implication right
prevents the usual cut-elimination step that commutes cut with the
left rule for par.  The main idea behind the counterexample is that in
the derivation before commuting the cut it is possible to discharge
$z$ using implication right, but after the cut is commuted past the
left rule for par, the variable $z$ becomes free in more than one
conclusion, and thus, can no longer be discharged.

In the conclusion of Bierman's note he gives an alternate left rule
for par that he attributes to Bellin.  This new left-rule is as
follows:
\begin{center}
  \begin{math}
    \FILLdruleParl{}
  \end{math}
\end{center}
In this rule $[[let-pat z (x (+) -) t]]$ and $[[let-pat z (- (+) y) t']]$ only let-bind
$z$ in $t$ or $t'$ if $x \in FV(t)$ or $y \in FV(t')$.  Otherwise the
terms are left unaltered.  Bellin conjectured that adopting this rule
results in FILL regaining cut-elimination.  However, no proof has been
given.  

\textbf{Contributions.} In this paper our main contribution is to
confirm Bellin's conjecture by adopting his proposed rule
(Section~\ref{sec:full_intuitionistic_linear_logic_(fill)}) and
proving cut-elimination (Section~\ref{sec:cut-elimination}).  In
addition, we show that the categorical model of FILL called $\dial$, a
the basic dialectica category, is also a linear/non-linear model of
Benton (Section~\ref{sec:full_lnl_models}).

\textbf{Related Work.} The first formalization of FILL with
cut-elimination was due to Bra\"uner and de Paiva \cite{Brauner:1998}.
Their formalization can be seen as a linear version of LK with a
sophisticated meta-level dependency tracking system.  A proof of a
FILL sequent in their formalization amounts to a classical derivation,
$\pi$, invariant in a what they call the FILL property:
\begin{center}
  \begin{itemize}
  \item The hypothesis discharged by an application of the implication
    right rule in $\pi$ is a dependency of the conclusion of the
    implication being introduced.
  \end{itemize}
\end{center}
They were able to show that their formalization is sound, complete,
and enjoys cut-elimination.  In favor of the term assignment
formalization given here over Brauner and de Paiva's formalization is
that the dependency tracking system complicates both the definition of
the logic and its use.  However, one might conjecture that their
system is more fundamental and hence more generalizable.

de Paiva and Pereira used annotations on the sequents of LK to arrive
at full intuitionistic logic (FIL) with multiple conclusion that
enjoys cut-elimination \cite{dePaiva:2005}. They annotate hypothesis
with natural number indices, and conclusions with finite sets of
indices.  The sets of indices on conclusions correspond to the
collection of the hypotheses that the conclusion depends on.  Then
they have a similar property to that of Bra\"uner and de Paiva's
formalization.  In fact, the dependency tracking system is very
similar to this formalization, but the dependency tracking has been
collapsed into the object language instead of being at the meta-level.
% section introduction (end)

\section{Full Intuitionistic Linear Logic (FILL)}
\label{sec:full_intuitionistic_linear_logic_(fill)}

In this section we give a brief description of FILL.  We first give
the syntax of formulas, patterns, terms, and contexts.  Following the
syntax we define several meta-functions that will be used when
defining the inference rules of the logic.

\begin{definition}
  \label{def:syntax}
  The syntax for FILL is as follows:
  \begin{center}
    \begin{math}
      \begin{array}{cll}
        \text{(Formulas)}       & [[A]], [[B]], [[C]], [[D]], [[E]] ::= [[I]] \mid [[_|_]]
        \mid [[A -o B]] \mid [[A (x) B]] \mid [[A (+) B]] \\
        \text{(Patterns)} & [[p]] ::= [[stp]] \mid [[-]] \mid [[x]] \mid [[p1 (x)
        p2]] \mid [[p1 (+) p2]]\\
        \text{(Terms)}          & [[t]], [[e]] ::= [[x]] \mid [[*]] \mid [[o]] \mid
        [[t1 (x) t2]] \mid [[t1 (+) t2]] \mid [[\x.t]] \mid [[let t be p in e]] \mid [[t1 t2]]\\
        \text{(Left Contexts)}  & [[G]] ::= [[.]] \mid [[x : A]] \mid [[G1,G2]]\\
        \text{(Right Contexts)} & [[L]] ::= [[.]] \mid [[t : A]] \mid [[L1,L2]]\\
      \end{array}
    \end{math}
  \end{center}
\end{definition}

The formulas of FILL are standard, but we denote the unit of tensor as
$\top$ and the unit of par as $\perp$. Patterns are used to
distinguish between the various let-expressions for tensor, par, and
their units.  There are three different let-expressions:
\begin{center}
  \begin{math}
    \begin{array}{lll}
      \begin{array}{lll}
        \text{Tensor:}\\
        \,\,\,\,[[let t be p1 (x) p2 in e]]
      \end{array}
      &
      \begin{array}{lll}
        \text{Par:}\\
        \,\,\,\,[[let t be p1 (+) p2 in e]]
      \end{array}
      &
      \begin{array}{lll}
        \text{Tensor Unit:}\\
        \,\,\,\,[[let t be stp in e]]
      \end{array}
    \end{array}
    %% \begin{array}{rll}
    %%   \text{Tensor:} & [[let t be p1 (x) p2 in e]]\\
    %%   \text{Par:} & [[let t be p1 (+) p2 in e]]\\
    %%   \text{Tensor Unit:} & [[let t be stp in e]]\\
    %% \end{array}
  \end{math}
\end{center}
In addition, each of these will have their own equational rules, see
Figure~\ref{def:FILL-eq}.  The role each term plays in the overall
logic will become clear after we introduce the inference rules.

At this point we introduce some syntax and meta-level functions that
will be used in the definition of the inference rules for FILL. Left
contexts are multisets of formulas labeled with a variable, and right
contexts are multisets of formulas labeled with a term.  We will often
write $[[L1 | L2]]$ as syntactic sugar for $[[L1,L2]]$.  The former
should be read as ``$[[L1]]$ or $[[L2]]$.''  We denote the usual
capture-avoiding substitution by $[[ [t/x]t']]$, and its
straightforward extension to right contexts as $[[ [t/x]L]]$.
% \begin{definition}
%   \label{def:delta-sub}
%   We extend the capture-avoiding substitution function to right
%   contexts as follows:
%   \begin{center}
%     \begin{math}
%       \begin{array}{lll}
%         [[ [t/x] .]] = [[.]]\\
%         [[ [t/x] (t' : A)]] = [[{[t/x]t'} : A]]\\
%         [[ [t/x] (L1 | L2)]] = [[([t/x]L1) | ([t/x]L2)]]\\
%       \end{array}
%     \end{math}
%   \end{center}
% \end{definition}
Similarly, we find it convenient to be able to do this style of
extension for the let-binding as well.
\begin{definition}
  \label{def:delta-let}
  We extend let-binding terms to right contexts as follows:
  \begin{center}
    \begin{math}
      \begin{array}{lll}
        [[ let t be p in .]] = [[.]]\\
        [[ let t be p in (t' : A)]] = [[{let t be p in t'} : A]]\\
        [[ let t be p in (L1 | L2)]] = [[(let t be p in L1) | (let t be p in L2)]]\\
      \end{array}
    \end{math}
  \end{center}
\end{definition}
\noindent
Lastly, we denote the usual function that computes the set of free
variables in a term by $\mathsf{FV}([[t]])$, and its straightforward
extension to right contexts as $\mathsf{FV}([[L]])$.
% \begin{definition}
%   \label{def:delta-FV}
%   We extend the free-variable function on terms to right contexts as
%   follows:
%   \begin{center}
%     \begin{math}
%       \begin{array}{lll}
%         [[FV(.)]] = \emptyset\\
%         [[FV(t : A)]] = [[FV(t)]]\\
%         [[FV(L1 | L2)]] = [[FV(L1)]] \cup [[FV(L2)]]\\
%       \end{array}
%     \end{math}
%   \end{center}
% \end{definition}

The inference rules for FILL are defined in Figure~\ref{def:infr}.
\begin{figure}
    \begin{center}
    \scriptsize
      \begin{mathpar}
        \FILLdruleAx{}    \and 
        \FILLdruleCut{}     \and 
        \FILLdruleIl{}            \and 
        \FILLdruleIr{}    \and 
        \FILLdruleTl{}    \and 
        \FILLdruleTr{}    \and 
        \FILLdrulePl{}    \and 
        \FILLdrulePr{}    \and 
        \FILLdruleParl{}    \and 
        \FILLdruleParr{}    \and 
        \FILLdruleImpl{}    \and 
        \FILLdruleImpr{}    \and 
        \FILLdruleExl{}    \and 
        \FILLdruleExr{}    
    \end{mathpar}
  \end{center}
  \caption{Inference rules for FILL}
  \label{def:infr}
\end{figure}
The $\FILLdrulename{Parl}$ rule depends on the function $[[let-pat z p
L]]$ which we define next.
\begin{definition}
  \label{def:let-pat-term}
  The function $[[let-pat z p t]]$ is defined as follows:
  \begin{center}
    \begin{math}
      \begin{array}{lll}      
        \begin{array}{lll}
          [[let-pat z (x (+) -) t]] = [[t]]\\
          \,\,\,\,\,\,\text{where } [[x]] \not\in \mathsf{FV}([[t]])\\
        \end{array}
        & 
          \begin{array}{lll}
            [[let-pat z (- (+) y) t]] = [[t]]\\
        \,\,\,\,\,\,\text{where } [[y]] \not\in \mathsf{FV}([[t]])\\
          \end{array}
        & 
          \begin{array}{lll}
            [[let-pat z p t]] = [[let z be p in t]]\\
            & \\
          \end{array}
      \end{array}
    \end{math}
  \end{center}
  It is straightforward to extend the previous definition to
  right-contexts, and we denote this extension by $[[let-pat z p L]]$.
  % \begin{center}
  %   \begin{math}
  %     \begin{array}{lll}      
  %       [[let-pat z p .]] = [[.]]\\                
  %       [[let-pat z p (t : A)]] = [[{let-pat z p t} : A]]\\
  %       [[let-pat z p (L1 | L2)]] = [[(let-pat z p L1) | (let-pat z p L2)]]\\
  %     \end{array}
  %   \end{math}
  % \end{center}
\end{definition}
The motivation behind this function is that it only binds the pattern
variables in $[[x (+) -]]$ and $[[- (+) y]]$ if and only if those
pattern variables are free in the body of the let.  This over comes the
counterexample given by Bierman in \cite{Bierman:1996}.  

The terms of FILL are equipped with an equivalence relation defined in
Figure~\ref{def:FILL-eq}.
\begin{figure}[t]
  \begin{center}
    \footnotesize
    \begin{mathpar}
      \FILLdruleAlpha{} \and
      \FILLdruleEtaFun{} \and
      \FILLdruleBetaFun{} \and
      \FILLdruleEtaOneI{} \and
\begin{report}
  \FILLdruleEtaTwoI{} \and
\end{report}
      \FILLdruleBetaI{} \and
      \FILLdruleNatI{} \and
      \begin{report}
        \FILLdruleEtaTen{} \and
      \end{report}
      \FILLdruleBetaOneTen{} \and
      \FILLdruleBetaTwoTen{} \and
      \FILLdruleNatTen{} \and
      \FILLdruleEtaParU{} \and
      \FILLdruleEtaPar{} \and
      \FILLdruleBetaOnePar{} \and
      \FILLdruleBetaTwoPar{} \and
      \FILLdruleNatOnePar{} \and
      \FILLdruleNatTwoPar{} 
      \begin{report}
        \and
      \FILLdruleLam{} \and
      \FILLdruleAppOne{} \and
      \FILLdruleAppTwo{} \and
      \FILLdruleTenOne{} \and
      \FILLdruleTenTwo{} \and
      \FILLdruleParOne{} \and
      \FILLdruleParTwo{} \and
      \FILLdruleLetOne{} \and
      \FILLdruleLetTwo{} \and
      \FILLdruleRefl{} \and
      \FILLdruleSym{} \and
      \FILLdruleTrans{}
      \end{report}
    \end{mathpar}
  \end{center}
  \caption{Equivalence on terms}
  \label{def:FILL-eq}
\end{figure}
There are a number of $\alpha$, $\beta$, and $\eta$ like rules as well
as several rules we call naturality rules.  These rules are similar to
the rules presented in \cite{Hyland:1993}.
% section full_intuitionistic_linear_logic_(fill) (end)

\section{Cut-elimination}
\label{sec:cut-elimination}

FILL can be viewed from two different angles: i. as an intuitionistic
linear logic with par, or ii. as a restricted form of classical linear
logic.  Thus, to prove cut-elimination of FILL one only need to start
with the cut-elimination procedure for intuitionistic linear logic,
and then dualize all of the steps in the procedure for tensor and its
unit to obtain the steps for par and its unit.  Similarly, one could
just as easily start with the cut-elimination procedure for classical
linear logic, and then apply the restriction on the implication right
rule producing a cut-elimination procedure for FILL.

The major difference between proving cut-elimination of FILL from
classical or intuitionistic linear logic is that we must prove an
invariant across each step in the procedure.  The invariant is that if
a derivation $\pi$ is transformed into a derivation $\pi'$, then the
terms in the conclusion of the final rule applied in $\pi$ must be
equivalent to the terms in the conclusion of the final rule applied in
$\pi'$ using the rules from Figure~\ref{def:FILL-eq}.


\begin{comment}
  
For example,
consider the following case in the cut-elimination procedure for FILL:

\ \\
\noindent
The proof
\begin{center}
  \scriptsize
  \begin{math}
    $$\mprset{flushleft}
    \inferrule* [right=\tiny Cut] {
      \inferrule* [right=] {
        \inferrule* [right=,vdots=1.5em,fraction=\,] {
          \,
        }{\pi_1}          
      }{[[G |- t : A | L]]}      
      \\
      $$\mprset{flushleft}
      \inferrule* [right=\tiny Parr] {
        \inferrule* [right=] {
        \inferrule* [right=,vdots=1.5em,fraction=\,] {
          \,
        }{\pi_2}          
      }{[[G1,x : A,G2 |- h(L1 | h(t1 : B | t2 : C)) | L2]]}                  
    }{[[G1,x : A,G2 |- h(L1 | h(t1 (+) t2 : B (+) C)) | L2]]}
  }{[[G1,G,G2 |- L | h(h(h([t/x]L1) | h([t/x]{t1 (+) t2} : B (+) C)) | [t/x]L2)]]}
  \end{math}
\end{center}
transforms into the proof
\begin{center}
  \scriptsize
  \begin{math}
    $$\mprset{flushleft}
\inferrule* [right=\tiny Parr] {
  $$\mprset{flushleft}
  \inferrule* [right=\tiny Cut] {
    \inferrule* [right=] {
        \inferrule* [right=,vdots=1.5em,fraction=\,] {
          \,
        }{\pi_1}          
      }{[[G |- t : A | L]]}      
      \\
      \inferrule* [right=] {
        \inferrule* [right=,vdots=1.5em,fraction=\,] {
          \,
        }{\pi_2}          
      }{[[G1,x : A,G2 |- h(L1 | h(t1 : B | t2 : C)) | L2]]}                  
    }{[[G1,G,G2 |- L | h(h(h([t/x]L1) | h(h{[t/x]t1} : B | h{[t/x]t2} : C)) | [t/x]L2)]]}
  }{[[G1,G,G2 |- L | h(h(h([t/x]L1) | h{[t/x]t1} (+) h{[t/x]t2} : B (+) C) | [t/x]L2)]]}
  \end{math}
\end{center}
Now we must show that $[[L]] = [[L]]$, $[[ [t/x]L1]] = [[ [t/x]L1]]$,
$[[ [t/x]{t1 (+) t2} = {[t/x]t1} (+) [t/x]t2]]$, and
$[[ [t/x]L2]] = [[ [t/x]L2]]$, but the only non-trivial case is
whether $[[ [t/x]{t1 (+) t2} = {[t/x]t1} (+) [t/x]t2]]$ holds, but
this clearly holds by a simple property of capture avoiding
substitution.  Every case of the cut-elimination procedure proceeds
just as this example does.
\end{comment}
\begin{report}
  The cut elimination procedure requires the following two basic
results:
\begin{lemma}[Substitution Distribution]
  \label{lemma:substitution_distribution}
  For any terms $[[t]]$, $[[t1]]$, and $[[t2]]$, $[[ [t1/x][t2/y]t]] = [[ [ [t1/x]t2/y][t2/x]t]]$.
\end{lemma}
\begin{proof}
  This proof holds by straightforward induction on the form of $t$.
\end{proof}

\begin{lemma}[Let-pat Distribution]
  \label{lemma:let-pat_distribution}
  For any terms $[[t]]$, $[[t1]]$, and $[[t2]]$, and pattern p, \\
  $[[ let-pat t p [t1/y]t2]] = [[ [ let-pat t p t1/y]t2]]$.
\end{lemma}
\begin{proof}
  This proof holds by case splitting over $p$, and then using the
  naturality equations for the respective pattern.
\end{proof}
\end{report}

We finally arrive at cut-elimination.
\begin{theorem}
  \label{thm:cut-pro}
  If $[[G |- t1 : A1,...,ti : Ai]]$ steps to $[[G |- t'1 : A1,...,t'i
  : Ai]]$ using the cut-elimination procedure, then $[[tj = t'j]]$
  for $1 \leq j \leq i$.
\end{theorem}
\begin{proof}
  The cut-elimination procedure given here is the standard
  cut-elimination procedure for classical linear logic except the
  cases involving the implication right rule have the FILL
  restriction. The structure of our procedure follows the structure of
  the procedure found in \cite{Mellies:2009}. Due to space limitations
  we only show one of the most interesting cases, but for the entire
  proof see the companion report \cite{Eades:2015}.
  \begin{itemize}
    \begin{report}      
  \item[Case:] commuting conversion cut vs cut (first case).
    The following proof
\begin{center}
  \begin{math}
    $$\mprset{flushleft}
    \inferrule* [right=Cut] {
      $$\mprset{flushleft}
      \inferrule* [right=] {
        \inferrule* [right=,vdots=1.5em,fraction=\,] {
          \,
        }{\pi_1}
      }{[[G |- t : A | L]]}
      \\
      $$\mprset{flushleft}
      \inferrule* [right=Cut] {
        $$\mprset{flushleft}
        \inferrule* [right=] {
          \inferrule* [right=,vdots=1.5em,fraction=\,] {
            \,
          }{\pi_2}
        }{[[G2,x : A,G3 |- t1 : B | L1]]}
        \\
        $$\mprset{flushleft}
        \inferrule* [right=] {
          \inferrule* [right=,vdots=1.5em,fraction=\,] {
            \,
          }{\pi_3}
        }{[[G1,y : B,G4 |- L2]]}
      }{[[G1,G2,x : A,G3,G4 |- h(L1 | [t1/y]L2)]]}
    }{[[G1,G2,G,G3,G4 |- h(L | [t/x]L1) | [t/x]h([t1/y]L2)]]}
  \end{math}
\end{center}
is transformed into the proof
\begin{center}
  \begin{math}
    $$\mprset{flushleft}
    \inferrule* [right=Cut] {
      $$\mprset{flushleft}
      \inferrule* [right=] {
        $$\mprset{flushleft}
      \inferrule* [right=] {
        \inferrule* [right=,vdots=1.5em,fraction=\,] {
          \,
        }{\pi_1}               
      }{[[G |- t : A | L]]}
      \\
      $$\mprset{flushleft}
        \inferrule* [right=] {
          \inferrule* [right=,vdots=1.5em,fraction=\,] {
            \,
          }{\pi_2}
        }{[[G2,x : A,G3 |- t1 : B | L1]]}
      }{[[G2,G,G3 |- h{[t/x]t1} : B | [t/x]L1]]}
      \\
      $$\mprset{flushleft}
        \inferrule* [right=] {
          \inferrule* [right=,vdots=1.5em,fraction=\,] {
            \,
          }{\pi_3}
        }{[[G1,y : B,G4 |- L2]]}
    }{[[G1,G2,G,G3,G4 |- h(L | [t/x]L1) | [ [t/x]t1/y]L2]]}
  \end{math}
\end{center}
First, if $[[L2]]$ is empty, then all the terms in the conclusion of
the previous two derivations are equivalent.  
So suppose $[[L2]] = [[t2 : C | L2']]$.  Then we know that the term
$[[ [t/x][t1/y]t2]]$ in the first derivation above is equivalent to
$[[ [ [t/x] t1/y][t/x] t2]]$ by
Lemma~\ref{lemma:substitution_distribution}.  Furthermore, by
inspecting the first derivation we can see that $[[x nin FV(t2)]]$,
and thus, $[[ [ [t/x] t1/y][t/x] t2 = [ [t/x] t1/y] t2]]$.  This
argument may be repeated for any term in $[[L2']]$, and thus, we know
$[[ [t/x][t1/y]L2 = [ [t/x]t1/y]L2]]$.

\item[Case:] commuting conversion cut vs. cut (second case).  The second commuting conversion on cut begins with the proof
\begin{center}
  \begin{math}
    $$\mprset{flushleft}
    \inferrule* [right=Cut] {
      $$\mprset{flushleft}
      \inferrule* [right=] {
        \inferrule* [right=,vdots=1.5em,fraction=\,] {
          \,
        }{\pi_1}
      }{[[G |- t : A | L]]}
      \\
      $$\mprset{flushleft}
      \inferrule* [right=Cut] {
        $$\mprset{flushleft}
        \inferrule* [right=] {
          \inferrule* [right=,vdots=1.5em,fraction=\,] {
            \,
          }{\pi_2}
        }{[[G' |- t' : B | L']]}
        \\
        $$\mprset{flushleft}
        \inferrule* [right=] {
          \inferrule* [right=,vdots=1.5em,fraction=\,] {
            \,
          }{\pi_3}
        }{[[G1,x : A,G2,y:B,G3 |- L1]]}
      }{[[G1,x : A,G2,G',G3 |- L' | [t'/y]L1]]}
    }{[[G1,G,G2,G',G3 |- h(L | [t/x]L') | [t/x]h([t'/y]L1)]]}
  \end{math}
\end{center}
is transformed into the following proof:
\begin{center}
  \begin{math}
    $$\mprset{flushleft}
    \inferrule* [right=\text{\scriptsize Series of Exchanges}] {
      $$\mprset{flushleft}
    \inferrule* [right=Cut] {
      $$\mprset{flushleft}
      \inferrule* [right=] {
        \inferrule* [right=,vdots=1.5em,fraction=\,] {
          \,
        }{\pi_2}
      }{[[G' |- t' : B | L']]}
      \\
      $$\mprset{flushleft}
      \inferrule* [right=Cut] {
        $$\mprset{flushleft}
        \inferrule* [right=] {
          \inferrule* [right=,vdots=1.5em,fraction=\,] {
            \,
          }{\pi_1}
        }{[[G |- t : A | L]]}      
        \\
        $$\mprset{flushleft}
        \inferrule* [right=] {
          \inferrule* [right=,vdots=1.5em,fraction=\,] {
            \,
          }{\pi_3}
        }{[[G1,x : A,G2,y:B,G3 |- L1]]}
      }{[[G1,G,G2,y:B,G3 |- L | [t/x]L1]]}
    }{[[G1,G,G2,G',G3 |- h(L' | [t'/y]L) | [t'/y]h([t/x]L1)]]}
    }{[[G1,G,G2,G',G3 |- h(h([t'/y]L) | L') | [t'/y]h([t/x]L1)]]}
  \end{math}
\end{center}
We know $[[x,y
nin
FV(L)]]$ by inspection of the first derivation, and so we know that
$[[L
= [t'/y]L]]$ and $[[L' =
[t/x]L']]$.  Without loss of generality suppose $[[L1]] = [[t1 : C |
L'1]]$. Then we know that $[[x,y nin FV(t)]]$ and $[[x,y nin
FV(t')]]$.  Thus, by this fact and
Lemma~\ref{lemma:substitution_distribution}, we know that $[[
[t/x][t'/y]t1 = [ [t/x]t'/y][t/x]t1]] = [[ [t'/y][t/x]t1
]]$.  This argument can be repeated for any term in
$[[L'1]]$, hence, $[[ [t/x][t'/y]L1 = [t'/y][t/x]L1]]$.

\item[Case:] the $\eta$-expansion cases: tensor.  
The proof
\begin{center}
  \begin{math}
    $$\mprset{flushleft}
    \inferrule* [right=Ax] {
      \,
    }{[[x : A (x) B |- x : A (x) B]]}
  \end{math}
\end{center}
is transformed into the proof
\begin{center}
  \begin{math}
    $$\mprset{flushleft}
    \inferrule* [right=Tl] {
      $$\mprset{flushleft}
      \inferrule* [right=Tr] {
        $$\mprset{flushleft}
        \inferrule* [right=Ax] {
          \,
        }{[[y : A |- y : A]]}
        \\
        $$\mprset{flushleft}
        \inferrule* [right=Ax] {
          \,
        }{[[z : B |- z : B]]}
      }{[[y : A, z : B |- y (x) z : A (x) B]]}
    }{[[x : A (x) B |- h{let x be y (x) z in {y (x) z}} : A (x) B]]}
  \end{math}
\end{center}
By the rule $\FILLdrulename{EQ\_EtaTensor}$
we know $[[let x be y (x) z in {y (x) z} = x]]$.

\item[Case:] the $\eta$-expansion cases: par.  
The proof
\begin{center}
  \begin{math}
    $$\mprset{flushleft}
    \inferrule* [right=Ax] {
      \,
    }{[[x : A (+) B |- x : A (+) B]]}
  \end{math}
\end{center}
is transformed into the proof
\begin{center}
  \begin{math}
    $$\mprset{flushleft}
    \inferrule* [right=Parr] {
      $$\mprset{flushleft}
      \inferrule* [right=Parl] {
        $$\mprset{flushleft}
        \inferrule* [right=Ax] {
          \,
        }{[[y : A |- y : A]]}
        \\
        $$\mprset{flushleft}
        \inferrule* [right=Ax] {
          \,
        }{[[z : B |- z : B]]}
      }{[[x : A (+) B |- h{let x be (y (+) -) in y} : A | h{let x be (- (+) z) in z} : B]]}
    }{[[x : A (+) B |- {let x be (y (+) -) in y} (+) {let x be (- (+) z) in z} : A (+) B]]}
  \end{math}
\end{center}
By rule $\FILLdrulename{Eq\_EtaPar}$ we know
$[[{{let x be (y (+) -) in y} (+) {let x be (- (+) z) in z}} = x]]$.

\item[Case:] the $\eta$-expansion cases: implication.
The proof
\begin{center}
  \begin{math}
    $$\mprset{flushleft}
    \inferrule* [right=Ax] {
      \,
    }{[[x : A -o B |- x : A -o B]]}
  \end{math}
\end{center}
transforms into the proof 
\begin{center}
  \begin{math}
    $$\mprset{flushleft}
    \inferrule* [right=ImpR] {
      $$\mprset{flushleft}
      \inferrule* [right=ImpL] {
        $$\mprset{flushleft}
        \inferrule* [right=Ax] {
          \,
        }{[[y : A |- y : A]]}
        \\
        $$\mprset{flushleft}
        \inferrule* [right=Ax] {
          \,
        }{[[z : B |- z : B]]}
      }{[[y : A, x : A -o B |- x y : B]]}
    }{[[x : A -o B |- \y.h{x y} : A -o B]]}
  \end{math}  
\end{center}
All terms in the two derivations are equivalent, because
$[[{\y.h{x y}} = x]]$ by the $\FILLdrulename{Eq\_EtaFun}$ rule.

\item[Case:] the $\eta$-expansion cases: tensor unit.
The proof
\begin{center}
  \begin{math}
    $$\mprset{flushleft}
    \inferrule* [right=Ax] {
      \,
    }{[[x : I |- x : I]]}
  \end{math}
\end{center}
transforms into the proof
\begin{center}
  \begin{math}
    $$\mprset{flushleft}
    \inferrule* [right=Il] {
      $$\mprset{flushleft}
      \inferrule* [right=Ir] {
        \,
      }{[[. |- * : I]]}
    }{[[x : I |- h{let x be stp in *} : I]]}
  \end{math}
\end{center}
We know $[[x = let x be stp in *]]$ by
$\FILLdrulename{Eq\_EtaI}$.  

\item[Case:] the $\eta$-expansion cases: par unit.
  The proof
\begin{center}
  \begin{math}
    $$\mprset{flushleft}
    \inferrule* [right=Ax] {
      \,
    }{[[x : _|_ |- x : _|_]]}
  \end{math}
\end{center}
transforms into the proof
\begin{center}
  \begin{math}
    $$\mprset{flushleft}
    \inferrule* [right=Pr] {
      $$\mprset{flushleft}
      \inferrule* [right=Pl] {
        \,
      }{[[x : _|_ |- .]]}
    }{[[x : _|_ |- o : _|_]]}
  \end{math}
\end{center}
We know $[[x = o]]$ by $\FILLdrulename{Eq\_EtaParU}$.

\item[Case:] the axiom steps: the axiom step.
The proof 
\begin{center}
  \begin{math}
    $$\mprset{flushleft}
    \inferrule* [right=Cut] {
      $$\mprset{flushleft}
      \inferrule* [right=Ax] {
        \,
      }{[[x : A |- x : A]]}
      \\
      $$\mprset{flushleft}
      \inferrule* [right=] {
        \inferrule* [right=,vdots=1.5em,fraction=\,] {
            \,
          }{\pi}          
      }{[[G1,y : A, G2 |- L]]}
    }{[[G1,x : A, G2 |- [x/y]L]]}
  \end{math}
\end{center}
transforms into the proof
\begin{center}
  \begin{math}
    $$\mprset{flushleft}
      \inferrule* [right=] {
        \inferrule* [right=,vdots=1.5em,fraction=\,] {
            \,
          }{\pi}          
      }{[[G1,y : A, G2 |- L]]}
  \end{math}
\end{center}
By $\FILLdrulename{Eq\_Alpha}$, we know, for any $[[t]]$ in
$[[L]]$, $[[t = [x/y]t]]$, and hence $[[L = [x/y]L]]$.

\item[Case:] the axiom steps: conclusion vs. axiom.
The proof 
\begin{center}
  \begin{math}
    $$\mprset{flushleft}
    \inferrule* [right=Cut] {
      $$\mprset{flushleft}
      \inferrule* [right=] {
        \inferrule* [right=,vdots=1.5em,fraction=\,] {
            \,
          }{\pi}          
      }{[[G |- t : A | L]]}
      \,
      $$\mprset{flushleft}
      \inferrule* [right=Ax] {
        \,
      }{[[x : A |- x : A]]}
    }{[[G |- L | h{[t/x]x} : A]]}
  \end{math}
\end{center}
transforms into 
\begin{center}
  \begin{math}
    $$\mprset{flushleft}
    \inferrule* [right=\text{\tiny Series of Exchanges}] {
      \inferrule* [right=] {
        \inferrule* [right=,vdots=1.5em,fraction=\,] {
            \,
          }{\pi}          
      }{[[G |- t : A | L]]}
    }{[[G |- L | t : A ]]}
  \end{math}
\end{center}
By the definition of the substitution function we know $[[t =
[t/x]x]]$.

\item[Case:] the exchange steps: conclusion vs. left-exchange (the
  first case).
The proof
\begin{center}
  \begin{math}
    $$\mprset{flushleft}
    \inferrule* [right=Cut] {
      \inferrule* [right=] {
        \inferrule* [right=,vdots=1.5em,fraction=\,] {
            \,
          }{\pi_1}          
      }{[[G |- t : A | L]]}
      \\
      $$\mprset{flushleft}
      \inferrule* [right=Exl] {        
        $$\mprset{flushleft}
        \inferrule* [right=] {
          \inferrule* [right=,vdots=1.5em,fraction=\,] {
            \,
          }{\pi_2}          
        }{[[G1,x : A, y : B, G2 |- L']]}        
      }{[[G1,y : B,x : A, G2 |- L']]}
    }{[[G1,y : B,G, G2 |- L | [t/x]L']]}
  \end{math}
\end{center}
transforms into the proof
\begin{center}
  \begin{math}
    $$\mprset{flushleft}
    \inferrule* [right=\text{Series of Exchanges}] {
      $$\mprset{flushleft}
      \inferrule* [right=Cut] {
        \inferrule* [right=] {
        \inferrule* [right=,vdots=1.5em,fraction=\,] {
            \,
          }{\pi_1}          
      }{[[G |- t : A | L]]}
      \\
        $$\mprset{flushleft}
        \inferrule* [right=] {
          \inferrule* [right=,vdots=1.5em,fraction=\,] {
            \,
          }{\pi_2}          
        }{[[G1,x : A, y : B, G2 |- L']]}        
      }{[[G1,G,y : B, G2 |- L | [t/x]L']]}
    }{[[G1,y : B,G, G2 |- L | [t/x]L']]}
  \end{math}
\end{center}
Clearly, all terms are equivalent.  

\item[Case:] the exchange steps: conclusion vs. left-exchange (the
  second case). The proof
\begin{center}
  \begin{math}
    $$\mprset{flushleft}
    \inferrule* [right=Cut] {
      \inferrule* [right=] {
        \inferrule* [right=,vdots=1.5em,fraction=\,] {
            \,
          }{\pi_1}          
      }{[[G |- t : B | L]]}
      \\
      $$\mprset{flushleft}
      \inferrule* [right=Exl] {        
        $$\mprset{flushleft}
        \inferrule* [right=] {
          \inferrule* [right=,vdots=1.5em,fraction=\,] {
            \,
          }{\pi_2}          
        }{[[G1,x : A, y : B, G2 |- L']]}        
      }{[[G1,y : B,x : A, G2 |- L']]}
    }{[[G1,G, x : A,G2 |- L | [t/y]L']]}
  \end{math}
\end{center}
transforms into the proof
\begin{center}
  \begin{math}
    $$\mprset{flushleft}
    \inferrule* [right=\text{Series of Exchanges}] {
      $$\mprset{flushleft}
      \inferrule* [right=Cut] {
        \inferrule* [right=] {
        \inferrule* [right=,vdots=1.5em,fraction=\,] {
            \,
          }{\pi_1}          
      }{[[G |- t : B | L]]}
      \\
        $$\mprset{flushleft}
        \inferrule* [right=] {
          \inferrule* [right=,vdots=1.5em,fraction=\,] {
            \,
          }{\pi_2}          
        }{[[G1,x : A, y : B, G2 |- L']]}        
      }{[[G1,x : A,G,G2 |- L | [t/y]L']]}
    }{[[G1,G,x : A,G2 |- L | [t/y]L']]}
  \end{math}
\end{center}
Clearly, all terms are equivalent.

\item[Case:] the exchange steps: conclusion vs. right-exchange
The proof
\begin{center}
  \begin{math}
    $$\mprset{flushleft}
    \inferrule* [right=Cut] {
        \inferrule* [right=] {
        \inferrule* [right=,vdots=1.5em,fraction=\,] {
            \,
          }{\pi_1}          
      }{[[G |- t : A | L]]}
      \\
      $$\mprset{flushleft}
      \inferrule* [right=Exr] {
        $$\mprset{flushleft}
        \inferrule* [right=] {
          \inferrule* [right=,vdots=1.5em,fraction=\,] {
            \,
          }{\pi_2}          
        }{[[G1,x : A, G2 |- h(h(L1 | t1 : B) | t2 : C) | L']]}        
      }{[[G1,x : A, G2 |- h(h(L1 | t2 : C) | t1 : B) | L']]}
    }{[[G1,G,G2 |- L | h(h(h(h([t/x]L1) | h{[t/x]t2} : C) | h{[t/x]t1} : B) | [t/x]L')]]}
  \end{math}
\end{center}
transforms into this proof
\begin{center}
  \begin{math}
    $$\mprset{flushleft}
    \inferrule* [right=Exr] {
      $$\mprset{flushleft}
      \inferrule* [right=Cut] {
        \inferrule* [right=] {
        \inferrule* [right=,vdots=1.5em,fraction=\,] {
            \,
          }{\pi_1}          
      }{[[G |- t : A | L]]}
      \\
      $$\mprset{flushleft}
        \inferrule* [right=] {
          \inferrule* [right=,vdots=1.5em,fraction=\,] {
            \,
          }{\pi_2}          
        }{[[G1,x : A, G2 |- h(h(L1 | t1 : B) | t2 : C) | L']]}        
      }{[[G1,G, G2 |- L | h(h(h(h([t/x]L1) | h{[t/x]t1} : B) | h{[t/x]t2} : C) | [t/x]L')]]}
    }{[[G1,G, G2 |- h(h(h([t/x]L1) | h{[t/x]t2} : C) | h{[t/x]t1} : B) | [t/x]L']]}
  \end{math}
\end{center}
Clearly, all terms are equivalent.  
  
\item[Case:] principal formula vs. principal formula: tensor.
The proof 
\begin{center} 
    \begin{math}
    $$\mprset{flushleft}
    \inferrule* [right=\footnotesize Cut] {
      $$\mprset{flushleft}
      \inferrule* [right=\footnotesize Tr] {
        \inferrule* [right=] {
        \inferrule* [right=,vdots=1.5em,fraction=\,] {
            \,
          }{\pi_1}          
      }{[[G1 |- t1 : A | L1]]}
      \\
      \inferrule* [right=] {
        \inferrule* [right=,vdots=1.5em,fraction=\,] {
            \,
          }{\pi_2}          
      }{[[G2 |- t2 : B | L2]]}
      }{[[G1,G2 |- t1 (x) t2 : A (x) B | h(L1 | L2)]]}
      \\
      $$\mprset{flushleft}
      \inferrule* [right=\footnotesize Tl] {
        \inferrule* [right=] {
          \inferrule* [right=,vdots=1.5em,fraction=\,] {
            \,
          }{\pi_3}          
        }{[[G3,x : A, y : B,G4 |- L3]]}
      }{[[G3,z : A (x) B,G4 |- h(let z be x (x) y in L3)]]}
    }{[[G3,G1,G2,G4 |- h(L1 | L2) | [t1 (x) t2/z](h(let z be x (x) y in L3))]]}
  \end{math}
\end{center}
is transformed into the proof
\begin{center}
  \begin{math}
    $$\mprset{flushleft}
    \inferrule* [right=Cut] {
      \inferrule* [right=] {
        \inferrule* [right=,vdots=1.5em,fraction=\,] {
            \,
          }{\pi_1}          
      }{[[G1 |- t1 : A | L1]]}
      \\
      $$\mprset{flushleft}
      \inferrule* [right=Cut] {
        \inferrule* [right=] {
        \inferrule* [right=,vdots=1.5em,fraction=\,] {
            \,
          }{\pi_2}          
      }{[[G2 |- t2 : B | L2]]}
      \\
      \inferrule* [right=] {
          \inferrule* [right=,vdots=1.5em,fraction=\,] {
            \,
          }{\pi_3}          
        }{[[G3,x : A, y : B,G4 |- L3]]}
      }{[[G3,x : A, G2,G4 |- L2 | [t2/y]L3]]}
    }{[[G3,G1, G2,G4 |- h(L1 | L2) | [t1/x][t2/y]L3]]}
  \end{math}
\end{center}
Without loss of generality suppose $[[L3 = t3
: C, L'3]]$.  We can see that $[[h{[t1 (x) t2/z]{let z be x (x) y in
    t3}} = let t1 (x) t2 be x (x) y in t3]]$ by the definition of
substitution, and by using the $\FILLdrulename{Eq\_Beta1Tensor}$ rule we obtain
$[[let t1 (x) t2 be x (x) y in t3 = [t1/x][t2/y]t3]]$.  This argument
can be repeated for any term in $[[ [t1 (x) t2/z](h(let z be x (x) y
in L'3))]]$, and thus, $[[ [t1 (x) t2/z](h(let z be x (x) y in L3)) =
[t1/x][t2/y]L3]]$.

Note that in the second derivation of the above transformation we
first cut on $[[B]]$, and then $[[A]]$, but we could have cut on
$[[A]]$ first, and then $[[B]]$, but this would yeild equivalent
derivations as above by using
Lemma~\ref{lemma:substitution_distribution}.

  
\item[Case:] principal formula vs. principal formula: par.
The proof
\begin{center}
  \scriptsize
    \begin{math}    
    $$\mprset{flushleft}
\inferrule* [right=\tiny Cut] {
  $$\mprset{flushleft}
  \inferrule* [leftskip=15px,left=\tiny Parr,rightskip=10px] {
    \inferrule* [] {
        \inferrule* [right=,vdots=1.5em,fraction=\,] {
            \,
          }{\pi_1}          
      }{[[G1 |- h(L1 | t1 : A) | h(t2 : B | L2)]]}
    }{[[G1 |- h(L1 | t1 (+) t2 : A (+) B) | L2]]}
  \\
  $$\mprset{flushleft}
  \inferrule* [leftskip=5px,right=\tiny Parl,rightskip=30px] {
    \inferrule* [right=] {
        \inferrule* [right=,vdots=1.5em,fraction=\,] {
            \,
          }{\pi_2}          
      }{[[G2,x : A |- L3]]}
      \\
      \inferrule* [right=] {
        \inferrule* [right=,vdots=1.5em,fraction=\,] {
            \,
          }{\pi_3}          
      }{[[G3, y : B |- L4]]}
  }{[[G2,G3,z : A (+) B |- h(let-pat z (x (+) -) L3) | h(let-pat z (- (+) y) L4)]]}
}{[[G2,G3,G1 |- h(h(L1 | L2) | h([t1 (+) t2/z](let-pat z (x (+) -) L3))) | h([t1 (+) t2/z](let-pat z (- (+) y) L4))]]}
  \end{math}    
\end{center}
is transformed into the proof
\begin{center}
  \scriptsize
  \begin{math}    
    $$\mprset{flushleft}
    \inferrule* [right=  \scriptsize \tiny Series of Exchanges] {
      $$\mprset{flushleft}
    \inferrule* [right=  \scriptsize Cut] {
      $$\mprset{flushleft}
      \inferrule* [right=  \scriptsize Cut] {
        \inferrule* [right=] {
          \inferrule* [right=,vdots=1.5em,fraction=\,] {
            \,
          }{\pi_1}          
        }{[[G1 |- h(L1 | t1 : A) | h(t2 : B | L2)]]}
        \\
        \inferrule* [right=] {
          \inferrule* [right=,vdots=1.5em,fraction=\,] {
            \,
          }{\pi_3}          
        }{[[G3, y : B |- L4]]}
      }{[[G3, G1 |- h(h(L1 | t1 : A) | L2) | [t2/y]L4]]}
      \\
      \inferrule* [right=] {
        \inferrule* [right=,vdots=1.5em,fraction=\,] {
          \,
        }{\pi_2}          
      }{[[G2,x : A |- L3]]}
    }{[[G2,G3,G1 |- h(h(L1 | L2) | [t2/y]L4) | [t1/x]L3]]}
    }{[[G2,G3,G1 |- h(L1 | L2) | h(h([t1/x]L3) | [t2/y]L4)]]}
  \end{math}
\end{center}
Without loss of generality consider the case when
$[[L3]] = [[t3 : C_1 | L'3]]$ and $[[L4]] = [[t4 : C_2 | L'4]]$.  First,
$[[ [t1 (+) t2/z]{let-pat z (x (+) -) t3} = let-pat {t1 (+) t2} (x (+)
-) t3]]$,
and by $\FILLdrulename{Eq\_Beta1Par}$ we know
$[[let-pat {t1 (+) t2} (x (+) -) t3 = [t1/x]t3]]$ if $[[x in FV(t3)]]$
or $[[let-pat {t1 (+) t2} (x (+) -) t3 = t3]]$ otherwise.  In the
latter case we can see that $[[t3 = [t1/x]t3]]$, thus, in both cases
$[[let-pat {t1 (+) t2} (x (+) -) t3 = [t1/x]t3]]$.  This argument can
be repeated for any terms in $[[L'3]]$, and hence
$[[ [t1 (+) t2/z](let-pat z (x (+) -) L3)]] = [[let-pat {t1 (+) t2} (x
(+) -) L3 = [t1/x]L3]]$.
We can apply a similar argument for\\
$[[ [t1 (+) t2/z]{let-pat z (- (+) y) t4}]]$ and
$[[ [t1 (+) t2/z](let-pat z (- (+) y) L4)]]$.


  Note that we could have first cut on $[[A]]$, and then on $[[B]]$ in
  the second derivation, but we would have arrived at the same result
  just with potentially more exchanges on the right.
\end{report}
\begin{report}
  Note that just as we mentioned about tensor we could have first cut on
$[[A]]$, and then on $[[B]]$ in the second derivation, but we would
have arrived at the same result just with potentially more exchanges
on the right.


  \item[Case:] principal formula vs. principal formula: implication.
The proof
\begin{center}
  \scriptsize
  \begin{math}
    $$\mprset{flushleft}
    \inferrule* [right=\scriptsize Cut] {
      $$\mprset{flushleft}
      \inferrule* [right=\scriptsize Impr] {
        \inferrule* [right=] {
          \inferrule* [right=,vdots=1.5em,fraction=\,] {
            \,
          }{\pi_1}          
        }{[[G, x : A |- t : B | L]]}
        \\
        [[x nin FV(L)]]
      }{[[G |- \x.t : A -o B | L]]}
      \\
      $$\mprset{flushleft}
      \inferrule* [right=\scriptsize Impl] {
        \inferrule* [right=] {
          \inferrule* [right=,vdots=1.5em,fraction=\,] {
            \,
          }{\pi_2}          
        }{[[G1 |- t1 : A | L1]]}
        \\
        \inferrule* [right=] {
          \inferrule* [right=,vdots=1.5em,fraction=\,] {
            \,
          }{\pi_3}          
        }{[[G2, y : B |- L2]]}
      }{[[G1, z : A -o B,G2 |- L1 | [z t1/y]L2]]}
    }{[[G1,G,G2 |- L | h(h([\x.t/z]L1) | h([\x.t/z][z t1/y]L2))]]}
  \end{math}
\end{center}
transforms into the proof
\begin{center}
  \scriptsize
  \begin{math}
    $$\mprset{flushleft}
    \inferrule* [right=\tiny Series of Exchanges] {
      $$\mprset{flushleft}
    \inferrule* [right=\scriptsize Cut] {
      $$\mprset{flushleft}
      \inferrule* [right=\scriptsize Cut] {
        \inferrule* [right=,rightskip=17px] {
          \inferrule* [right=,vdots=1.5em,fraction=\,] {
            \,
          }{\pi_2}          
        }{[[G1 |- t1 : A | L1]]}
        \\
        \inferrule* [right=,rightskip=13px] {
          \inferrule* [right=,vdots=1.5em,fraction=\,] {
            \,
          }{\pi_1}          
        }{[[G, x : A |- t : B | L]]}
        \\
        [[x nin FV(L)]]
      }{[[G, G1 |- L1 | h(h{[t1/x]t} : B | [t1/x]L)]]}
      \\
      \inferrule* [right=] {
          \inferrule* [right=,vdots=1.5em,fraction=\,] {
            \,
          }{\pi_3}          
        }{[[G2, y : B |- L2]]}
      }{[[G2, G,G1 |- h(L1 | [t1/x]L) | [ [t1/x]t/y]L2]]}
    }{[[G1, G,G2 |- h([t1/x]L) | h(L1 | h([ [t1/x]t/y]L2))]]}
  \end{math}
\end{center}
Without loss of generality consider the case when $[[L2]] = [[t2 : C | L'2]]$. 
First, by hypothesis we know $[[x nin FV(L)]]$, and so we know $[[L = [t1/x]L]]$.  We can see
that $[[ [\x.t/z][z t1/y]t2 = [{\x.t} t1/y]t2]] = [[ [ [t1/x]t/y]t2]]$ by using the congruence rules
of equality and the rule $\FILLdrulename{Eq\_BetaFun}$.  This argument can be repeated for any term
in $[[ [\x.t/z][z t1/y]L'2]]$, and so $[[ [\x.t/z][z t1/y]L2 = [ [t1/x]t/y]L2]]$.  Finally, by inspecting the previous
derivations we can see that $[[z nin FV(L1)]]$, and thus, $[[L1 =
[\x.t/z]L1]]$.  

\item[Case:] principal formula vs. principal formula: tensor unit.
    The proof
\begin{center}
  \begin{math}
    $$\mprset{flushleft}
    \inferrule* [right=Cut] {
      $$\mprset{flushleft}
      \inferrule* [right=Ir] {
        \,
      }{[[. |- * : I]]}
      \\
      $$\mprset{flushleft}
      \inferrule* [right=Il] {
        \inferrule* [right=] {
          \inferrule* [right=,vdots=1.5em,fraction=\,] {
            \,
          }{\pi}          
        }{[[G |- L]]}
      }{[[G, x : I |- let x be stp in L]]}
    }{[[G |- [*/x](let x be stp in L)]]}
  \end{math}
\end{center}
is transformed into the proof
\begin{center}
  \begin{math}
    \inferrule* [right=] {
      \inferrule* [right=,vdots=1.5em,fraction=\,] {
        \,
      }{\pi}          
    }{[[G |- L]]}
  \end{math}
\end{center}

Without loss of generality suppose $[[L]] = [[t : A | L']]$. We can see that 
$[[ [*/x]{let x be stp in t} = let * be stp in t]] = [[t]]$ by the definition of 
substitution and the $\FILLdrulename{Eq\_EtaI}$ rule.  This argument can be repeated for any
term in $[[ [*/x](let x be stp in L')]]$, and hence, $[[ [*/x](let x be
stp in L) = L]]$.

\item[Case:] principal formula vs. principal formula: par unit.
  The proof
\begin{center}
  \begin{math}
    $$\mprset{flushleft}
    \inferrule* [right=Cut] {
      $$\mprset{flushleft}
      \inferrule* [right=Pr] {
        \inferrule* [right=] {
          \inferrule* [right=,vdots=1.5em,fraction=\,] {
            \,
          }{\pi}          
        }{[[G |- L]]}
      }{[[G |- o : _|_ | L]]}
      \\
      $$\mprset{flushleft}
      \inferrule* [right=Pl] {
        \,
      }{[[x : _|_ |- .]]}
    }{[[G |- L | [o/x].]]}
  \end{math}
\end{center}
transforms into the proof
\begin{center}
  \begin{math}
    \inferrule* [right=] {
      \inferrule* [right=,vdots=1.5em,fraction=\,] {
        \,
      }{\pi}          
    }{[[G |- L]]}
  \end{math}
\end{center}
Clearly, $[[ [o/x]. = .]]$.

\item[Case:] secondary conclusion: left introduction of implication.
The proof 
\begin{center}
  \begin{math}
    $$\mprset{flushleft}
    \inferrule* [right=Cut] {
      $$\mprset{flushleft}
      \inferrule* [right=Impl] {
        \inferrule* [right=] {
          \inferrule* [right=,vdots=1.5em,fraction=\,] {
            \,
          }{\pi_1}          
        }{[[G |- t1 : A | L]]}
        \\
        \inferrule* [right=] {
          \inferrule* [right=,vdots=1.5em,fraction=\,] {
            \,
          }{\pi_2}          
        }{[[G1, x : B,G2 |- t2 : C | L2]]}
      }{[[G,y : A -o B,G1,G2 |- h(L | h([y t1/x]t2 : C)) | [y t1/x]L2]]}
      \\      
      \inferrule* [right=] {
          \inferrule* [right=,vdots=1.5em,fraction=\,] {
            \,
          }{\pi_3}          
        }{[[G3,z : C, G4 |- L3]]}
      }{[[G3,G,y : A -o B,G1,G2,G4 |- L | h(h([y t1/x]L2) | [ [y t1/x]t2/z]L3)]]}
  \end{math}
\end{center}
transforms into the proof
\begin{center}
  \begin{math}
    $$\mprset{flushleft}
    \inferrule* [right=\tiny Series of Exchanges] {
      $$\mprset{flushleft}
    \inferrule* [right=Impl] {
      \inferrule* [right=] {
          \inferrule* [right=,vdots=1.5em,fraction=\,] {
            \,
          }{\pi_1}          
        }{[[G |- t1 : A | L]]}
      \\
      $$\mprset{flushleft}
      \inferrule* [right=Cut] {
        \inferrule* [right=] {
          \inferrule* [right=,vdots=1.5em,fraction=\,] {
            \,
          }{\pi_2}          
        }{[[G1, x : B,G2 |- t2 : C | L2]]}
        \\
        \inferrule* [right=] {
          \inferrule* [right=,vdots=1.5em,fraction=\,] {
            \,
          }{\pi_3}          
        }{[[G3,z : C, G4 |- L3]]}
      }{[[G3,G1, x : B,G2,G4 |-L2 | [t2/z]L3]]}
    }{[[G, y : A -o B,G3,G1,G2,G4 |- h(L | h([y t1/x]L2)) | [y t1/x][t2/z]L3]]}
  }{[[G3, G,y : A -o B,G1,G2,G4 |- h(L | h([y t1/x]L2)) | [y t1/x][t2/z]L3]]}
  \end{math}
\end{center}
This case is similar to
Section~\ref{subsec:commuting_conversion_cut_vs_cut_(first_case)}.
Thus, we can prove that $[[ [y t1/x][t2/z]L3 = [ [y t1/x]t2/z]L3]]$ by
Lemma~\ref{lemma:substitution_distribution} and the fact that $[[x nin
FV(L3)]]$.

\item[Case:] secondary conclusion: left introduction of exchange.
  The proof
\begin{center}
  \begin{math}
    $$\mprset{flushleft}
    \inferrule* [right=Cut] {
      $$\mprset{flushleft}
      \inferrule* [right=Exl] {
        \inferrule* [right=] {
          \inferrule* [right=,vdots=1.5em,fraction=\,] {
            \,
          }{\pi_1}          
        }{[[G,y : B, x : A, G' |- t : C | L]]}        
      }{[[G, x : A, y : B, G' |- t : C | L]]}
      \\
      \inferrule* [right=] {
        \inferrule* [right=,vdots=1.5em,fraction=\,] {
          \,
        }{\pi_2}          
      }{[[G1,z : C, G2 |- L2]]}
    }{[[G1,G, x : A, y : B, G', G2 |- L | [t/z]L2]]}
  \end{math}
\end{center}
transforms into the proof
\begin{center}
  \begin{math}
    $$\mprset{flushleft}
    \inferrule* [right=Exl] {
      $$\mprset{flushleft}
      \inferrule* [right=Cut] {
        \inferrule* [right=] {
          \inferrule* [right=,vdots=1.5em,fraction=\,] {
            \,
          }{\pi_1}          
        }{[[G,y : B, x : A, G' |- t : C | L]]}        
        \\
        \inferrule* [right=] {
        \inferrule* [right=,vdots=1.5em,fraction=\,] {
          \,
        }{\pi_2}          
      }{[[G1,z : C, G2 |- L2]]}
    }{[[G1,G,y : B, x : A, G', G2 |- L | [t/z]L2]]}
  }{[[G1,G, x : A, y : B, G', G2 |- L | [t/z]L2]]}
  \end{math}
\end{center}
Clearly, all terms are equivalent.

\item[Case:] secondary conclusion: left introduction of tensor.
The proof 
\begin{center}
  \begin{math}
    $$\mprset{flushleft}
    \inferrule* [right=Cut] {
      $$\mprset{flushleft}
      \inferrule* [right=Tl] {
        \inferrule* [right=] {
        \inferrule* [right=,vdots=1.5em,fraction=\,] {
          \,
        }{\pi_1}          
      }{[[G, x : A, y : B |- t : C | L]]}      
      }{[[G, z : A (x) B |- h{let z be x (x) y in t} : C | let z be x (x) y in L]]}
      \\
      \inferrule* [right=] {
        \inferrule* [right=,vdots=1.5em,fraction=\,] {
          \,
        }{\pi_2}          
      }{[[G1,w : C, G2 |- L2]]}
    }{[[G1,G, z : A (x) B, G2 |- h(let z be x (x) y in L) | [h{let z be x (x) y in t}/w]L2]]}
  \end{math}
\end{center}
transforms into the proof
\begin{center}
  \begin{math}
    $$\mprset{flushleft}
    \inferrule* [right=Tl] {
      $$\mprset{flushleft}
      \inferrule* [right=Cut] {
        \inferrule* [right=] {
        \inferrule* [right=,vdots=1.5em,fraction=\,] {
          \,
        }{\pi_1}          
      }{[[G, x : A, y : B |- t : C | L]]}      
      \\
      \inferrule* [right=] {
        \inferrule* [right=,vdots=1.5em,fraction=\,] {
          \,
        }{\pi_2}          
      }{[[G1,w : C, G2 |- L2]]}
    }{[[G1,G, x : A, y : B, G2 |- L | [t/w]L2]]}      
  }{[[G1,G, z : A (x) B, G2 |- h(let z be x (x) y in L) | let z be x (x) y in ([t/w]L2)]]}
  \end{math}
\end{center}
It suffices to show that $[[let z be x (x) y in ([t/w]L2) = [h{let z
  be x (x) y in t}/w]L2]]$.  This is a simple consequence of the rule
\FILLdrulename{Eq\_NatTensor}.

\item[Case:] secondary conclusion: left introduction of Par
The proof 
\begin{center}
  \scriptsize
  \begin{math}
    $$\mprset{flushleft}
    \inferrule* [right=\scriptsize Cut] {
      $$\mprset{flushleft}
      \inferrule* [right=\scriptsize Parl] {
        \inferrule* [right=] {
          \inferrule* [right=,vdots=1.5em,fraction=\,] {
            \,
          }{\pi_1}          
        }{[[G, x : A |- L]]}      
        \\
        \inferrule* [right=] {
          \inferrule* [right=,vdots=1.5em,fraction=\,] {
            \,
          }{\pi_2}          
        }{[[G', y : B |- t' : C | L']]}      
      }{[[G,G',z : A (+) B |- h(h(let-pat z (x (+) -) L) | h{let-pat z (- (+) y) t'} : C) | h(let-pat z (- (+) y) L')]]}
      \\
      \inferrule* [right=] {
        \inferrule* [right=,vdots=1.5em,fraction=\,] {
          \,
        }{\pi_3}          
      }{[[G1,w : C, G2 |- L2]]}
    }{[[G1,G,G',z : A (+) B, G2 |- h(h(let-pat z (x (+) -) L) | h(let-pat z (- (+) y) L')) | [h{let-pat z (- (+) y) t'}/w]L2 ]]}
  \end{math}
\end{center}
is transformed into the proof
\begin{center}
  \begin{math}
    $$\mprset{flushleft}
    \inferrule* [right=\tiny Series of Exchanges] {
      $$\mprset{flushleft}
    \inferrule* [right=Parl] {
      \inferrule* [right=] {
        \inferrule* [right=,vdots=1.5em,fraction=\,] {
          \,
        }{\pi_1}          
      }{[[G, x : A |- L]]}      
      \\
      $$\mprset{flushleft}
      \inferrule* [right=Cut] {
        \inferrule* [right=] {
          \inferrule* [right=,vdots=1.5em,fraction=\,] {
            \,
          }{\pi_2}          
        }{[[G', y : B |- t' : C | L']]}      
        \\
        \inferrule* [right=] {
          \inferrule* [right=,vdots=1.5em,fraction=\,] {
            \,
          }{\pi_3}          
        }{[[G1,w : C, G2 |- L2]]}
      }{[[G1,G', y : B, G2 |- L' | [t'/w]L2]]}
    }{[[G,G1,G', G2, z : A (+) B |- h(h(let-pat z (x (+) -) L) | h(let-pat z (- (+) y) L')) | h(let-pat z (- (+) y) [t'/w]L2)]]}
  }{[[G1,G,G',z : A (+) B,G2 |- h(h(let-pat z (x (+) -) L) | h(let-pat z (- (+) y) L')) | h(let-pat z (- (+) y) [t'/w]L2)]]}
  \end{math}
\end{center}
It suffices to show that $[[let-pat z (- (+) y) [t'/w]L2]] = [[
[h{let-pat z (- (+) y) t'}/w]L2]]$.  This follows from the rule
$\FILLdrulename{Eq\_Nat2Par}$.

\item[Case:] secondary conclusion: left introduction of tensor unit.
The proof
\begin{center}
  \begin{math}
    $$\mprset{flushleft}
    \inferrule* [right=Cut] {
      $$\mprset{flushleft}
      \inferrule* [right=Il] {
        \inferrule* [right=] {
        \inferrule* [right=,vdots=1.5em,fraction=\,] {
          \,
        }{\pi_1}          
      }{[[G |- t : C | L]]}      
      }{[[G, x : I |- t : C | L]]}
      \\
      \inferrule* [right=] {
          \inferrule* [right=,vdots=1.5em,fraction=\,] {
            \,
          }{\pi_2}          
        }{[[G1,w : C, G2 |- L1]]}
      }{[[G1,G,x : I,G2 |- L | [t/w]L1]]}
  \end{math}
\end{center}
is transformed into the following:
\begin{center}
  \begin{math}
    $$\mprset{flushleft}
    \inferrule* [right=\tiny Series of Exchanges] {
      $$\mprset{flushleft}
    \inferrule* [right=Il] {
      $$\mprset{flushleft}
      \inferrule* [right=Cut] {
        \inferrule* [right=] {
          \inferrule* [right=,vdots=1.5em,fraction=\,] {
            \,
          }{\pi_1}          
        }{[[G |- t : C | L]]}      
        \\
        \inferrule* [right=] {
          \inferrule* [right=,vdots=1.5em,fraction=\,] {
            \,
          }{\pi_2}          
        }{[[G1,w : C, G2 |- L1]]}  
      }{[[G1,G, G2 |- L | [t/w]L1]]}
    }{[[G1,G, G2, x : I |- L | [t/w]L1]]}
  }{[[G1,G, x : I,G2 |- L | [t/w]L1]]}
  \end{math}
\end{center}
Clearly, all terms are equivalent.  Note that we do not give a case
for secondary conclusion of the left introduction of par's unit,
because it can only be introduced given an empty right context, and
thus there is no cut formula.

\item[Case:] secondary hypothesis: left introduction of tensor.
The proof
\begin{center}
  \begin{math}
    $$\mprset{flushleft}
    \inferrule* [right=Cut] {
      \inferrule* [right=] {
        \inferrule* [right=,vdots=1.5em,fraction=\,] {
          \,
        }{\pi_1}          
      }{[[G |- t : A | L]]}      
      \\
      $$\mprset{flushleft}
      \inferrule* [right=Tl] {
        \inferrule* [right=] {
        \inferrule* [right=,vdots=1.5em,fraction=\,] {
          \,
        }{\pi_2}          
      }{[[G1,x : A,G2,y : B,z : C,G3 |- t1 : D | L1]]}                  
    }{[[G1,x : A,G2,w : B (x) C,G3 |- h{let w be y (x) z in t1} : D | let w be y (x) z in L1]]}
  }{[[G1,G,G2,w : B (x) C,G3 |- L | h(h{[t/x]{let w be y (x) z in t1}} : D | [t/x](let w be y (x) z in L1))]]}
  \end{math}
\end{center}
transforms into the proof
\begin{center}
  \begin{math}
    $$\mprset{flushleft}
    \inferrule* [right=Tl] {
      $$\mprset{flushleft}
      \inferrule* [right=Cut] {
        \inferrule* [right=] {
        \inferrule* [right=,vdots=1.5em,fraction=\,] {
          \,
        }{\pi_1}          
      }{[[G |- t : A | L]]}      
      \\
      \inferrule* [right=] {
        \inferrule* [right=,vdots=1.5em,fraction=\,] {
          \,
        }{\pi_2}          
      }{[[G1,x : A,G2,y : B,z : C,G3 |- t1 : D | L1]]}                  
    }{[[G1,G,G2,y : B,z : C,G3 |- h(L | h{[t/x]t1} : D) | [t/x]L1]]}
  }{[[G1,G,G2,w : B (x) C,G3 |- h(h(let w be x (x) y in L) | h{let w be x (x) y in [t/x]t1} : D) | let w be x (x) y in [t/x]L1]]}
  \end{math}
\end{center}
First, we can see by inspection of the previous derivations that
$[[x,y nin FV(L)]]$, thus, by using similar reasoning as above we can
use the $\FILLdrulename{EtaTensor}$ rule to obtain $[[let w be x (x) y in
L = L]]$.  It is a well-known property of substitution that 
$[[ [t/x]{let w be x (x) y in t1}]] = [[let [t/x]w be x (x) y in
[t/x]t1]] = [[let w be x (x) y in [t/x]t1]]$.

\item[Case:] secondary hypothesis: right introduction of tensor (first
  case).
The proof
\begin{center}
  \begin{math}
    $$\mprset{flushleft}
    \inferrule* [right=Cut] {
      \inferrule* [right=] {
        \inferrule* [right=,vdots=1.5em,fraction=\,] {
          \,
        }{\pi_1}          
      }{[[G |- t : A | L]]}      
      \\
      $$\mprset{flushleft}
      \inferrule* [right=Tr] {
        \inferrule* [right=] {
        \inferrule* [right=,vdots=1.5em,fraction=\,] {
          \,
        }{\pi_2}          
      }{[[G1,x : A, G2 |- t1 : B | L1]]}      
      \\
      \inferrule* [right=] {
        \inferrule* [right=,vdots=1.5em,fraction=\,] {
          \,
        }{\pi_3}          
      }{[[G3 |- t2 : C | L2]]}      
      }{[[G1,x :A,G2,G3 |- t1 (x) t2 : B (x) C | h(L1 | L2)]]}
    }{[[G1,G,G2,G3 |- L | h(h{[t/x]{t1 (x) t2}} : B (x) C | h(h([t/x]L1) | [t/x]L2))]]}
  \end{math}
\end{center}
transforms into the proof
\begin{center}
  \begin{math}
    $$\mprset{flushleft}
    \inferrule* [right=\tiny Series of Exchanges] {
      $$\mprset{flushleft}
    \inferrule* [right=Tr] {
      $$\mprset{flushleft}
      \inferrule* [right=Cut] {
        \inferrule* [right=] {
        \inferrule* [right=,vdots=1.5em,fraction=\,] {
          \,
        }{\pi_1}          
      }{[[G |- t : A | L]]}      
      \\
      \inferrule* [right=] {
        \inferrule* [right=,vdots=1.5em,fraction=\,] {
          \,
        }{\pi_2}          
      }{[[G1,x : A, G2 |- t1 : B | L1]]}      
    }{[[G1,G, G2 |- L | h(h{[t/x]t1} : B | [t/x]L1)]]}
    \\
    \inferrule* [right=] {
        \inferrule* [right=,vdots=1.5em,fraction=\,] {
          \,
        }{\pi_3}          
      }{[[G3 |- t2 : C | L2]]}      
    }{[[G1,G, G2,G3 |- h(h{[t/x]t1} (x) t2 : B (x) C) | h(L | h(h([t/x]L1) | L2))]]}
  }{[[G1,G, G2,G3 |- L | h(h({[t/x]t1} (x) t2 : B (x) C) | h(h([t/x]L1) | L2))]]}
  \end{math}
\end{center}
By inspection of the previous derivations we can see that $[[x nin
FV(t2)]]$ and $[[x nin FV(L2)]]$.  Thus, $[[ [t/x]L2 = L2]]$ and
$[[ [t/x]{t1 (x) t2} = {[t/x]t1} (x) {[t/x]t2}]] = [[{[t/x]t1} (x)
t2]]$.

\item[Case:] secondary hypothesis: right introduction of tensor
  (second case).
The proof
\begin{center}
  \begin{math}
    $$\mprset{flushleft}
    \inferrule* [right=Cut] {
      \inferrule* [right=] {
        \inferrule* [right=,vdots=1.5em,fraction=\,] {
          \,
        }{\pi_1}          
      }{[[G |- t : A | L]]}      
      \\
      $$\mprset{flushleft}
      \inferrule* [right=Tr] {
        \inferrule* [right=] {
        \inferrule* [right=,vdots=1.5em,fraction=\,] {
          \,
        }{\pi_2}          
      }{[[G1 |- t1 : B | L1]]}      
      \\
      \inferrule* [right=] {
        \inferrule* [right=,vdots=1.5em,fraction=\,] {
          \,
        }{\pi_3}          
      }{[[G2,x : A, G3  |- t2 : C | L2]]}      
      }{[[G1,G2,x :A,G3 |- t1 (x) t2 : B (x) C | h(L1 | L2)]]}
    }{[[G1,G,G2,G3 |- L | h(h{[t/x]{t1 (x) t2}} : B (x) C | h(h([t/x]L1) | [t/x]L2))]]}
  \end{math}
\end{center}
transforms into the proof
\begin{center}
  \begin{math}
    $$\mprset{flushleft}
    \inferrule* [right=\tiny Series of Exchanges] {
      $$\mprset{flushleft}
      \inferrule* [right=Tr] {
        \inferrule* [right=] {
        \inferrule* [right=,vdots=1.5em,fraction=\,] {
          \,
        }{\pi_2}          
      }{[[G1 |- t1 : B | L1]]}      
      \\
      $$\mprset{flushleft}
      \inferrule* [right=Cut] {
        \inferrule* [right=] {
        \inferrule* [right=,vdots=1.5em,fraction=\,] {
          \,
        }{\pi_1}          
      }{[[G |- t : A | L]]}      
      \\
      \inferrule* [right=] {
        \inferrule* [right=,vdots=1.5em,fraction=\,] {
          \,
        }{\pi_3}          
      }{[[G2,x : A, G3 |- t2 : C | L2]]}      
    }{[[G2,G, G3 |- L | h(h{[t/x]t2} : C | [t/x]L2)]]}       
    }{[[G1,G2,G,G3 |- h(t1 (x) {[t/x]t2} : B (x) C) | h(L1 | h(L | [t/x]L2))]]}
  }{[[G1,G, G2,G3 |- L | h(h(t1 (x) {[t/x]t2} : B (x) C) | h(L1 | h([t/x]L2)))]]}
  \end{math}
\end{center}
This case is similar to the previous case.  

\item[Case:] secondary hypothesis: right introduction of par.
The proof
\begin{center}
  \begin{math}
    $$\mprset{flushleft}
    \inferrule* [right=Cut] {
      \inferrule* [right=] {
        \inferrule* [right=,vdots=1.5em,fraction=\,] {
          \,
        }{\pi_1}          
      }{[[G |- t : A | L]]}      
      \\
      $$\mprset{flushleft}
      \inferrule* [right=Parr] {
        \inferrule* [right=] {
        \inferrule* [right=,vdots=1.5em,fraction=\,] {
          \,
        }{\pi_2}          
      }{[[G1,x : A,G2 |- h(L1 | h(t1 : B | t2 : C)) | L2]]}                  
    }{[[G1,x : A,G2 |- h(L1 | h(t1 (+) t2 : B (+) C)) | L2]]}
  }{[[G1,G,G2 |- L | h(h(h([t/x]L1) | h([t/x]{t1 (+) t2} : B (+) C)) | [t/x]L2)]]}
  \end{math}
\end{center}
transforms into the proof
\begin{center}
  \begin{math}
    $$\mprset{flushleft}
\inferrule* [right=Parr] {
  $$\mprset{flushleft}
  \inferrule* [right=Cut] {
    \inferrule* [right=] {
        \inferrule* [right=,vdots=1.5em,fraction=\,] {
          \,
        }{\pi_1}          
      }{[[G |- t : A | L]]}      
      \\
      \inferrule* [right=] {
        \inferrule* [right=,vdots=1.5em,fraction=\,] {
          \,
        }{\pi_2}          
      }{[[G1,x : A,G2 |- h(L1 | h(t1 : B | t2 : C)) | L2]]}                  
    }{[[G1,G,G2 |- L | h(h(h([t/x]L1) | h(h{[t/x]t1} : B | h{[t/x]t2} : C)) | [t/x]L2)]]}
  }{[[G1,G,G2 |- L | h(h(h([t/x]L1) | h{[t/x]t1} (+) h{[t/x]t2} : B (+) C) | [t/x]L2)]]}
  \end{math}
\end{center}
Clearly, $[[ [t/x]{t1 (+) t2} = {[t/x]t1} (+) [t/x]t2]]$.
\end{report}

\item[Case:] secondary hypothesis: left introduction of par (first
  case).
The proof
\begin{center}
  \begin{math}
    $$\mprset{flushleft}
    \inferrule* [leftskip=10px,right=\tiny Cut] {
      \inferrule* [right=] {
        \inferrule* [right=,vdots=1.5em,fraction=\,] {
          \,
        }{\pi_1}          
      }{[[G |- t : A | L]]}      
      \\
      $$\mprset{flushleft}
      \inferrule* [right=\tiny Parl] {
        \inferrule* [right=] {
        \inferrule* [right=,vdots=1.5em,fraction=\,] {
          \,
        }{\pi_2}          
      }{[[G1,x : A,G2,y : B |- L1]]}      
      \\
      \inferrule* [right=] {
        \inferrule* [right=,vdots=1.5em,fraction=\,] {
          \,
        }{\pi_3}          
      }{[[G3,z : C |- L2]]}      
      }{[[G1,x : A,G2,G3,w : B (+) C |- h(let-pat w (y (+) -) L1) | h(let-pat w (- (+) z) L2)]]}
    }{[[G1,G,G2,G3,w : B (+) C |- L | h(h([t/x](let-pat w (y (+) -) L1)) | h([t/x](let-pat w (- (+) z) L2)))]]}
  \end{math}
\end{center}
\noindent
transforms into the proof
\begin{center}
  \begin{math}
    $$\mprset{flushleft}
    \inferrule* [right=\tiny ParL] {
      $$\mprset{flushleft}
      \inferrule* [right=\tiny Cut] {
        \inferrule* [right=] {
        \inferrule* [right=,vdots=1.5em,fraction=\,] {
          \,
        }{\pi_1}          
      }{[[G |- t : A | L]]}      
      \\
      \inferrule* [right=] {
        \inferrule* [right=,vdots=1.5em,fraction=\,] {
          \,
        }{\pi_2}          
      }{[[G1,x : A,G2,y : B |- L1]]}      
      }{[[G1,G,G2,y : B |- L | [t/x]L1]]}
      \\
      \inferrule* [right=] {
        \inferrule* [right=,vdots=1.5em,fraction=\,] {
          \,
        }{\pi_3}          
      }{[[G3,z : C |- L2]]}      
    }{[[G1,G,G2,G3,w : B (+) C |- h(h(let-pat w (y (+) -) L) | h(let-pat w (y (+) -) [t/x]L1)) | let-pat w (- (+) z) L2]]}
  \end{math}
\end{center}
First, by inspection of the previous proofs we can see that $[[y nin
    FV(L)]]$ and $[[x nin FV(L2)]]$.  Thus, $[[let-pat w (y (+) -) L =
    L]]$, and $[[ [t/x](let-pat w (- (+) z) L2) = let-pat w (- (+) z)
    L2]]$. It suffices to show that $[[ [t/x](let-pat w (y (+) -) L1)
    = let-pat w (y (+) -) [t/x]L1]]$ but this follows by distributing
the substitution into the let-pat, and then simplifying using the fact
that $w \neq x$.

\begin{report}
\item[Case:] secondary hypothesis: left introduction of par (second
  case).
The proof
\begin{center}
  \begin{math}
    $$\mprset{flushleft}
    \inferrule* [right=Cut] {
      \inferrule* [right=] {
        \inferrule* [right=,vdots=1.5em,fraction=\,] {
          \,
        }{\pi_1}          
      }{[[G |- t : A | L]]}      
      \\
      $$\mprset{flushleft}
      \inferrule* [right=Parl] {
        \inferrule* [right=] {
        \inferrule* [right=,vdots=1.5em,fraction=\,] {
          \,
        }{\pi_2}          
      }{[[G1,y : B |- L1]]}      
      \\
      \inferrule* [right=] {
        \inferrule* [right=,vdots=1.5em,fraction=\,] {
          \,
        }{\pi_3}          
      }{[[G2,x : A,G3,z : C |- L2]]}      
      }{[[G1,G2,x : A,G3,w : B (+) C |- h(let-pat w (y (+) -) L1) | h(let-pat w (- (+) z) L2)]]}
    }{[[G1,G2,G,G3,w : B (+) C |- L | h(h([t/x](let-pat w (y (+) -) L1)) | h([t/x](let-pat w (- (+) z) L2)))]]}
  \end{math}
\end{center}
transforms into the proof
\begin{center}
  \begin{math}
    $$\mprset{flushleft}
    \inferrule* [right=Parl] {
      \inferrule* [right=] {
        \inferrule* [right=,vdots=1.5em,fraction=\,] {
          \,
        }{\pi_2}          
      }{[[G1,y : B |- L1]]}      
      \\
      $$\mprset{flushleft}
      \inferrule* [right=Cut] {
        \inferrule* [right=] {
        \inferrule* [right=,vdots=1.5em,fraction=\,] {
          \,
        }{\pi_1}          
      }{[[G |- t : A | L]]}      
      \\
      \inferrule* [right=] {
        \inferrule* [right=,vdots=1.5em,fraction=\,] {
          \,
        }{\pi_3}          
      }{[[G2,x : A,G3,z : C |- L2]]}      
    }{[[G2,G,G3,z : C |- L | [t/x]L2]]}
    }{[[G1,G2,G,G3,w : B (+) C |- h(let-pat w (y (+) -) L1) | h(h(let-pat w (- (+) z) L) | h(let-pat w (- (+) z) [t/x]L2))]]}
  \end{math}
\end{center}

Similar to the previous case.

\item[Case:] secondary hypothesis: left introduction of implication
  (first case).
The proof
\begin{center}
  \begin{math}
    $$\mprset{flushleft}
    \inferrule* [right=Cut] {
      \inferrule* [right=] {
        \inferrule* [right=,vdots=1.5em,fraction=\,] {
          \,
        }{\pi_1}          
      }{[[G |- t : A | L]]}      
      \\
      $$\mprset{flushleft}
      \inferrule* [right=Impl] {
        \inferrule* [right=] {
          \inferrule* [right=,vdots=1.5em,fraction=\,] {
            \,
          }{\pi_2}          
        }{[[G1,x : A,G2 |- t1 : B | L1]]}      
        \\
        \inferrule* [right=] {
          \inferrule* [right=,vdots=1.5em,fraction=\,] {
            \,
          }{\pi_3}          
        }{[[G3,y : C |- L2]]}      
      }{[[G1,x:A,G2,G3,z : B -o C |- L1 | [z t1/y]L2]]}
    }{[[G1,G,G2,G3,z : B -o C |- L | h(h([t/x]L1) | [t/x]h([z t1/y]L2))]]}
  \end{math}
\end{center}
transforms into the proof
\begin{center}
  \begin{math}
    $$\mprset{flushleft}
    \inferrule* [right=Impl] {
      $$\mprset{flushleft}
      \inferrule* [right=Cut] {
        \inferrule* [right=] {
        \inferrule* [right=,vdots=1.5em,fraction=\,] {
          \,
        }{\pi_1}          
      }{[[G |- t : A | L]]}      
      \\
      \inferrule* [right=] {
          \inferrule* [right=,vdots=1.5em,fraction=\,] {
            \,
          }{\pi_2}          
        }{[[G1,x : A,G2 |- t1 : B | L1]]}      
      }{[[G1,G,G2 |- h(L | h{[t/x]t1} : B) | [t/x]L1]]}
      \\
      \inferrule* [right=] {
          \inferrule* [right=,vdots=1.5em,fraction=\,] {
            \,
          }{\pi_3}          
        }{[[G3,y : C |- L2]]}      
    }{[[G1,G,G2,G3, z : B -o C |- h(L | h([t/x]L1)) | [z {[t/x]t1}/y]L2]]}
  \end{math}
\end{center}
By inspection of the above derivations we can see that $[[x nin
FV(L2)]]$, and hence, by this fact and substitution distribution
(Lemma~\ref{lemma:substitution_distribution}) we know 
$[[ [t/x]h([z t1/y]L2) = h([{[t/x]z} {[t/x]t1}/y][t/x]L2)]] = [[h([z {[t/x]t1}/y]L2)]]$.

\item[Case:] secondary hypothesis: left introduction of implication
  (second case).
The proof
\begin{center}
  \begin{math}
    $$\mprset{flushleft}
    \inferrule* [right=Cut] {
      \inferrule* [right=] {
        \inferrule* [right=,vdots=1.5em,fraction=\,] {
          \,
        }{\pi_1}          
      }{[[G |- t : A | L]]}      
      \\
      $$\mprset{flushleft}
      \inferrule* [right=Impl] {
        \inferrule* [right=] {
          \inferrule* [right=,vdots=1.5em,fraction=\,] {
            \,
          }{\pi_2}          
        }{[[G1 |- t1 : B | L1]]}      
        \\
        \inferrule* [right=] {
          \inferrule* [right=,vdots=1.5em,fraction=\,] {
            \,
          }{\pi_3}          
        }{[[G2,x : A,G3,y : C |- L2]]}      
      }{[[G1,G2,x:A,G3,z : B -o C |- L1 | [z t1/y]L2]]}
    }{[[G1,G2,G,G3,z : B -o C |- L | h(h([t/x]L1) | [t/x]h([z t1/y]L2))]]}
  \end{math}
\end{center}
transforms into the proof
\begin{center}
  \begin{math}
    $$\mprset{flushleft}
    \inferrule* [right=\tiny Series of Exchanges] {
      $$\mprset{flushleft}
    \inferrule* [right=Impl] {
      \inferrule* [right=] {
        \inferrule* [right=,vdots=1.5em,fraction=\,] {
          \,
        }{\pi_2}          
      }{[[G1 |- t1 : B | L1]]}      
      \\
      $$\mprset{flushleft}
      \inferrule* [right=Cut] {
        \inferrule* [right=] {
        \inferrule* [right=,vdots=1.5em,fraction=\,] {
          \,
        }{\pi_1}          
      }{[[G |- t : A | L]]}      
      \\
      \inferrule* [right=] {
          \inferrule* [right=,vdots=1.5em,fraction=\,] {
            \,
          }{\pi_3}          
        }{[[G2,x : A,G3,y : C |- L2]]}      
      }{[[G2,G,G3,y : C |- L | [t/x]L2]]}
    }{[[G1,G2,G,G3,z : B -o C |- L1 | h(h([z t1/y]L) | h([z t1/y][t/x]L2))]]}
  }{[[G1,G2,G,G3,z : B -o C |- h(h([z t1/y]L) | h(L1 | h([z t1/y][t/x]L2)))]]}
  \end{math}
\end{center}
By inspection of the above proofs we can see that $[[y nin
FV(L)]]$. Thus, $[[ [z t1/y]L = L]]$.  The same can be said for the
variable $[[x]]$ and context $[[L1]]$, and hence, $[[ [t/x]L1 = L1]]$.
Finally, by inspection of the above proofs $[[x nin FV(t1)]]$ and so
by substitution distribution
(Lemma~\ref{lemma:substitution_distribution}) we know $[[ [t/x]h([z
t1/y]L2) = [z t1/y][t/x]L2]]$.

\item[Case:] secondary hypothesis: left introduction of implication
  (third case).
The proof
\begin{center}
  \begin{math}
    $$\mprset{flushleft}
    \inferrule* [right=Cut] {
      \inferrule* [right=] {
        \inferrule* [right=,vdots=1.5em,fraction=\,] {
          \,
        }{\pi_1}          
      }{[[G |- t : A | L]]}      
      \\
      $$\mprset{flushleft}
      \inferrule* [right=Impl] {
        \inferrule* [right=] {
          \inferrule* [right=,vdots=1.5em,fraction=\,] {
            \,
          }{\pi_2}          
        }{[[G1 |- t1 : B | L1]]}      
        \\
        \inferrule* [right=] {
          \inferrule* [right=,vdots=1.5em,fraction=\,] {
            \,
          }{\pi_3}          
        }{[[G2,y : C,G3,x : A |- L2]]}      
      }{[[G1,G2,z : B -o C,G3,x:A |- L1 | [z t1/y]L2]]}
    }{[[G1,G2,z : B -o C,G3,G |- L | h(h([t/x]L1) | [t/x]h([z t1/y]L2))]]}
  \end{math}
\end{center}
transforms into the proof
\begin{center}
  \begin{math}
    $$\mprset{flushleft}
    \inferrule* [right=\tiny Series of Exchanges] {
      $$\mprset{flushleft}
    \inferrule* [right=Impl] {
      \inferrule* [right=] {
        \inferrule* [right=,vdots=1.5em,fraction=\,] {
          \,
        }{\pi_2}          
      }{[[G1 |- t1 : B | L1]]}      
      \\
      $$\mprset{flushleft}
      \inferrule* [right=Cut] {
        \inferrule* [right=] {
        \inferrule* [right=,vdots=1.5em,fraction=\,] {
          \,
        }{\pi_1}          
      }{[[G |- t : A | L]]}      
      \\
      \inferrule* [right=] {
          \inferrule* [right=,vdots=1.5em,fraction=\,] {
            \,
          }{\pi_3}          
        }{[[G2,y : C,G3,x : A |- L2]]}      
      }{[[G2,y : C,G3,G |- L | [t/x]L2]]}
    }{[[G1,G2,z : B -o C,G3,G |- L1 | h(h([z t1/y]L) | h([z t1/y][t/x]L2))]]}
  }{[[G1,G2,z : B -o C,G3,G |- h(h([z t1/y]L) | h(L1 | h([z t1/y][t/x]L2)))]]}
  \end{math}
\end{center}
Similar to the previous case.

\item[Case:] secondary hypothesis: right introduction of implication.
The proof
\begin{center}
  \begin{math}
    $$\mprset{flushleft}
    \inferrule* [right=Cut] {
      \inferrule* [right=] {
        \inferrule* [right=,vdots=1.5em,fraction=\,] {
          \,
        }{\pi_1}          
      }{[[G |- t : A | L]]}      
      \\
      $$\mprset{flushleft}
      \inferrule* [right=Impr] {
        \inferrule* [right=] {
        \inferrule* [right=,vdots=1.5em,fraction=\,] {
          \,
        }{\pi_2}          
      }{[[G1,x : A,G2,y : B |- t1 : C | L1]]}      
      \\
      [[y nin FV(L1)]]
      }{[[G1,x : A,G2 |- \y.t1 : B -o C | L1]]}
    }{[[G1,G,G2 |- L | h(h{[t/x]{\y.t1}} : B -o C | [t/x]L1)]]}
  \end{math}
\end{center}
transforms into the proof
\begin{center}
  \begin{math}
    $$\mprset{flushleft}
    \inferrule* [right=Impr] {
      $$\mprset{flushleft}
      \inferrule* [right=Cut] {
        \inferrule* [right=] {
          \inferrule* [right=,vdots=1.5em,fraction=\,] {
            \,
          }{\pi_1}          
        }{[[G |- t : A | L]]}      
        \\        
          \inferrule* [right=] {
            \inferrule* [right=,vdots=1.5em,fraction=\,] {
              \,
            }{\pi_2}          
          }{[[G1,x : A,G2,y : B |- t1 : C | L1]]}      
        }{[[G1,G,G2,y : B |- L | h(h{[t/x]t1} : C | [t/x]L1)]]}
      }{[[G1,G,G2 |- L | h(h{\y.[t/x]t1} : B -o C | [t/x]L1)]]}
    \end{math}
  \end{center}
Clearly, $[[ [t/x]{\y.t1} = \y.[t/x]t1]]$.

\item[Case:] secondary hypothesis: left introduction of tensor unit.
The proof 
\begin{center}
  \begin{math}
    $$\mprset{flushleft}
    \inferrule* [right=Cut] {
      \inferrule* [right=] {
          \inferrule* [right=,vdots=1.5em,fraction=\,] {
            \,
          }{\pi_1}          
        }{[[G |- t : A | L]]}      
        \\
        $$\mprset{flushleft}
        \inferrule* [right=Il] {
          \inferrule* [right=] {
          \inferrule* [right=,vdots=1.5em,fraction=\,] {
            \,
          }{\pi_2}          
        }{[[G1, x : A, G2 |- L1]]}      
      }{[[G1,x:A,G2,y : I |- let y be stp in L1]]}
    }{[[G1,G,G2,y : I |- L | [t/x](let y be stp in L1)]]}
  \end{math}
\end{center}
transforms into the proof
\begin{center}
  \begin{math}
    $$\mprset{flushleft}
    \inferrule* [right=Il] {
      $$\mprset{flushleft}
      \inferrule* [right=Cut] {
        \inferrule* [right=] {
          \inferrule* [right=,vdots=1.5em,fraction=\,] {
            \,
          }{\pi_1}          
        }{[[G |- t : A | L]]}      
        \\
        \inferrule* [right=] {
          \inferrule* [right=,vdots=1.5em,fraction=\,] {
            \,
          }{\pi_2}          
        }{[[G1, x : A, G2 |- L1]]}      
      }{[[G1, G, G2 |- L | [t/x]L1]]}
    }{[[G1, G, G2, y : I |- h(let y be stp in L) | let y be stp in [t/x]L1]]}
  \end{math}
\end{center}
It suffices to show that $[[L]] = [[let y be stp in L]]$ and
$[[ [t/x](let y be stp in L1)]] = [[let y be stp in [t/x]L1]]$.
Without loss of generality suppose $[[L]] = [[t : B, L']]$.  We know
that it must be the case that $[[y nin FV(t)]]$, and we know that
$[[ [y/z]t]] = [[t]]$ when $[[z nin FV(t)]]$.  Then by
$\FILLdrulename{Eq\_Eta2I}$ we have $[[t]] = [[let y be stp in t]]$.  This
argument can be repeated for any other term in $[[L']]$.  Thus,
$[[L]] = [[let y be stp in L]]$.  It is easy to see that
$[[ [t/x](let y be stp in L1)]] = [[let y be stp in [t/x]L1]]$ using
the rule $\FILLdrulename{Eq\_NatI}$.

\item[Case:] secondary hypothesis: right introduction of par unit.
The proof
\begin{center}
  \begin{math}
    $$\mprset{flushleft}
    \inferrule* [right=Cut] {
      \inferrule* [right=] {
        \inferrule* [right=,vdots=1.5em,fraction=\,] {
            \,
          }{\pi_1}          
        }{[[G |- t : A | L]]}      
        \\
        $$\mprset{flushleft}
        \inferrule* [right=Pr] {
          \inferrule* [right=] {
          \inferrule* [right=,vdots=1.5em,fraction=\,] {
            \,
          }{\pi_2}          
        }{[[G1,x : A, G2 |- L1]]}      
        }{[[G1,x : A, G2 |- o : _|_ | L1]]}
      }{[[G1,G, G2 |- L | h(h{[t/x]o} : _|_ | [t/x]L1)]]}
  \end{math}
\end{center}
transforms into the proof
\begin{center}
  \begin{math}
    $$\mprset{flushleft}
    \inferrule* [right=\tiny Series of Exchanges] {
      $$\mprset{flushleft}
    \inferrule* [right=Pr] {
      $$\mprset{flushleft}
      \inferrule* [right=Cut] {
        \inferrule* [right=] {
        \inferrule* [right=,vdots=1.5em,fraction=\,] {
            \,
          }{\pi_1}          
        }{[[G |- t : A | L]]}      
        \\
        \inferrule* [right=] {
          \inferrule* [right=,vdots=1.5em,fraction=\,] {
            \,
          }{\pi_2}          
        }{[[G1,x : A, G2 |- L1]]}      
      }{[[G1,G, G2 |- L | [t/x]L1]]}
    }{[[G1,G, G2 |- o : _|_ | h(L | [t/x]L1)]]}
  }{[[G1,G, G2 |- L | h(o : _|_ | [t/x]L1)]]}
  \end{math}
\end{center}
Clearly, $[[ [t/x]o = o]]$.

\item[Case:] secondary hypothesis: left introduction of exchange.
The proof
\begin{center}
  \begin{math}
    $$\mprset{flushleft}
    \inferrule* [right=Cut] {
      \inferrule* [right=] {
        \inferrule* [right=,vdots=1.5em,fraction=\,] {
            \,
          }{\pi_1}          
        }{[[G |- t : A | L]]}      
        \\
        $$\mprset{flushleft}
        \inferrule* [right=Exl] {
          \inferrule* [right=] {
        \inferrule* [right=,vdots=1.5em,fraction=\,] {
            \,
          }{\pi_2}          
        }{[[G1,x : A,G2, w : B, y : C, G3 |- L1]]}      
      }{[[G1,x : A,G2, y : C, w : B, G3 |- L1]]}
    }{[[G1,G,G2, y : C, w : B, G3 |- L | [t/x]L1]]}
  \end{math}
\end{center}
tranforms into the proof
\begin{center}
  \begin{math}
    $$\mprset{flushleft}
    \inferrule* [right=Exl] {
      $$\mprset{flushleft}
      \inferrule* [right=Cut] {
        \inferrule* [right=] {
        \inferrule* [right=,vdots=1.5em,fraction=\,] {
            \,
          }{\pi_1}          
        }{[[G |- t : A | L]]}      
        \\
        \inferrule* [right=] {
        \inferrule* [right=,vdots=1.5em,fraction=\,] {
            \,
          }{\pi_2}          
        }{[[G1,x : A,G2, w : B, y : C, G3 |- L1]]}      
      }{[[G1,G,G2, w : B, y : C, G3 |- L | [t/x]L1]]}
    }{[[G1,G,G2, y : C, w : B,G3 |- L | [t/x]L1]]}
  \end{math}
\end{center}
Clearly, all terms are equivalent.

\item[Case:] secondary hypothesis: right introduction of exchange.
The proof
\begin{center}
  \begin{math}
    $$\mprset{flushleft}
    \inferrule* [right=Cut] {
      \inferrule* [right=] {
        \inferrule* [right=,vdots=1.5em,fraction=\,] {
            \,
          }{\pi_1}          
        }{[[G |- t : A | L]]}      
        \\
        $$\mprset{flushleft}
        \inferrule* [right=Exr] {
          \inferrule* [right=] {
        \inferrule* [right=,vdots=1.5em,fraction=\,] {
            \,
          }{\pi_2}          
        }{[[G1, x : A, G2 |- L1 | h(h(t1 : B | t2 : C) | L2)]]}      
      }{[[G1, x : A, G2 |- L1 | h(h(t2 : C | t1 : B) | L2)]]}
    }{[[G1, G, G2 |- L | h(h([t/x]L1) | h(h(h{[t/x]t2} : C | h{[t/x]t1} : B) | [t/x]L2))]]}
  \end{math}
\end{center}
is transformed into 
\begin{center}
  \begin{math}
    $$\mprset{flushleft}
    \inferrule* [right=Exr] {
      $$\mprset{flushleft}
      \inferrule* [right=Cut] {
        \inferrule* [right=] {
        \inferrule* [right=,vdots=1.5em,fraction=\,] {
            \,
          }{\pi_1}          
        }{[[G |- t : A | L]]}      
        \\
        \inferrule* [right=] {
        \inferrule* [right=,vdots=1.5em,fraction=\,] {
            \,
          }{\pi_2}          
        }{[[G1, x : A, G2 |- L1 | h(h(t1 : B | t2 : C) | L2)]]}      
      }{[[G1, G, G2 |- L | h(h([t/x]L1) | h(h(h{[t/x]t1} : B | h{[t/x]t2} : C) | [t/x]L2))]]}
    }{[[G1, G, G2 |- L | h(h([t/x]L1) | h(h(h{[t/x]t2} : C | h{[t/x]t1} : B) | [t/x]L2))]]}
  \end{math}
\end{center}
Clearly, all terms are equivalent.
\end{report}
\end{itemize}
\end{proof}

\begin{corollary}[Cut-Elimination]
  \label{corollary:cut-elimination}
  Cut-elimination holds for FILL.
\end{corollary}
% section cut-elimination (end)

\section{Full LNL Models}
\label{sec:full_lnl_models}
One of the difficult questions considering the categorical models of
linear logic was how to model Girard's exponential, $!$, which is read
``of course''.  The $!$ modality can be used to translate
intuitionistic logic into intuitionistic linear logic, and so the
correct categorical interpretation of $!$ should involve a
relationship between a cartesian closed category, and the model of
intuitionistic linear logic.

de Paiva gave some of the first categorical models of both classical
and intuitionistic linear logic in her thesis \cite{dePaiva:1988}. She
showed that a particular dialectica category called $\dial$ is a model
of FILL where $!$ is interpreted as a comonad which produces natural
comonoids, see page 76 of \cite{dePaiva:1988}.
\begin{definition}
  \label{def:dial2sets}
  The category $\dial$ consists of 
  \begin{itemize}
  \item objects that are triples, $A = (U,X,\alpha)$, where $U$ and
    $X$ are sets, and $\alpha \subseteq U \times X$ is a relation, and
  \item maps that are pairs $(f,F) : (U,X,\alpha) \to (V,Y,\beta)$
    where $f : U \to V$ and $F : Y \to X$ such that
    \begin{itemize}
    \item For any $u \in U$ and $y \in Y$, $\alpha(u,F(y))$ implies $\beta(f(u),y)$.
    \end{itemize}
  \end{itemize}
  Suppose $A = (U,X,\alpha)$, $B = (V,Y,\beta)$, and
  $C = (W,Z,\gamma)$.  Then identities are given by
  $(\id_U,\id_X) : A \to A$.  The composition of the maps
  $(f,F) : A \to B$ and $(g, G) : B \to C$ is defined as
  $(f;g,G;F) : A \to C$.
\end{definition}
In her thesis de Paiva defines a particular class of dialectica
categories called $GC$ over a base category $C$, see page 41 of
\cite{dePaiva:1988}.  The category $\dial$
defined above can be seen as an instantiation of $GC$ by setting $C$
to be the category $\sets$ of sets and functions between them.

Seely gave a different, syntactic categorical model that confirmed
that the of-course exponential should be modeled by a comonad
\cite{Seely:1989}.  However, Seely's model turned out to be unsound,
as pointed out by Bierman \cite{Bierman:1994}.  This then prompted
Bierman, Hyland, de Paiva, and Benton to define another categorical
model called linear categories (Definition~\ref{def:linear-cat}) that
are sound, and also model $!$ using a monoidal comonad
\cite{Bierman:1994}.  
\begin{definition}
  \label{def:linear-cat}
  A \textbf{linear category}, $\mathcal{L}$, consists of:
  \begin{center}
    \begin{itemize}
    \item A symmetric monoidal closed category $\mathcal{L}$,
    \item A symmetric monoidal comonad $(!, \epsilon, \delta, m_{A,B},
      m_I)$ such that 
      \begin{itemize}
      \item For every free $!$-coalgebra $(!A,\delta_A)$ there are two
        distinguished monoidal natural transformations $e_A : !A \to
        I$ and $d_A : !A \to !A \otimes !A$ which form a commutative
        comonoid and are coalgebra morphisms.
      \item If $f : (!A,\delta_A) \to (!B,\delta_B)$ is a coalgebra
        morphism between free coalgebras, then it is also a comonoid
        morphism.
      \end{itemize}
    \end{itemize}
  \end{center}
  This definition is the one given by Bierman in his thesis, see
  \cite{Bierman:1994} for full definitions.
\end{definition}
\noindent
Intuitionistic logic can be interpreted in a linear category as a
full subcategory of the category of $!$-coalgebras for the comonad,
see proposition 17 of \cite{Bierman:1994}.

Benton gave a more balanced view of linear categories called LNL
models.
\begin{definition}
  \label{def:LNL-model}
  A \textbf{linear/non-linear model (LNL model)} consists of
  \begin{itemize}
  \item a cartesian closed category $(\cat{C}, 1, \times,
    \Rightarrow)$,
  \item a SMCC $(\cat{L},I,\otimes,\limp)$, and
  \item a pair of symmetric monoidal functors $(G,n) : \cat{L} \to
    \cat{C}$ and $(F,m) : \cat{C} \to \cat{L}$ between them that form
    a symmetric monoidal adjunction with $F \dashv G$.
  \end{itemize}
  See Benton, \cite{Benton:1994}, for the definitions of symmetric
  monoidal functors and adjunctions.
\end{definition}
A non-trivial consequence of the definition of a LNL model is that the
$!$ modality can indeed be interpreted as a monoidal comonad.  Suppose
$(\cat{L}, \cat{C},F,G)$ is a LNL model. Then the comonad is given by
$(\mathop{!}, \epsilon : \mathop{!} \to \mathsf{Id}, \delta :
\mathop{!} \to \mathop{!!})$
where $! = FG$, $\epsilon$ is the counit of the adjunction and
$\delta$ is the natural transformation $\delta_A = F(\eta_{G(A)})$,
see page 15 of \cite{Benton:1994}.  We recall the following result
from Benton \cite{Benton:1994}:
\begin{theorem}[LNL Models and Linear Categories, page 16 of \cite{Benton:1994}]
  \label{thm:lnl_models_are_linear_categories}
  \begin{itemize}
  \item[]
  \item[i.] Every LNL model is a linear category.
  \item[ii.] Every linear category is a LNL model.
  \end{itemize}
\end{theorem}
\begin{proof}
  The proof of part i. is a matter of checking that each part of the
  definition of a linear category can be constructed using the
  definition of a LNL model. See lemmata 3-7 of \cite{Benton:1994}.

  As for the proof of part ii. Given a linear category we have a SMCC and so the difficulty of
  proving this result is constructing the CCC and the adjunction
  between both parts of the model.  Suppose $\cat{L}$ is a linear
  category.  Benton constructs the CCC out of the full subcategory of
  Eilenberg-Moore category $\cat{L}^!$ whose objects are exponentiable
  coalgebras denoted $\mathsf{Exp}(\cat{L}^!)$.  He shows that
  this subcategory is cartesian closed, and contains the (co)Kleisli
  category, $\cat{L}_!$, Lemma~11 on page 23 of \cite{Benton:1994}.
  As for the adjunction
  $F : \mathsf{Exp}(\cat{L}^!) \to L : G$ can be defined using the
  adjunct functors $F(A,h_A) = A$ and $G(A) = (!A,\delta_A)$, see
  lemmata 13 - 16 of \cite{Benton:1994}.
\end{proof}

Next we show that the category $\dial$ is a full version of a linear
category. First, we extend the definitions of linear categories and
LNL models to be equipped with the necessary categorical structure to
model par and its unit.
\begin{definition}
  \label{def:full-linear-cat}
  A \textbf{full linear category}, $\mathcal{L}$, consists of a linear
  category \\$(\mathcal{L}, \top, \otimes, \limp,e_A,d_A)$, a symmetric
  monoidal structure on $L$, $(\perp, \parr)$, and distribution
  natural transformations $\mathsf{dist}_1 : A \otimes (B \parr C) \to
  (A \otimes B) \parr C$ and $\mathsf{dist}_2 : (A \parr B) \otimes C
  \to A \parr (B \otimes C)$.
\end{definition}
\begin{definition}
  \label{def:full-lnl-model}
  A \textbf{full linear/non-linear model (full LNL model)} consists of
  a LNL model $(\cat{L}, \cat{C},F,G)$, a symmetric monoidal structure
  on $L$, $(\perp, \parr)$, as above.
\end{definition}

\noindent
Our result is to first prove that $\dial$ is a full linear category,
and then using the proof by Benton that linear categories are LNL
models we obtain that $\dial$ is a full LNL model, but in order for
this to work we need to know that $\dial$ has a symmetric monoidal
comonad $(!, \epsilon, \delta, m_{A,B}, m_I)$.  However, at the time
of de Paiva's thesis it was not known that the comonad modeling the
of-course exponential needed to be monoidal.  We were able to show
that the maps $m_{A,B}$ and $m_I$ exist in the more general setting of
dialectica categories, and thus, these maps exist in
$\dial$. Intuitively, given two objects $A = (X,U,\alpha)$ and $B =
(V,Y,\beta)$ of $\dial$ the map $m_{A,B}$ is defined as the pair
$(\id_{U \times V},F)$, where $F = (F_1,F_2)$, $F_1 : (U \times V)
\Rightarrow (V \Rightarrow X)^* \to V \Rightarrow (U \Rightarrow X^*)$
and $F_2 : (U \times V) \Rightarrow (U \Rightarrow Y)^* \to U
\Rightarrow (V \Rightarrow Y^*)$.  The maps $F_1$ and $F_2$ build the
sequence of all the results of applying each function in the input
sequence to the input coordinate.

We can now show our main result of this section.
\begin{lemma}
  \label{lemma:ddial_is_a_linear_category}
  The category $\dial$ is a full linear category.
\end{lemma}
\begin{proof}
  \begin{paper}
    We only give a sketch of the proof here, but for the full details
    see that companion report \cite{Eades:2015}.  First, we must show
    that $\dial$ is a linear category.  The majority of the linear
    structure of $\dial$ is in de Paiva's thesis \cite{dePaiva:1988}.
    However, we had to extend her definitions to show that the comonad
    $(!A,\delta,\epsilon)$ is monoidal, however, this is
    straightforward.

    After showing that $\dial$ is a linear category one must show that
    $\dial$ is a model of par and its unit.  This easily follows from
    de Paiva's thesis.  The bifunctor which models par is given by de
    Paiva in Definition 10 on page 47 of \cite{dePaiva:1988}.

    Finally, $\dial$ must be distributive.  The natural
    transformations $dist_1$ and $dist_2$ can be defined in terms of
    the maps $k : (A \otimes A') \otimes (B \parr C) \to (A \otimes B)
    \parr (A' \otimes C)$ and $k' : (A \parr B) \otimes (C \otimes C')
    \to (A \otimes C) \parr (B \otimes C')$ given on page 52 of
    \cite{dePaiva:1988}.  Set $A' = \top$ in $k$ and $C = \top$ in
    $k'$ to obtain $dist_1$ and $dist_2$ respectively.
  \end{paper}
\begin{report}
  This proof holds by constructing each piece of a full linear
  category using the structure of $\dial$.  We use some notation to
  make it easier to define and use functions over sequences.  Given a
  function $g : A \to X^*$ we will denote taking the $i$th projection
  of the sequence returned by $g(a)$ for some $a \in A$ by $g(a)_i$.
  To construct set-theoretic anonymous functions we use
  $\lambda$-notation.  Lastly, we often use let-expressions to pattern
  match on sequences.  For example, $\lett (x_1,\ldots,x_i) = g(a)
  \inn (f(x_1),\cdots,f(x_i))$.
    
  First, we must construct a linear category. It is well known that
  $\mathsf{Sets}$ is a CCC, and in fact, locally cartesian closed, and
  so by using the results of de Paiva's thesis we can easily see that
  $\dial$ is symmetric monoidal closed:
  \begin{center}
    \begin{itemize}
    \item (Definition 7, page 43 of \cite{dePaiva:1988}). Suppose
      $A,B \in \obj{\dial}$.  Then there are sets $X$, $Y$, $V$, and
      $U$, and relations $\alpha \subseteq U \times X$ and
      $\beta \subseteq V \times Y$, such that, $A = (U,X,\alpha)$ and
      $B = (V,Y,\beta)$.  The tensor product of $A$ and $B$ can now be
      defined by $A \otimes B = (U \times V, (V \Rightarrow X) \times
      (U \Rightarrow Y),\alpha \otimes \beta)$, where $(- \Rightarrow
      -)$ is the internal hom of $\mathsf{Sets}$.  We define
      $((u,v),(f,g)) \in \alpha \otimes \beta$ if and only if
      $(u,f(v)) \in \alpha$ and $(v,g(u)) \in \beta$.  

      Suppose $A = (U,X,\alpha)$, $B = (V,Y,\beta)$,
      $C = (W,Z,\gamma)$, and $D = (S,T,\delta)$ objects of $\dial$,
      and $m_1 = (f, F) : A \to C$ and $m_2 = (g,G) : B \to D$ are
      maps of $\dial$.  Then the map $m_1 \otimes m_2 : A \otimes B
      \to C \otimes D$ is defined by $(f \times g, F_\otimes)$ where
      $f \times g$ is the ordinary cartesian product functor in
      $\sets$, and we define $F \otimes G$ as follows:
      \begin{center}
        \begin{math}
          \begin{array}{lll}
            F_\otimes : (S \Rightarrow Z) \times (W \Rightarrow T) \to (V \Rightarrow X) \times (U \Rightarrow Y)\\
            F_\otimes(h_1,h_2) = (\lambda v.F(h_1(g(v))),\lambda u.G(h_2(f(u))))
          \end{array}
        \end{math}
      \end{center}
      It is straightforward to confirm the relation condition on maps
      for $m_1 \otimes m_2$.

    \item (Definition 7, page 44 of \cite{dePaiva:1988}). Suppose
      $1 \in \obj{\sets}$ is the final object, and
      $\id_1 \subseteq 1 \times 1$.  Then we can define tensors
      unit by the object $\top = (1,1,\id_1)$.

    \item Suppose $A = (U,X,\alpha)$ is an object of $\dial$.  Then
      the map $\lambda_A : \top \otimes A \to A$ is defined by
      $(\hat{\lambda}_U,F_\lambda)$ where
      $\hat{\lambda}_U$ is the left unitor for the cartesian product in $\sets$,
      $F_\lambda(x) = (\diamond,\lambda y.x) : X \to (U \Rightarrow 1) \times (1 \Rightarrow X)$, 
      and $\diamond$ is the terminal arrow in $\sets$.
      It is easy to see that both
      $\hat{\lambda}_U$ and $F_\lambda$ have
      inverses, and thus, $\lambda_A$ has an inverse. It is straightforward to confirm the relation condition on maps
      for $\lambda_A$ and its inverse.

    \item Suppose $A = (U,X,\alpha)$ is an object of $\dial$.  Then
      the map $\rho_A : A \otimes \top \to A$ is defined similarly to
      $\lambda_A$ given above.

    \item Suppose $A = (U,X,\alpha) \in \obj{\dial}$ and
      $B = (V,Y,\beta) \in \obj{\dial}$.  Then we define the map
      $\beta_{A,B} : A \otimes B \to B \otimes A$ by
      $(\hat{\beta}_{U,V}, \hat{\beta}_{V \Rightarrow X,U \Rightarrow
        Y})$
      where $\hat{\beta}$ is the symmetry of the cartesian product in
      $\sets$.  Again, it is straightforward to see that $\beta$ has
      an inverse, and the relation condition on maps is satisfied.

    \item Suppose $A = (U,X,\alpha) \in \obj{\dial}$,
      $B = (V,Y,\beta) \in \obj{\dial}$, and
      $C = (W,Z,\gamma) \in \obj{\dial}$.  Then we define
      $\alpha_{A,B,C} : (A \otimes B) \otimes C \to A \otimes (B \otimes C)$
      by $(\hat{\alpha}_{U,V,W}, F_\alpha)$ where
      $\hat{\alpha}_{U,V,W}$ is the associator for the cartesian
      product in $\sets$ and $F_\alpha$ is defined as follows:
      \begin{center}
        \scriptsize
        \begin{math}
          \begin{array}{lll}
            F_\alpha : ((V \times W) \Rightarrow X) \times (U \Rightarrow ((W \Rightarrow Y) \times (V \Rightarrow Z))) \to 
            (W \Rightarrow ((V \Rightarrow X) \times (U \Rightarrow Y))) \times ((U \times V) \Rightarrow Z)\\
            F_\alpha(h_1,h_2) = (\lambda w.(\lambda v.h_1(v,w),\lambda u.h_2(u)_1(w)), \lambda (u,v).h_2(u)_2(v))\\
          \end{array}
        \end{math}
      \end{center}
      The inverse of $\alpha_{A,B,C}$ is similar, and it is
      straightforward to confirm the relation condition on maps.

    \item (Definition 9, page 44 of \cite{dePaiva:1988}). Suppose
      $A,B \in \obj{\dial}$.  Then there are sets $X$, $Y$, $V$, and
      $U$, and relations $\alpha \subseteq U \times X$ and
      $\beta \subseteq V \times Y$, such that, $A = (U,X,\alpha)$ and
      $B = (V,Y,\beta)$.  Then we define the internal hom of $\dial$
      by $A \limp B = ((U \Rightarrow V) \times (Y \Rightarrow X),U
      \times Y,\alpha \Rightarrow \beta)$.  We define $((f,g),(u,y))
      \in \alpha \Rightarrow \beta$ if and only if whenever $(u,g(y))
      \in \alpha$, then $(f(u),y) \in \beta$.  The locally cartesian 
      closed structure of $\sets$ guarantees that for any two objects $A, B \in
      \dial$ the internal hom $A \limp B \in \dial$ exists.      
    \end{itemize}
  \end{center}
  Using the constructions above $\dial$ is a SMCC by Proposition 24
  on page 46 of \cite{dePaiva:1988}.  

  Next we define the symmetric monoidal comonad
  $(!,\epsilon,\delta,m_{A,B},m_I)$ of the linear category:
  \begin{center}
    \begin{itemize}
    \item (Section 4.5, on page 76 of \cite{dePaiva:1988}). The
      endofunctor $! : \dial \to \dial$ is defined as follows:
      \begin{itemize}
      \item Objects. Suppose $A = (U,X,\alpha) \in \obj{\dial}$.  Then
        we set $!A = (U,U \Rightarrow X^*, !\alpha)$, where
        $(u,f) \in !\alpha$ if and only if
        $(u,f(u)_1) \in \alpha \text{ and } \cdots \text{ and }
        (u,f(u)_i) \in \alpha$ where $f(u)$ is a sequence of length $i$.

      \item Morphisms. Suppose $A = (U,X,\alpha) \in \obj{\dial}$,
        $B = (V,Y,\beta) \in \obj{\dial}$, and
        $(f,F) : A \to B \in \mor{\dial}$.  Then we define 
        $!(f,F) = (f,!F) : !A \to !B$, where $!F(g) = \lambda
        x.F^*(g(f(x))) : V \Rightarrow Y^* \to U \Rightarrow X^*$.
      \end{itemize}

    \item (Section 4.5, page 77 of \cite{dePaiva:1988}). The
      endofunctor $!$ defined above is the functor part of the comonad
      $(!,\epsilon, \delta)$.  Suppose
      $A = (U,X,\alpha) \in \obj{\dial}$. Then the co-unit
      $\epsilon : !A \to A$ is defined by $\epsilon =
      (\mathsf{id}_U,F_0)$ where $F_0(x) = \lambda y.(x) : X \to U
      \Rightarrow X^*$.  
      Furthermore, the co-multiplication $\delta_A
      : !A \to !!A$ is defined by $\delta_A = (\mathsf{id}_U,F_1)$
      where $F_1(g) = \lambda u.(f_1(u) \circ \cdots \circ f_i(u)) : U \Rightarrow (U
      \Rightarrow X^*)^* \to U \Rightarrow X^*$ where $g(u) = (f_1,\ldots,f_i)$.

    \item The following diagrams commute:
      \begin{center}
        \begin{math}
          \begin{array}{lll}
            \bfig
            \square[!A`!!A`!!A`!!!A;\delta_A`\delta_A`!\delta_A`\delta_{!A}]
            \efig
            &
              \,\,\,\,\,\,\,\,\,\,\,\,\,\,\,\,\,\,\,\,
            &
              \bfig
             \Atrianglepair/=`->`=`<-`->/[!A`!A`!!A`!A;`\delta_A``\epsilon_{!A}`!\epsilon_A]
           \efig
          \end{array}
        \end{math}
      \end{center}
      We show the left most diagram commutes first.  It suffices to
      show that
      $\delta_a;!\delta_A = (\id_U,!F_1;F_1) = (\id_U,F_1;F_1)$.  
      Suppose $g \in U \Rightarrow (U \Rightarrow X^*)^*$.  Then       
      \begin{center}
        \small
        \begin{math}
          \begin{array}{lll}
            F_1(F_1(g)) \\
            \,\,= F_1(\lambda u.g(u)_1(u) \circ \cdots \circ g(u)_i(u))\\
            \,\,= \lambda u.g(u)_1(u)_1(u) \circ \cdots \circ g(u)_1(u)_j(u) \circ \cdots \circ g(u)_i(u)_1(u) \circ \cdots \circ g(u)_i(u)_k(u)\\
          \end{array}
        \end{math}
      \end{center}
      Consider the other direction in the diagram.  
      \begin{center}
        \begin{math}
          \begin{array}{lll}
            F_1(!F_1(g)) 
            & = & F_1(\lambda x.F^*_1(g(x)))\\
            & = & \lambda u.F_1(g(u)_1)(u) \circ \cdots \circ F_1(g(u)_i)(u)\\            
          \end{array}
        \end{math}
      \end{center}
      Note that we have the following:
      \begin{center}
        \begin{math}
          \begin{array}{lll}
            F_1(g(u)_1)(u) & = & g(u)_1(u)_1(u) \circ \cdots \circ g(u)_1(u)_k(u)\\
            & \vdots & \\
            F_1(g(u)_i)(u) & = & g(u)_i(u)_1(u) \circ \cdots \circ g(u)_i(u)_k(u)\\
          \end{array}
        \end{math}
      \end{center}
      Clearly, the above reasoning implies that $F_1;F_1 = !F_1;F_1$.

      Now we prove that the second diagram commutes, but we break it
      into two.  
      We define $\delta_A;!\epsilon_A = (\id_U,!F_0;F_1)$ where for
      any $g \in U \Rightarrow X^*$,
      \begin{center}
        \begin{math}
          \begin{array}{lll}
            (!F_0;F_1)(g) 
            & = & F_1(!F_0(g))\\
            & = & F_1(\lambda u'.F_0^*(g(u')))\\
            & = & F_1(\lambda u'.(\lambda y.(g(u')_1),\ldots,(\lambda y.g(u')_i)))\\
            & = & \lambda u.(g(u)_1) \circ \cdots \circ (g(u)_i)\\
            & = & g\\
          \end{array}
        \end{math}
      \end{center}
      and we can define $\delta_A;\epsilon_{!A} = (\id_U,F_0;F_1)$ where for
      any $g \in U \Rightarrow X^*$,
      \begin{center}
        \begin{math}
          \begin{array}{lll}
            (F_0;F_1)(g) 
            & = & F_1(F_0(g))\\
            & = & F_1(\lambda y.(g))\\
            & = & \lambda u.g(u)\\
            & = & g\\
          \end{array}
        \end{math}
      \end{center}
      We can see by the reasoning above that
      $!F_0;F_1 = F_0;F_1 = \id_{U \Rightarrow X^*}$.

    \item The monoidal natural transformation $m_{\top} : \top \to !\top$
      is defined by $m_{\top} = (\id_1,\lambda f.\star)$ where $\star \in \top$. 
      It is easy to see that the relation condition
      on maps for $\dial$ is satisfied.  The following two diagrams
      commute:
      \begin{center}
        \begin{math}
          \begin{array}{lll}
            \bfig
            \square[\top`!\top`!\top`!!\top;m_\top`m_\top`\delta_\top`!m_\top]
            \efig
            &
              \,\,\,\,\,\,\,\,\,\,\,\,\,\,\,\,\,\,\,\,
            &
              \bfig
              \btriangle/->`=`->/[\top`!\top`\top;m_\top``\epsilon_{\top}]
           \efig
          \end{array}
        \end{math}
      \end{center}
      It is straightforward to see that the above two diagrams commute
      using the fact that the second coordinate of $m_\top$ is a
      constant function.

    \item The monoidal natural transformation
      $m_{A,B} : !A \otimes !B \to !(A \otimes B)$ is defined by
      $m_{A,B} = (\id_{U \times V},F_2)$. We need to define $F_2$,
      but two auxiliary functions are needed first:
      \begin{center}
        \begin{math}
          \begin{array}{lll}
            \begin{array}{lll}
              h_1 : (U \times V) \Rightarrow ((V \Rightarrow X) \times (U \Rightarrow Y))^* \to 
                    (V \Rightarrow (U \Rightarrow X^*))\\
              h_1(g,v,u) = (f_1(v),\ldots,f_i(v))
              \text{ where } g(u,v) = ((f_1,g_1),\ldots,(f_i,g_i))
            \end{array}
            \\
            &  \\
            \begin{array}{lll}
              h_2 : (U \times V) \Rightarrow ((V \Rightarrow X) \times (U \Rightarrow Y))^* \to 
                    (U \Rightarrow (V \Rightarrow Y^*))\\
              h_2(g,u,v) = (g_1(u),\ldots,g_i(u))
              \text{ where } g(u,v) = ((f_1,g_1),\ldots,(f_i,g_i))
            \end{array}
          \end{array}
        \end{math}
      \end{center}
      Then $F_2(g) = (h_1(g),h_2(g))$.  In order for $m_{A,B}$ to be
      considered a full fledge map in $\dial$ we have to verify that
      the relation condition on maps is satisfied.  Suppose $(u,v) \in U \times V$ and 
      $g \in (U \times V) \Rightarrow ((V \Rightarrow X) \times (U \Rightarrow Y))^*$, 
      where $g(u,v) = ((f_1,g_1), \ldots,(f_i,g_i))$. Then we know the following by definition:
      \begin{center}
        \begin{math}
          \begin{array}{lll}
            ((u,v),F(g)) \in !\alpha \otimes !\beta 
            & \text{ iff } & ((u,v),(h_1(g),h_2(g))) \in !\alpha \otimes !\beta\\
            & \text{ iff } & (u,h_1(g)(v)) \in !\alpha \text{ and } (v,h_2(g)(u)) \in !\beta\\
            & \text{ iff } & (u,f_1(v)) \in \alpha \text{ and } \cdots \text{ and } (u,f_i(v)) \text{ and }\\
            &              & (v,g_1(u)) \in \beta \text{ and } \cdots \text{ and } (v,g_i(u)) 
          \end{array}
        \end{math}
      \end{center}
      and
      \begin{center}
        \begin{math}
          \begin{array}{lll}
            ((u,v),g) \in !(\alpha \otimes \beta) 
            & \text{ iff } & ((u,v),(f_1,g_1)) \in \alpha \otimes \beta \text{ and } \cdots \text{ and } \\
            &              & ((u,v),(f_i,g_i)) \in \alpha \otimes \beta\\
            & \text{ iff } & (u,f_1(v)) \in \alpha \text{ and } (v,g_1(u)) \in \beta \text{ and } \cdots \text{ and } \\
            &              & (u,f_i(v)) \in \alpha \text{ and } (v,g_i(u)) \in \beta
          \end{array}
        \end{math}
      \end{center}
      The previous definitions imply that
      $((u,v),F(g)) \in !\alpha \otimes !\beta$ implies
      $((u,v),g) \in !(\alpha \otimes \beta)$.  Thus, $m_{A,B}$ is a map in $\dial$.
      
      At this point we show that the following diagrams commute:
      \begin{center}
          \begin{mathpar}
            \bfig
            \square/->`<-`->`->/<700,700>[!\top \otimes !A`!(\top \otimes A)`\top \otimes !A`!A;m_{\top,A}`m_\top \otimes \id_{!A}`!\lambda_A`\lambda_{!A}]
            \place(350,350)[\text{A}]
            \efig
            \and
            \bfig
            \square/->`<-`->`->/<700,700>[!A \otimes !\top`!(A \otimes \top)`!A \otimes \top`!A;m_{A,\top}`\id_{!A} \otimes m_\top`!\rho_A`\rho_{!A}]
            \place(350,350)[\text{B}]
            \efig
          \end{mathpar}
      \end{center}
      \begin{center}
        \begin{mathpar}
            \bfig
            \square/->`->`->`->/<700,700>[!A \otimes !\top`!(A \otimes \top)`!A \otimes \top`!A;m_{A,\top}`\id_{!A} \otimes m_\top`!\rho_A`\rho_{!A}]
            \place(350,350)[\text{C}]
            \efig
            \and
            \bfig
            \square/->`->`->`->/<700,700>[!A \otimes !B`!(A \otimes !B)`!B \otimes !A`!(B \otimes A);m_{A,B}`\beta`!\beta`m_{B,A}]
            \place(350,350)[\text{D}]
            \efig
            \and 
            \bfig
            \hSquares/->`->`->``->`->`->/[(!A \otimes !B) \otimes !C`!(A \otimes B) \otimes !C`!((A \otimes B) \otimes C)`!A \otimes (!B \otimes !C)`!A \otimes !(B \otimes C)`!(A \otimes (B \otimes C));m_{A,B} \otimes \id_{!C}`m_{A \otimes B,C}`\alpha_{!A,!B,!C}``!\alpha_{A,B,C}`\id_{!A} \otimes m_{B,C}`m_{A,B \otimes C}]
            \place(1300,250)[\text{E}]
            \efig
          \end{mathpar}          
      \end{center}
      We first prove that diagram A commutes, and then diagrams B, C and D will
      commute using similar reasoning. Following this we show that diagram E commutes. It suffices to show that
      \[(m_\top \otimes id_{!A});m_{\top,A};!\lambda_A = 
      (\hat{\lambda}_U,\lambda g.F_\otimes(F_2(F_\lambda(g)))) = (\hat{\lambda}_U,\lambda g.(\diamond,\lambda t.\lambda u.g(u))).\]
      Suppose
      $g \in U \Rightarrow X^*$, $(\star,u) \in 1 \times U$, and
      $g(u) = (x_1,\ldots,x_i)$. Then 
      \begin{center}
        \begin{math}
          \begin{array}{lll}
            F_\lambda(g)(\star,u) 
            & = & (\lambda x'.(\diamond,\lambda y.x'))^*(g(\hat{\lambda}_U(\star,u)))\\
            & = & (\lambda x'.(\diamond,\lambda y.x'))^*(g(u))\\
            & = & (\lambda x'.(\diamond,\lambda y.x'))^*(x_1,\ldots,x_i)\\
            & = & ((\diamond,\lambda y.x_1),\ldots,(\diamond,\lambda y.x_i))\\
          \end{array}
        \end{math}
      \end{center}
      This implies that 
      \[ F_\lambda(g) = \lambda (\star,u).\mathsf{let}\,(x_1,\ldots,x_i) = g(u)\,\mathsf{in}\,((\diamond,\lambda y.x_1),\ldots,(\diamond,\lambda y.x_i)).\]
      Using this reasoning we can see the following:
      \begin{center}
        \begin{math}
          \begin{array}{lll}
            F_2(F_\lambda(g))
            & = & (\lambda u.\lambda t.\mathsf{let}\,((\diamond,\lambda y.x_1),\ldots,(\diamond,\lambda y.x_i)) = F_\lambda(g)(t,u)\,\mathsf{in}\,\\
                  & & (\diamond(u),\ldots,\diamond(u)),\\            
            & & \lambda t.\lambda u.\mathsf{let}\,((\diamond,\lambda y.x_1),\ldots,(\diamond,\lambda y.x_i)) = F_\lambda(g)(t,u)
                \,\mathsf{in}\, \\
            & & ((\lambda y.x_1)(t),\ldots,(\lambda y.x_1)(t)))\\
            & = & (\lambda u.\lambda t.\mathsf{let}\,((\diamond,\lambda y.x_1),\ldots,(\diamond,\lambda y.x_i)) = F_\lambda(g)(t,u)\,\mathsf{in}\,\\
            & & (\star,\ldots,\star),\\            
            & & \lambda t.\lambda u.\mathsf{let}\,((\diamond,\lambda y.x_1),\ldots,(\diamond,\lambda y.x_i)) = F_\lambda(g)(t,u)
                \,\mathsf{in}\, \\
            & & (x_1,\ldots,x_1))\\
            & = & (\lambda u.\lambda t.(\star,\ldots,\star),\\            
            & & \lambda t.\lambda u.\mathsf{let}\,((\diamond,\lambda y.x_1),\ldots,(\diamond,\lambda y.x_i)) = F_\lambda(g)(t,u)
                \,\mathsf{in}\, \\
            & & (x_1,\ldots,x_1))\\
            & = & (\lambda u.\lambda t.(\star,\ldots,\star), \lambda t.\lambda u.g(u))
          \end{array}
        \end{math}
      \end{center}
      Finally, the previous allows us to infer the following:
      \begin{center}
        \begin{math}
          \begin{array}{lll}
            F_\otimes(F_2(F_\lambda(g)))
            & = & (\diamond, \lambda t.\lambda u.g(u))\\
          \end{array}
        \end{math}
      \end{center}
      Thus, we obtained our desired result.

      \noindent
      We show that diagram C commutes by observing that
      \begin{center}
        \begin{math}
          \begin{array}{lll}
            (m_{A,B} \otimes !\id_{!C});m_{A\otimes B,C};!\alpha_{A,B,C}
            & = & (\id,F_\otimes);(\id,F_2);(\hat{\alpha},!F_\alpha)\\
            & = & (\hat{\alpha},!F_\alpha;F_2;F_\otimes)\\
            & = & (\hat{\alpha},F_2;F_\otimes;F_\alpha)\\
            & = & (\hat{\alpha},F_\alpha);(\id,F_\otimes);(\id,F_2)\\
          \end{array}
        \end{math}
      \end{center}
      It suffices to show that $!F_\alpha;F_2;F_\otimes = F_2;F_\otimes;F_\alpha$:
      \begin{center}
        \begin{math}
          \begin{array}{lll}
            (!F_\alpha;F_2;F_\otimes)(g)
            & = & F_\otimes(F_2(!F_\alpha(g)))\\
            & = & F_\otimes(F_2(\lambda x.F^*_\alpha(g(x))))\\
          \end{array}
        \end{math}
      \end{center}
      Suppose
      \begin{center}
        \vspace{-20px}
        \begin{math}
          \begin{array}{lll}
            h_1 = \lambda v.\lambda u.\mathsf{let}\,((f_1,g_1),\ldots,(f_i,g_i)) = F^*_\alpha(g(u,v))\,\mathsf{in}\,
            (f_1(v),\ldots,f_i(v)),\\
            \\
            h_2 = \lambda u.\lambda v.\mathsf{let}\,((f_1,g_1),\ldots,(f_i,g_i)) = F^*_\alpha(g(u,v)) \,\mathsf{in}\,
           (g_1(u),\ldots,g_i(u)), \text{ and }\\
           \\
           F^*_\alpha(g(u,v)) = \mathsf{let}\, ((f'_1,g'_1),\ldots,(f'_j,g'_j)) = g(u,v) \,\mathsf{in}\,\\
      \,\,\,\,\,\,\,\,\,\,(\lambda w.(\lambda v'.f_1'(v',w),\lambda u.g'_1(u)_1(w)),\lambda (u,v').g'_1(u)_2(v')),\ldots,\\
      \,\,\,\,\,\,\,\,\,\,\,(\lambda w.(\lambda v'.f_j'(v',w),\lambda u.g'_j(u)_1(w)),\lambda (u,v').g'_j(u)_2(v'))).
          \end{array}
        \end{math}
      \end{center}
      Then we can simplify $h_1$ and $h_2$ as follows:
      \begin{center}
        \begin{math}
          \begin{array}{lll}
            \begin{array}{lll}
              h_1 = \lambda v.\lambda u.\mathsf{let}\,((f'_1,g'_1),\ldots,(f'_j,g'_j)) = g(u,v)\,\mathsf{in}\,\\
              \,\,\,\,\,\,\,\,\,((\lambda v'.f_1'(v',v),\lambda u'.g'_1(u')_1(v)),\ldots,(\lambda v'.f_j'(v',v),\lambda u'.g'_j(u')_1(v))) \\
            \end{array}\\
            \text{ and }\\
            \begin{array}{lll}
              h_2 = \lambda u.\lambda v.\mathsf{let}\,(u_1,u_2) = u\,\mathsf{in}\, \\
              \,\,\,\,\,\,\,\mathsf{let}\,((f'_1,g'_1),\ldots,(f'_j,g'_j)) = g((u_1,u_2),v) \,\mathsf{in}\,\\
              \,\,\,\,\,\,\,\,\,\,\,(g'_1(u_1)_2(u_2),\ldots,g'_j(u_1)_2(u_2))
            \end{array}
          \end{array}
        \end{math}
      \end{center}
      By the definition of $F_2$ the previous reasoning implies:
      \begin{center}
        \begin{math}
          \begin{array}{lll}
            F_\otimes(F_2(\lambda x.F^*_\alpha(g(x))))
            & = & F_\otimes(h_1,h_2)\\
            & = & (\lambda v.F_2(h_1(v)),h_2)\\
          \end{array}
        \end{math}
      \end{center}
      Expanding the definition of $F_2(h_1(v))$ in the above definitions yields:
      \begin{center}
        \begin{math}
          \begin{array}{lll}
            F_2(h_1(v))
            & = & (h'_1,h'_2)\\
          \end{array}
        \end{math}
      \end{center}
      where
      \begin{center}
        \begin{math}
          \begin{array}{lll}
            h'_1 = \lambda v''.\lambda u''.(f'_1(v'',v),\ldots,f'_j(v'',v))\\
            h'_2 = \lambda u''.\lambda v''.(g'_1(u')_1(v),\ldots,g'_j(u')_1(v))\\
          \end{array}
        \end{math}
      \end{center}
      At this point we can see that
      \begin{center}
        \begin{math}
          \begin{array}{lll}
            (\lambda v.F_2(h_1(v)),h_2) & = & (\lambda v.(h'_1,h'_2),h_2)\\
          \end{array}
        \end{math}
      \end{center}
      We now simplify $F_2;F_\otimes;F_\alpha$.  We know by definition:
      \begin{center}
        \begin{math}
          \begin{array}{lll}
            F_2(g) & = & (h''_1,h''_2)
          \end{array}
        \end{math}
      \end{center}
      where
      \begin{center}
        \begin{math}
          \begin{array}{lll}
            \begin{array}{lll}
              h''_1 & = & \lambda v.\lambda u.\mathsf{let}\,((f'_1,g'_1),\ldots,(f'_j,g'_j)) = g(u,v)\,\mathsf{in}\\
              & & \,\,\,\,\,\,\,\,\,\mathsf{let}\,(v',v'') = v\,\mathsf{in}\,(f'_1(v',v''),\ldots,f'_k(v',v''))
              
            \end{array}\\
            \text{ and }\\
            \begin{array}{lll}
              h''_2 & = & \lambda u.\lambda v.\mathsf{let}\,((f'_1,g'_1),\ldots,(f'_j,g'_j)) = g(u,v)\,\mathsf{in}\\
              & & \,\,\,\,\,\,\,\,\,(g'_1(u),\ldots,g'_k(u))\\
            \end{array}
          \end{array}
        \end{math}
      \end{center}
      This implies that
      \begin{center}
        \begin{math}
          \begin{array}{lll}
            F_\alpha(F_\otimes(F_2(g)))
            & = & F_\alpha(F_\otimes(h''_1,h''_2))\\
            & = & F_\alpha(h''_1,\lambda u_4.F_2(h''_2(u_4)))\\
            & = & (\lambda w.(\lambda v.h''_1(v,w),\lambda u.F_2(h''_2(u))_1(w)),\lambda (u,v).F_2(h''_2(u))_2(v))
          \end{array}
        \end{math}
      \end{center}
      Finally, by expanding the definition of $F_2$ in the last line
      of the above reasoning we can see that
      \[ (\lambda v.(h'_1,h'_2),h_2) = (\lambda w.(\lambda v.h''_1(v,w),\lambda u.F_2(h''_2(u))_1(w)),\lambda (u,v).F_2(h''_2(u))_2(v)) \]
      modulo currying of set-theoretic functions.

    \item There are two coherence diagrams that $m_{A,B}$ and $\delta$
      must ad hear to.  They are listed as follows:
      \begin{center}
        \begin{mathpar}
          \bfig
          \square/->`->`->`=/<700,700>[!A \otimes !B`!(A \otimes B)`A \otimes B`A \otimes B;m_{A,B}`\epsilon_A \otimes \epsilon_B`\epsilon_{A \otimes B}`]
            \place(350,350)[\text{F}]
            \efig            
            \and
            \bfig
            \morphism(150,500)<1615,0>[`;m_{A,B}]
            \hSquares/``->``->`->`->/[!A \otimes !B``!(A \otimes B)`!!A \otimes !!B`!(!A \otimes !B)`!!(A \otimes B);``\delta_A \otimes \delta_B``\delta_{A \otimes B}`m_{!A,!B}`!m_{A,B}]
            \place(980,250)[\text{G}]
            \efig
        \end{mathpar}
      \end{center}
      Diagram F holds by simply expanding the definitions using an
      arbitrary input of a pair of functions.  We now show diagram G
      commutes.

      It suffices to show the following:
      \begin{center}
        \begin{math}
          \begin{array}{lll}
            m_{A,B};\delta_{A \otimes B}
            & = & (\id_{U \times V},F_1;F_2)\\
            & = & (id_{u \times V},!F_2;F_2;F_\otimes)\\
            & = & (\delta_A \otimes \delta_B);m_{!A,!B};!m_{A,B}
          \end{array}
        \end{math}
      \end{center}
      Suppose $g \in (U \times V) \Rightarrow ((U \times V)
      \Rightarrow ((V \Rightarrow X) \times (U \Rightarrow Y))^*)^*$.
      Then we know by the type of $g$ and the definition of $F_1$ it
      must be the case that $F_1(g)$ first extracts all of the
      functions $(f_1,\ldots,f_i)$ returned by $g(u,v)$ for arbitrary
      $u \in U$ and $v \in V$ -- note that each $f_i$ returns a
      sequence of pairs of functions,
      $((f'_i,g'_i),\ldots,(f'_j,g'_j))$ -- then $F_1(g)$ returns the
      concatenation of all of these sequences.  Finally, $F_2(F_1(g))$
      returns two functions $h_1(v,u)$ and $h_2(u,v)$, where $h_1$
      returns the sequence $(f'_i(v),\ldots,f'_j(v))$, and $h_2$
      returns the sequence $(g'_i(u),\ldots,g'_j(u))$ from the
      sequence returned by $F_1(g)$.  Note that each $f'_i$ and $g'_i$
      returns a pair of functions.

      Now consider applying $!F_2;F_2;F_\otimes$ to $g$.  The function
      $!F_2$ will construct the function $\lambda x.F^*_2(g(x))$ by
      definition, and $F^*_2(g(x))$ will construct a sequence of pairs
      of functions $((h'_1,h''_1),\ldots,(h'_k,h''_k))$. The function
      $g$ as we saw above returns a sequence of functions,
      $(f_1,\ldots,f_i)$, where each $f_i$ returns a sequence of pairs
      of functions, $((f'_i,g'_i),\ldots,(f'_j,g'_j))$.  This tells us
      that by definition $h'_k(v,u)$ will return the sequence
      $(f'_i(v),\ldots,f'_j(v))$ and $h''_k(u,v)$ will construct the
      sequence $(g'_1(u),\ldots,g'_j(u))$.  Applying $F_2$ to $\lambda
      x.F^*_2(g(x))$ will construct two more functions $t_1(v,u)$ and
      $t_2(u,v)$ where the first returns the sequence of functions
      $(h'_1(v),\ldots,h'_k(v))$, and the second returns
      $(h''_1(u),\ldots,h''_k(u))$.  Finally, applying the function
      $F_\otimes$ to the pair $(t_1,t_2)$ will result in a pair of
      functions
      \begin{center}
        \begin{math}
          \begin{array}{lll}
            (\lambda v.F_1(t_1(v)),\lambda u.F_1(t_2(u)))
            & = & (\lambda v.\lambda u.h'_1(v)(u) \circ \cdots \circ h'_k(v)(u),\\
            &   & \,\,\,\lambda u.\lambda v.h''_1(u)(v) \circ \cdots \circ h''_k(u)(v))\\
            & = & (\lambda v.\lambda u.(f'_i(v),\ldots,f'_j(v)),\\
            &   & \,\,\,\lambda u.\lambda v.(g'_i(u), \ldots, g'_k(u)))\\
          \end{array}
        \end{math}
      \end{center} 
      We can now see that the pair
      $(\lambda v.\lambda u.(f'_i(v),\ldots,f'_j(v)),\lambda u.\lambda v.(g'_i(u), \ldots, g'_k(u)))$ is indeed
      equivalent to the pair $(h_1,h_2)$ given above, and thus, the diagram commutes.      
    \end{itemize}
  \end{center}
  Next we must show that whenever $(!A,\delta)$ is a free comonoid, we
  have the distinguished natural transformations $e_A : !A \to \top$
  and $d_A : !A \to !A \otimes !A$.  Suppose $!A = (U, U \Rightarrow
  X^*)$ and $(!A,\delta)$ is a free comonoid.  Then we have the
  following definitions:
  \begin{itemize}
  \item (Proposition 53, page 77 of \cite{dePaiva:1988}).  We define
    $e_A : !A \to \top$ has the pair $(\diamond,\lambda x.\lambda
    u.())$, where $\diamond$ is the terminal map on $U$ and $()$ is
    the empty sequence.
    
  \item (Proposition 53, page 77 of \cite{dePaiva:1988}).  We define
    $d_A : !A \to !A \otimes !A$ has the pair $(\Delta, \theta)$
    where $\Delta : U \to U \times U$ is the diagonal map in
    $\sets$, and
    \begin{center}
      \begin{math}
        \begin{array}{lll}
          \theta : ((U \times U) \Rightarrow X^*) \times ((U \times U) \Rightarrow X^*) \to U \Rightarrow X^*\\
          \theta(f,g) = \lambda u.f(u,u) \circ g(u,u).
        \end{array}
      \end{math}
    \end{center}
  \end{itemize}
  The maps $e_A$ and $d_A$ must satisfy several coherence diagrams.
  \begin{itemize}
  \item We must show that the map $e_A$ is a monoidal natural
    transformation.  This requires that the following diagrams hold
    (for any arbitrary map $f$):
    \begin{center}
      \begin{mathpar}
        \bfig
        \square/->`->`=`->/<700,700>[!A`\top`!B`\top;e_A`!f``e_B]
        \place(350,350)[\text{H}]
        \efig
        \and
        \bfig
        \btriangle/->`=`->/<700,700>[\top`!\top`\top;m_I``e_\top]
        \place(230,250)[\text{I}]
        \efig            
        \and
        \bfig
        \square/->`->`->`->/<700,700>[!A \otimes !B`\top \otimes \top`!(A \otimes B)`\top;e_A \otimes e_B`m_{A,B}`\lambda`e_{A \otimes B}]
        \place(350,350)[\text{J}]
        \efig        
      \end{mathpar}
    \end{center}
    Diagrams H and I follow easily by the definition of $e_A$ and
    $m_I$.  We now show that diagram J commutes.  It suffices to show
    the following:
    \begin{center}
      \begin{math}
        \begin{array}{lll}
          (e_A \otimes e_B);\lambda
          & = & (\diamond_U \times \diamond_V,F_\otimes);(\hat{\lambda}_\top,F_\lambda) \\
          & = & ((\diamond_U \times \diamond_V);\hat{\lambda}_\top,F_\lambda;F_\otimes)\\
          & = & (\diamond_{U \times V},F_\lambda;F_\otimes)\\
          & = & (\diamond_{U \times V},F_2(\lambda u.()))\\
          & = & (\diamond_{U \times V},(\lambda x.\lambda u.());F_2)\\          
          & = & (\id_{U \times V};\diamond_{U \times V},(\lambda x.\lambda u.());F_2)\\
          & = & (\id_{U \times V},F_2);(\diamond_{U \times V},\lambda x.\lambda u.())\\
          & = & m_{A,B};e_{A \otimes B}
        \end{array}
      \end{math}
    \end{center}
    It suffices to show $F_\lambda;F_\otimes = F_2(\lambda u.())$, but
    this easily follows by definition.

  \item The map $d_A$ must be a monoidal natural transformation.  This
    requires the following diagrams to commute:
    \begin{center}
      \begin{mathpar}
        \bfig
        \square/->`->`->`->/<700,700>[!A`!A \otimes !A`!B`!B \otimes !B;d_A`!f`!f \otimes !f`d_B]
        \place(350,350)[\text{K}]
        \efig
        \and
        \bfig
        \square/->`->`->`->/<700,700>[\top`\top \otimes \top`!\top`!\top \otimes !\top;\lambda^{-1}`m_\top`m_\top \otimes m_\top`d_\top]
        \place(350,350)[\text{L}]
        \efig
        \and
        \bfig
        \morphism(190,0)<1880,0>[`;d_{A \otimes B}]
        \hSquares/->`->`->``->``/[!A \otimes !B`(!A \otimes !A) \otimes (!B \otimes !B)`(!A \otimes !B) \otimes (!A \otimes !B)`!(A \otimes B)``!(A \otimes B) \otimes !(A \otimes B);d_A \otimes d_B`iso`m_{A,B}``m_{A,B} \otimes m_{A,B}``]
        \place(1150,250)[\text{M}]
        \efig
      \end{mathpar}
    \end{center}
    Diagrams K and L follow easily from unfolding their
    definitions. We show that diagram M next.  The morphism $iso$ in
    $\dial$ is a isomorphism that can be built out of the SMCC
    structure.  For its definition in terms of the SMCC maps see
    footnote 9 on page 141 of \cite{Bierman:1994}, but we give a
    direct definition instead.  The second coordinate of the
    definition of the map $iso$ is defined as follows:
    \begin{center}
      \begin{math}
        \begin{array}{lll}
          \begin{array}{rll}
            \hat{iso}((u,u'),(v,v')) & = & ((u,v),(u',v'))
          \end{array}\\\\        
          \begin{array}{rll}            
            F_{iso}(f,g) & = & (\lambda (v',v'').(\lambda u'.\lambda u''.f(u',v')_1(v'',u''),\lambda u'.\lambda u''.g(u',v')_1(v'',u'')),\\
            && \,\,\lambda (u',u'').(\lambda v'.\lambda v''.f(u',v')_2(u'',v''),\lambda v'.\lambda v''.g(u',v')_1(u'',v'')))\\
          \end{array}\\\\
          \begin{array}{rll}
            F_{iso}^{-1}(h_1,h_2) & = & (\lambda (u,v).(\lambda v'.\lambda u'.h_1(v,v')_1(u,u'),\lambda u'.\lambda v'.h_2(u,u')_2(v,v')),\\
            && \,\,\lambda (u,v).(\lambda v'.\lambda u'.h_1(v,v')_2(u,u'),\lambda u'.\lambda v'.h_2(u,u')_2(v,v')))\\
          \end{array}
        \end{array}
      \end{math}
    \end{center}
    We omit the proof that $F_{iso}$ is an isomorphism, but it is
    straightforward.  Now $iso = (\hat{iso},F_{iso})$.

    It suffices to show the following:
    \begin{center}
      \begin{math}
        \begin{array}{lll}
          (d_A \otimes d_B);iso;(m_{A,B} \otimes m_{A,B})
          & = & (\Delta_U \times \Delta_V,F_\otimes);(\hat{iso},F_{iso});(\id_{(U \times V) \times (U \times V)},F_\otimes)\\
          & = & ((\Delta_U \times \Delta_V);\hat{iso};\id_{(U \times V) \times (U \times V)},F_\otimes;F_{iso};F_\otimes)\\
          & = & ((\Delta_U \times \Delta_V);\hat{iso};,F_\otimes;F_{iso};F_\otimes)\\
          & = & (\Delta_{U \times V},\Theta;F_2)\\
          & = & (\id_{U \times V};\Delta_{U \times V},\Theta;F_2)\\
          & = & (\id_{U \times V},F_2);(\Delta_{U \times V},\Theta)\\
          & = & m_{A,B};d_{A,B}\\
        \end{array}
      \end{math}
    \end{center}
    At this point it suffices to show that
    $F_\otimes;F_{iso};F_\otimes = \Theta;F_2$, but this follows using
    similar reasoning as above, because $F_{iso};F_\otimes$ will
    reorganize the streams obtained by applying $g_1$ and $g_2$, and
    then the final $F_\otimes$ combines these sequences using $\Theta$.
    However, $F_2$ does the same reorganization, and then the streams
    are combined using $\Theta$.
    %% \begin{center}
    %%   \begin{math}
    %%     \begin{array}{lll}
    %%       F_\otimes(F_{iso}(F_\otimes(g_1,g_2)))
    %%       & = & F_\otimes(F_{iso}(\lambda v.F_2(g_1(v)),\lambda u.F_2(g_2(u)))\\
    %%       & = & (\lambda v.\Theta(F_2(g_1(\Delta_V(v)))),\lambda u.\Theta(F_2(g_2(\Delta_U(u)))))\\
    %%     \end{array}
    %%   \end{math}
    %% \end{center}

    %% \begin{center}
    %%   \begin{math}
    %%     \begin{array}{lll}
    %%       F_2(\Theta(g_1,g_2))
    %%       & = & F_2(\lambda u.g_1(u,u) \circ g_2(u,u))\\
    %%     \end{array}
    %%   \end{math}
    %% \end{center}
  \end{itemize}
\end{report}
\end{proof}

\begin{corollary}
  \label{corollary:dial-FLNL}
  The category $\dial$ is a full LNL model.
\end{corollary}
\begin{proof}
  This follows directly from the previous lemma and
  Theorem~\ref{thm:lnl_models_are_linear_categories} which shows that
  linear categories are LNL models.
\end{proof}

\textbf{Remark:} It would appear, from the fact that tensorial logic
\cite{Mellies:2008} is a relaxing of linear logic where instead of an
involutive negation we have a natural transformation $\eta_A\colon A
\to \neg\neg A$ that $\dial$ would be a model of tensorial
logic. After all in $\dial$ we do have a natural transformation of the
shape described, taking an object $(U,X, \alpha)$ to $(X, U, \neg
\alpha)$ and then to $(U, X, \neg\neg \alpha)$ which is "almost" an
isomorphism: we use identities in U and X, but unless the predicate
$\alpha$ itself is double-negated, we have a morphism $\alpha \to
\neg\neg \alpha$, but not a converse one. But we have not had the time
to check whether the other structure of tensorial logic is present and
hence we leave this to future work.
% section full_linear_categories (end)

\section{Conclusion and Future Work}
\label{sec:conclusion_and_future_work}

We first gave the definition of full intuitionistic linear logic using
the left rule for par proposed by Bellin in
Section~\ref{sec:full_intuitionistic_linear_logic_(fill)}.  We then
proved cut-elimination of FILL in Section~\ref{sec:cut-elimination} by
adapting the well-known cut-elimination procedure for classical linear
logic to FILL.  Finally, in Section~\ref{sec:full_lnl_models} we
showed that the category $\dial$, a model of FILL, is a full LNL model
by showing that it is a full linear category, and then replaying the
proof that linear categories are LNL models by Benon.

\textbf{Future work.} Lorenzen games are a particular type of game
semantics for various logics developed by Lorenz, Felscher, and Rahman
et al. \cite{Keiff:2011,Rahman:2005}. %% for an introduction.
%% Lorenzen games consist of a first-order language consisting of the
%% logical connectives of the logic one wishes to study.  Then the
%% structure of the games are defined by two types of rules: particle
%% rules and structural rules.  The particle rules describe how formulas
%% can be attacked or defended based on the formulas main connective.
%% Then the structural rules orchestrate the particle rules as the game
%% progresses.  They describe the overall organization of the game.

Rahman showed that Lorenzen games could be defined for classical
linear logic \cite{Rahman:2002}.  He was able to define a sound and
complete semantics in Lorenzen games for classical linear logic, but
he does mention that one could adopt a particular structural rule that
enforces intuitionism.  We plan to show that by adopting this rule we
actually obtain a sound and complete semantics in Lorenzen games for
FILL.
% section conclusion_and_future_work (end)


\bibliographystyle{plain}
\bibliography{ref}

%% \appendix

%% \begin{report}
%%   \section{The full specification of FILL}
%% \label{sec:fill_specification}
%% \FILLall{}
%% % section the_full_fill_specification (end)
%% \end{report}

%%% Local Variables: 
%%% mode: latex
%%% TeX-master: t
%%% End: 
