\documentclass{article}
\usepackage[utf8]{inputenc}
\usepackage{amssymb,amsmath,amsthm}
\usepackage{cmll}
\usepackage{stmaryrd}
\usepackage{mathpartir}
\usepackage{hyperref}

\title{SuperPower Lorenzen Games}
\author{Valeria de Paiva and Harley Eades III}
\date{May 2014}

% Theorems
\newtheorem{theorem}{Theorem}
\newtheorem{lemma}[theorem]{Lemma}
\newtheorem{fact}[theorem]{Fact}
\newtheorem{corollary}[theorem]{Corollary}
\newtheorem{definition}[theorem]{Definition}
\newtheorem{remark}[theorem]{Remark}
\newtheorem{proposition}[theorem]{Proposition}
\newtheorem{notn}[theorem]{Notation}
\newtheorem{observation}[theorem]{Observation}

% Commands that are useful for writing about type theory and programming language design.
%% \newcommand{\case}[4]{\text{case}\ #1\ \text{of}\ #2\text{.}#3\text{,}#2\text{.}#4}
\newcommand{\interp}[1]{\llbracket #1 \rrbracket}
\newcommand{\normto}[0]{\rightsquigarrow^{!}}
\newcommand{\join}[0]{\downarrow}
\newcommand{\redto}[0]{\rightsquigarrow}
\newcommand{\nat}[0]{\mathbb{N}}
\newcommand{\fun}[2]{\lambda #1.#2}
\newcommand{\CRI}[0]{\text{CR-Norm}}
\newcommand{\CRII}[0]{\text{CR-Pres}}
\newcommand{\CRIII}[0]{\text{CR-Prog}}
\newcommand{\subexp}[0]{\sqsubseteq}
%% Must include \usepackage{mathrsfs} for this to work.
\newcommand{\powerset}[0]{\mathscr{P}}


%\usepackage{natbib}
%\usepackage{graphicx}

% Ott includes.
\usepackage{color}
\usepackage{supertabular}
\input{FILL-ott}

\begin{document}

\maketitle

\begin{abstract}
  This note discusses game semantics, in the style of Lorenzen, for
  Full Intuitionistic Linear Logic, following Blass and Rahman.
\end{abstract}

%\begin{keyword}
%% keywords here, in the form: keyword \sep keyword

%% MSC codes here, in the form: \MSC code \sep code
%% or \MSC[2008] code \sep code (2000 is the default)

%\end{keyword}

%%
%% Start line numbering here if you want
%%
% \linenumbers

%% main text

%\section*{Introduction}
\label{}
%% The Appendices part is started with the command \appendix;
%% appendix sections are then done as normal sections
%% \appendix
\section*{Blass  on Linear Logic Games}
Blass is responsible, to a large extent, for the introduction of
game-theoretic ideas related to Linear Logic\cite{Blass:1992}. The
project was enthusiastically embraced by Hyland, Abramsky and many
others and there are now several conferences dedicated to the theme,
especially to the application of game-theoretical ideas to programming
language semantics.
 
The essence of the idea of a game theoretic semantics of a logic is
that a game (or an argument, dialogue or protocol) consists of two
players, one (the Proponent or Player) seeking to establish the truth
of a formula under consideration (trying to prove it) while the other
(the Opponent) disputing it, trying to prove it false. The two players
alternate, attacking and defending their positions. As Blass puts it,
the heart of the semantics consists of rules of the debate between the
players.
 
The idea of using these dialogues goes back to, at least, Lorenzen in
the late 50s.  Lorenzen wanted to show that his dialogues were an
alternative semantics for {\em Intuitionistic} Logic, but somehow the
games looked much more like Classical Logic. Felscher came up with
conditions that guaranteed that only constructive formulas were
provable, but the conditions are not very transparent. Similarly Blass
first introduced games for Linear Logic, but the composition of
strategies in his games was not associative, a problem fixed by
Abramsky and Jaghadeesan. But their semantics has problems of its own,
as described in \cite{Blass:1997}.
 
We recall that Blass presented two slightly different semantics for
Linear Logic using games in\cite{Blass:1994,Blass:1995}. The semantics
in \cite{Blass:1992} is for affine logic, that is the rule of
weakening is validated. Both semantics say that to play the game $A\&B
$, the opponent has to choose one of $A$ or $B$ and then the chosen
game is played. And to play the game $A\oplus B$ the proponent chooses
one of $A$ or $B$ and then the chosen game is played.  A play of
$A\otimes B$ consists of interleaved runs of the two games $A$ and
$B$, whenever it is the proponent's turn to play in either or both
constituents, he must play there. When it's the opponent's turn to
move in both components, he must choose one and move in it. Blass also
provided an interpretation of the modality ``of course" , denoted by
``!", similar to a repeated tensor product. Since linear negation is
interpreted simply as exchange of players, ``par", the multiplicative
disjunction, and linear implication were {\em not} described, as in
Classical Linear Logic they can be defined in terms of tensor and
linear negation.
  
 \section*{Lorenzen Games}
 Lorenzen games were developed by Lorenz, Felscher and Rahman, who
 together with co-workers, established a collection of games for
 specific (non-classical) logics, including Linear Logic. This general
 framework was named Dialogic. We recap the basics of Lorenzen games,
 as expounded by the Rahman school (\cite{Rahman:2005}).

 The language $L$ is composed of the standard components of
 first-order logic, with four connectives $\land, \lor, \to, \neg$,
 with small letters $(a, b, c,\ldots)$ for prime formulas and capital
 italic letters $(A, B, C, \ldots)$ for complex formulas. We need two
 labels, $\bf O$ and $\bf P$ (standing for the players, Opponent and
 Proponent, respectively).  We also need some special force symbols:
 ?... and !... , where ? stands for attack (or question) and ! stands
 for defense.  The set of rules in dialogic is divided into {\em
   particle} rules and {\em structural} rules. Particle rules describe
 the way a formula can be attacked and defended, according to its main
 connective. Structural rules, on the other hand, specify the general
 organization of the game. We restrict ourselves to propositional
 logic.  In the following we describe three systems, intuitionistic
 logic, classical logic and classical linear logic, before we embark
 on the novelty of this note, {\em full intuitionistic} linear logic.

\section*{Games for Intuitionistic and Classical Logic}
The table with particle rules for intuitionistic and classical logic
is as follows:

\begin{center}
  \begin{tabular}{| l | c || r | }
    \hline
  Assertion & Attack  & Response \\ \hline
  $ A\land B$ & $\land_{L}$ & $A$ \\ \hline
   $ A\land B$ & $\land_{R}$ & $B$ \\ \hline
    $ A\lor B$ & $?$ & $A$ or $B$ \\ \hline
    $ A\to B$ & $A$ & $B$ \\ \hline
     $ \neg A$ & $A$ & $-$ \\ \hline
    \hline
  \end{tabular}
 % \caption{Table 1: Particle rules for dialogue games}
\end{center}
Disjunction and conjunction differ only by the player who has the
choice of the immediate subformula with which the game will proceed:
in a conjunction, the challenger may choose, confident that either
conjunct can be refuted; in a disjunction the choice lies with the
defender. Thus, to defend a conjunction, a player must be able to
defend any of the conjuncts, while in the case of a disjunction, it is
sufficient to be able to defend one of the disjuncts.  Most
importantly to challenge an implication essentially amounts to
providing a proof of the antecedent and claiming that the other player
will fail to build a proof of the consequent from it. The defence
against such an attack then consists of a proof of the consequent. And
for negation, the only way to attack the assertion $\neg A$ is to
assert $A$, and be prepared to defend this assertion. Thus there is no
proper defence against such an attack, but it may be possible to
counterattack the assertion of $A$. Lorenzen games were born in the
context of constructivism, so this is not a surprise, the constructive
negation of $A$ is $A\to \bot$.

The difference between classical and intuitionistic logic resides on
the structural rules.  For classical logic the rules are:
\begin{itemize}
\item[S1] $\bf P$ may assert an atomic formula only after it has been
  asserted by $\bf O$.  \iffalse
\item[S2] If p is an $\bf P$-position, and if at round $n - 1$ there
  are several open attacks made by $\bf O$, then only the latest of
  them may be answered at $n$ (and the same with $\bf P$ and $\bf O$
  reversed).
\item[S3] An attack may be answered at most once.
\fi
\item[S4] A $\bf P$-assertion may be attacked at most once.
\item[S5] $\bf O$ can react only upon the immediately preceding $\bf
  P$-statement.
\end{itemize}
while for intuitionistic logic the structural rules are the ones above
plus:
 \begin{itemize}
   % \item[D1] $\bf P$ may assert an atomic formula only after it has
   %   been asserted by $\bf O$ before.
 \item[S2] If p is an {\bf P}-position, and if at round $n - 1$ there
   are several open attacks made by {\bf O}, then only the latest of
   them may be answered at $n$ (and the same with {\bf P} and {\bf O}
   reversed).
 \item[S3] An attack may be answered at most once.
%\item[D4] A P-assertion may be attacked at most once.
%\item[E] O can react only upon the immediately preceding P-statement.
\end{itemize}


A {\em dialogue} can be thought of as a set of dialogical games,
structured as a tree, the root of which is the thesis of the
dialogue. Splits in the dialogue tree are generated by the
propositional choices of the Opponent. Any possible attack by the
Opponent against a conjunction, any possible defence of a disjunction,
and either possible reaction in the case of an attack by the Proponent
against an implication he defends (counterattack or defence) will
generate a new branch in the dialogue tree. No move of the Proponent
will open a new branch. A completed dialogue tree will thus contain
all the Opponent's possible choices.

Starting with the thesis produced by the Proponent, the dialogue
unravels connectives, until we get to atomic propositions. A
dialogical game is said to be closed iff there is some atomic formula
which has been played by both players. A dialogical game is finished
iff it is closed or the rules do not allow any further move by the
player who has to move.

For a set $S$ of dialogue rules governing how the game is to proceed,
a formula $A$ is $S$-valid iff Proponent $\bf P$ has a {\em winning
  strategy} for $A$.  A strategy for a player in a dialogical game is
a function telling him what to do, according to what has previously
happened in the game, to win it.

The problem with the structural rules is that it is not clear which
modifications can be made to them without changing the set of provable
formulas.

\section*{Lorenzen Games for Linear Logic}

We now recall Rahman's\cite{Rahman:2002} definitions of Lorenzen games
for Linear logic, see \cite{Keiff:2011}. In Linear Logic each
occurrence of one formula in a proof must be taken as a distinct
resource for the inference process. In the dialogical framework, each
move is an action. It is therefore natural to consider that two
distinct moves are different actions, consuming different resources,
even when the two moves have the same propositional content. But we
have to introduce some book-keeping to deal with usage of resources.

To keep linearity and show validity of a formula we must use all and
each formula that has been asserted throughout the dialogue. And we
cannot use one round more than once.
 
To adapt the Lorenzen games to Linear Logic, we have to provide a new
table of particle rules (for the new linear connectives) and we have
to adopt some new principles, like:
\begin{enumerate}
\item An atomic $\bf O$-formula has been used iff this formula is the
  propositional content of a $\bf P$-move. Atomic $\bf P$-formulas are
  used by the very move in which they appear.
\item A complex formula $A$ has been used iff all the possible
  aggressive and defensive moves related to $A$ have been stated.
\end{enumerate}

Linear dialogues are contextual, the flow of information within the
proof is constrained by an explicit structure of {\em contexts},
ordered by a relation of subordination. The introduction of a new
context (which corresponds to a splitting of contexts in the sequent
calculus) will always be a consequence of the Opponent's choices. The
Proponent will stay in the same context as long as he can.

A sequent $\Gamma \vdash A$ is the statement that from assumptions
$\Gamma$ one can infer conclusion $A$. From the dialogical point of
view, assumptions are the Opponent's conc1essions, while conclusions
are the Proponent's claims. Splitting contexts for tensor occurs when
it is asserted by the Proponent, so the dialogical particle rule will
let the challenger choose the context where the dialogue will
proceed. Dually, the multiplicative disjunction par will generate a
splitting of contexts when asserted by the Opponent, thus the particle
rule will let the defender choose the context. The interpretation of
linear implication $\multimap$ remains constant, to attack it, we one
must assert the antecedent, hoping that the opponent cannot use it to
prove the consequent. Also the interpretation of linear negation stays
the same.

\begin{center}
  \begin{tabular}{| l | c || r |}
    \hline
  Assertion & Attack  & Response \\ \hline
  $ A\otimes B$& $\otimes_{L}$& $A$   \\ 
   in context $\mu$ & in subcontext $\nu$ of $\mu$ &in $\nu$ \\  \hline
   $A\otimes B$  & $\otimes_{R}$ & $B$ \\ 
      in context $\mu$ & in subcontext $\nu'$ of $\mu$ &in $\nu'$ \\  \hline
    $ A\parr B$ & $?$ & $A$  \\ 
       in context $\mu$ &in $\mu$  &in subcontext $\nu$ of $\mu$ \\  \hline
     $ A\parr B$ & $?$ & $B$ \\ 
           in context $\mu$ &in $\mu$  &in subcontext $\nu'$ of $\mu$ \\\hline
    $ A\multimap B$ & $A$ & $B$ \\ 
          in context $\mu$ & in subcontext $\nu$ of $\mu$ & in subcontext $\nu$ of $\mu$\\ \hline
     $ \neg A$ & $A$ & $-$ \\ 
       in context $\mu$ & in subcontext $\nu$ of $\mu$ & in subcontext $\nu$ of $\mu$\\ \hline
    $ A\with B$& $?\with_{L}$& $A$   \\   
      $ A\with B$& $?\with_{R}$& $B$   \\ \hline
        $ A\oplus B$& $?$& $A$   \\   
          $ A\oplus B$& $?$& $B$   \\    
    \hline
  \end{tabular}
 % \caption{Table 1: Particle rules for dialogue games}
\end{center}

% More about linear contexts.Only $\bf O$ can open a new context. $\bf
% P$ can only play on contexts opened by $\bf O$. if during a dialogue
% $\bf O$ opens a new context, we say that the new context is a
% sub-context of the initial one.
 
Rahman also introduces an interpretation for the linear logic
exponentials in terms of games, but says that the proof that the games
are sound and complete for the exponentials is future work.
 
It is worth noticing that while Rahman deals exclusively with
classical linear logic, he does mention that the intuitionistic
structural rule could be used instead. Using the intuitionistic
structural rule we should arrive at a semantics for Full
Intuitionistic Linear Logic.
 
\section*{Full Intuitionistic Linear Logic}

Full Intuitionistic Linear Logic (FILL) is a variant of Linear Logic
introduced by de Paiva and Hyland in \cite{Hyland:1993}. FILL is an
interesting system, as it has independent multiplicative and additive
connectives, just like Intuitionistic Logic has three independent
connectives, conjunction, disjunction and implication.

FILL came into existence from work on Dialectica categories, which can
be considered a different kind of game, a superpower game where
players, instead of playing a proper game, simply face-off their
collections of strategies. In one round the superpower game is
decided, either Proponent has a winning strategy or Opponent has one
or neither has one.

There are two main formalizations of FILL that enjoy cut-elimination.
In the next two sections we give detailed definitions of both of these
formalizations.

\subsection*{FILL Using Dependency Relations}
\label{subsec:fill_using_dependency_relations}
One of the major obstacles of defining a formalization of FILL is
obtaining cut-elimination.  An initial formalization of FILL
constrained the implication right rule to at most one formula on the
right, but it was pointed out by Schellinx that this breaks
cut-elimination \cite{Schellinx:1991}.  A second formalization of FILL
used a term-assignment to constrain Classical Linear Logic to FILL
\cite{Hyland:1993}, but this formalization did not enjoy
cut-elimination either as pointed out Biereman \cite{Bierman:1996a}.
Later, Bra\"ner and de Paiva gave an alternate formalization of FILL
based on dependency-relations that starts with Classical Linear Logic,
and then constrains derivations using a dependency tracking system
based on a dependency relation \cite{Brauner:1998}.  Then a FILL
derivation is defined as a Classical Linear Logic derivation where
only a single conclusion can depend on the hypothesis begin
discharged.

\begin{definition}
  \label{def:syntax-dep-rel}
  The syntax of formulas and contexts of Classical Linear Logic are
  defined as follows:
  \begin{center}
    \begin{math}
      \begin{array}{lll}
        \text{(Formulas)} & A,B,C ::= [[A (x) B]] \mid [[I]] \mid [[A
        (+) B]] \mid [[_|_]] \mid [[A -o B]]\\
        \text{(Contexts)} & [[G]],[[L]] ::= [[.]] \mid A \mid [[G1,G2]]\\
      \end{array}
    \end{math}
  \end{center}
\end{definition}
We will consider contexts to be lists of formulas where $[[.]]$ is the
empty list, and $[[G1,G2]]$ denotes appending the list $[[G1]]$ with
the list $[[G2]]$.

\begin{definition}
  \label{def:infr-dep-rel}
  The inference rules for Classical Linear Logic are defined as
  follows:
  \begin{center}
    \begin{mathpar}
      \FILLdruleDax{} \and
      \FILLdruleDcut{} \and
      \FILLdruleDtl{} \and
      \FILLdruleDtr{} \and
      \FILLdruleDil{} \and
      \FILLdruleDir{} \and
      \FILLdruleDparl{} \and
      \FILLdruleDparr{} \and
      \FILLdruleDpl{} \and
      \FILLdruleDpr{} \and
      \FILLdruleDimpl{} \and
      \FILLdruleDimpr{}
    \end{mathpar}
  \end{center}
\end{definition}
Now we must define when a classical derivation is a FILL derivation,
but this requires the notion of a dependency relation.  To define the
latter we will need a generalization of the usual notion of
composition of relations.

\newcommand{\grc}[0]{\mathop{\star}}

\begin{definition}
  \label{def:gen-rel-comp}
  Suppose $A_1$, $A_2$, $B_1$, and $B_2$ are sets, and $r_1 \subseteq
  A_1 \times B_1$ and $r_2 \subseteq A_2 \times B_2$ are relations.
  Then we define the relation $r_1 \grc r_2 \subseteq (A_1
  \cup (A_2 - B_1)) \times (B_2 \cup (B_1 - A_2))$ as 
  \begin{center}
    \begin{math}
      \begin{array}{lll}
        \lbrack (r_1 \cap (A_1 \times (B_1 \cap A_2))) * (r_2 \cap
        ((A_2 \cap B_1) \times B_2))\rbrack \cup \\
        \lbrack r_1 \cap (A_1 \times (B_1 - A_2))\rbrack \cup \\
        \lbrack r_2 \cap ((A_2 - B_1) \times B_2) \rbrack
      \end{array}
    \end{math}
  \end{center}
  where $*$ is the usual composition of relations.
\end{definition}
The previous definition is due to Bra\"uner and de Paiva in
\cite{Brauner:1998}.  In that paper they give further explanations of
the definition that the reader might find helpful.  At this point we
define a relation that calculates the set of dependencies of a
classical derivation.  
\begin{definition}
  \label{def:dep-dep-rel}
  Suppose $\tau$ is a classical derivation whose end sequent is $[[G
  |- L]]$.  Now we define the relation $Dep(\tau) \subseteq [[G]]
  \times [[L]]$ where $[[G]]$ and $[[L]]$ are considered as sets of
  formula occurrences by induction on the structure of $\tau$:
  \begin{itemize}
  \item[Case.] If $\tau$ consists solely of an application of either of the rules
    $\FILLdrulename{Dir}$ or $\FILLdrulename{Dpl}$, then $Dep(\tau) =
    \emptyset$.  

  \item[Case.] If $\tau$ consists of only an application of the axiom
    rule where $[[G |- L]] \equiv [[A |- A]]$, then $Dep(\tau) =
    \{(A,A)\}$.

    
  \item[Case.] If the last rule applied in $\tau$ was
    $\FILLdrulename{Dcut}$ whose two immediate subderivations are
    $\tau_1$ and $\tau_2$ with final sequents $[[G1 |- B', L1]]$ and
    $[[G2, B'' |- L2]]$, then $Dep(\tau) = Dep(\tau_1) \grc \{B'\}
    \grc \{B''\} \grc Dep(\tau_2)$.

  \item[Case.] If the last rule applied in $\tau$ was
    $\FILLdrulename{Dtl}$ whose immediate subderivation is
    $\tau_1$ with final sequent $[[G1,A,B |- L1]]$, 
    then 
    $Dep(\tau) = \{[[A (x) B]]\} \grc \{A,B\} \grc Dep(\tau_1)$.

  \item[Case.] If the last rule applied in $\tau$ was
    $\FILLdrulename{Dtr}$ whose two immediate subderivations are
    $\tau_1$ and $\tau_2$ with final sequents $[[G1 |- A,L1]]$ and
    $[[G2 |- B,L2]]$ respectively, then 
    $Dep(\tau) = (Dep(\tau_1) \cup Dep(\tau_2)) \grc \{A,B\} \times
    \{[[A (x) B]]\}$.

  \item[Case.] If the last rule applied in $\tau$ was
    $\FILLdrulename{Dil}$ whose immediate subderivation is
    $\tau_1$ with final sequent $[[G1 |- L1]]$, and $[[G |- L]] \equiv
    [[G1,I |- L1]]$, 
    then 
    $Dep(\tau) = \{[[I]]\} \grc \emptyset \grc Dep(\tau_1)$.

  \item[Case.] If the last rule applied in $\tau$ was
    $\FILLdrulename{Dparl}$ whose two immediate subderivations are
    $\tau_1$ and $\tau_2$ with final sequents $[[G1, A |- L1]]$ and
    $[[G3, B |- L2]]$ respectively, then 
    $Dep(\tau) = \{[[A (+) B]]\} \times \{A,B\} \grc (Dep(\tau_1) \cup
    Dep(\tau_2))$.

  \item[Case.] If the last rule applied in $\tau$ was
    $\FILLdrulename{Dparr}$ whose immediate subderivation is
    $\tau_1$ with final sequent $[[G1 |- A,B,L1]]$, 
    then 
    $Dep(\tau) = Dep(\tau_1) \grc \{A,B\} \times \{[[A (+) B]]\}$.

  \item[Case.] If the last rule applied in $\tau$ was
    $\FILLdrulename{Dpr}$ whose immediate subderivation is
    $\tau_1$ with final sequent $[[G1 |- A,B,L1]]$, and $[[G |- L]]
    \equiv [[G1 |- _|_, L1]]$, 
    then 
    $Dep(\tau) = Dep(\tau_1) \grc \emptyset \times \{[[_|_]]\}$.

  \item[Case.] If the last rule applied in $\tau$ was
    $\FILLdrulename{Dimpl}$ whose two immediate subderivations are
    $\tau_1$ and $\tau_2$ with final sequents $[[G1 |- A, L1]]$ and
    $[[G2, B |- L2]]$ respectively, then 
    $Dep(\tau) = Dep(\tau_1) \grc \{A,[[A -o B]]\} \times \{B\} \grc
    Dep(\tau_2)$.

  \item[Case.] If the last rule applied in $\tau$ was
    $\FILLdrulename{Dimpr}$ whose immediate subderivation is
    $\tau_1$ with final sequent $[[G1, A |- B, L1]]$,
    then 
    $Dep(\tau) = \emptyset \times \{A\} \grc Dep(\tau_1) \grc \{B\}
    \times \{[[A -o B]]\}$.
  \end{itemize}   
  If $\tau$ is a derivation of $[[G |- L]]$ where $[[A]]$ and $[[B]]$
  are formulas in $[[G]]$ and $[[L]]$ respectively, then we say that
  $[[B]]$ depends on $[[A]]$ if and only if $(A,B) \in Dep(\tau)$.
\end{definition}
Finally, we arrive at this sections main definition.
\begin{definition}
  \label{def:dep-rel-FILL-deriv}
  A FILL derivation is a classical derivation where whenever the
  implication right rule ($\FILLdrulename{Dimpr}$) is applied to a
  derivation $\tau$ of the sequent $[[G,A |- B,L]]$ to obtain $[[G |-
  A -o B,L]]$ none of the formulas in $[[L]]$ depend on $[[A]]$ in
  $\tau$.  That is, for all $[[C]] \in [[L]]$, $(A,C) \not\in
  Dep(\tau)$.
\end{definition}
For a more detailed explanation of this logic see \cite{Brauner:1998}.
We will freely use the cut-elimination theorem for FILL whose proof
can also be found in \cite{Brauner:1998}.
% subsection fill_using_dependency_relations (end)

\subsection*{FILL Using a Term Assignment}
\label{subsec:fill_using_a_term_assignment}
TODO
% subsection fill_using_a_term_assignment (end)

\section*{Lorenzen Games for FILL}
\label{sec:lorenzen_games_for_fill}

% section lorenzen_games_for_fill (end)

 \section*{Conclusions}
 We recalled games for Linear Logic in two traditions, Blass-style
 games (as discussed by Hyland, Abramsky and Jagadhesan and many
 others) and Lorenzen-style games (as introduced by Rahman, Keiff and
 others). It seems clear that Lorenzen games have the ability to model
 full intuitionistic linear logic easily. Emphasis is given to the
 interpretation of linear implication, instead of negation.
 
 This note is just a first step in the programme to develop {\em
   compositional} versions of Lorenzen Linear logic games. We need
 still to prove our soundness and completeness result and more
 importantly we need to make sure that strategies are compositional,
 to make sure we can produce categories of Lorenzen games.
%% \label{}

%% References
%%
%% Following citation commands can be used in the body text:
%% Usage of \cite is as follows:
%%   \cite{key}         ==>>  [#]
%%   \cite[chap. 2]{key} ==>> [#, chap. 2]
%%

%% References with bibTeX database:

\bibliographystyle{plain}
\bibliography{ref}

%% Authors are advised to submit their bibtex database files. They are
%% requested to list a bibtex style file in the manuscript if they do
%% not want to use elsarticle-num.bst.

%% References without bibTeX database:

% \begin{thebibliography}{000}

% %% \bibitem must have the following form:
% %%   \bibitem{key}...
% %%

% \bibitem{blass92} Blass, A. {\it A Game Semantics for Linear Logic}, Annals of Pure and 
% Applied Logic 56, 183-220, 1992. 

% \bibitem{blass94} {\it Is game semantics necessary?} In B\"orger, Gurevich and Meinke, eds. Computer Science Logic, 7th Workshop, 1993, springer-Verlag, 1994.

% \bibitem{blasssll} Blass A. {\it Some Semantical Aspects of Linear Logic}, Journal of the 
% Interest Group in Pure and Applied Logic 5, 487-503, 1997.

% \bibitem{blassqa}  Blass A. {\it Questions and Answers -- a category arising in Linear Logic, Complexity Theory and Set Theory} In Girard, Lafont and Regnier,  eds. ``Advances in Linear Logic", vol. 222 of London Mathematical Society Lecture Notes, Cambridge University Press, 1995.

% \bibitem{rahmankeiff04} Rahman, S. and Keiff, L. {\it On how to be a dialogician}. In D. Vanderveken (ed.): Logic, Thought and Action, Dordrecht: Springer, pp. 359-408, 2004.

% \bibitem{keiff09} Keiff, L. {\it Dialogical Logic}. In Stanford Encyclopedia of Philosophy, 2009.

% \bibitem{rahman02} Rahman, S. {\it Un desafio para las  teorias cognitivas de la competencia logica: los fundamentos pragmaticos de la semantica de la logica linear}, Manuscrito, volume XXV(2), pp. 381-432, October 2002. 

% \bibitem{hyland-depaiva93} Hyland, J. M. E. and de Paiva, V. C.V. {\it Full Intuitionistic Linear Logic} (extended abstract). Annals of Pure and Applied Logic, 64(3), pp.273-291, 1993.

% \bibitem{felscher85}Felscher, W. {\it Dialogues as a Foundation for Intuitionistic 
% Logic}. In: Gabbay, D. and  Guenthner, F. (eds.), Handbook of 
% Philosophical Logic, Bd. III, Dordrecht, 1985, pp. 341-372. 

% \end{thebibliography}

\appendix

\section{The full specification of FILL}
\label{sec:fill_specification}
\FILLall{}
% section the_full_fill_specification (end)


\end{document}

%%% Local Variables: ***
%%% mode: latex ***
%%% TeX-master: "FILLDialogical.tex" ***
%%% End: ***