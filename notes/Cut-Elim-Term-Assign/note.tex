\documentclass{article}
\usepackage[utf8]{inputenc}
\usepackage{amssymb,amsmath,amsthm}
\usepackage{cmll}
\usepackage{stmaryrd}
\usepackage{todonotes}
\usepackage{mathpartir}
\usepackage{fullpage}

% Theorems
\newtheorem{theorem}{Theorem}
\newtheorem{lemma}[theorem]{Lemma}
\newtheorem{fact}[theorem]{Fact}
\newtheorem{corollary}[theorem]{Corollary}
\newtheorem{definition}[theorem]{Definition}
\newtheorem{remark}[theorem]{Remark}
\newtheorem{proposition}[theorem]{Proposition}
\newtheorem{notn}[theorem]{Notation}
\newtheorem{observation}[theorem]{Observation}

% Commands that are useful for writing about type theory and programming language design.
%% \newcommand{\case}[4]{\text{case}\ #1\ \text{of}\ #2\text{.}#3\text{,}#2\text{.}#4}
\newcommand{\interp}[1]{\llbracket #1 \rrbracket}
\newcommand{\normto}[0]{\rightsquigarrow^{!}}
\newcommand{\join}[0]{\downarrow}
\newcommand{\redto}[0]{\rightsquigarrow}
\newcommand{\nat}[0]{\mathbb{N}}
\newcommand{\fun}[2]{\lambda #1.#2}
\newcommand{\CRI}[0]{\text{CR-Norm}}
\newcommand{\CRII}[0]{\text{CR-Pres}}
\newcommand{\CRIII}[0]{\text{CR-Prog}}
\newcommand{\subexp}[0]{\sqsubseteq}
%% Must include \usepackage{mathrsfs} for this to work.
\newcommand{\powerset}[0]{\mathscr{P}}


\title{Cut-elimination of the APAL term assignment formulation of FILL}
\author{Harley Eades III}
\date{September 2014}

%\usepackage{natbib}
%\usepackage{graphicx}

% Ott includes.
\usepackage{color}
\usepackage{supertabular}
\input{FILL-ott}

\begin{document}

\maketitle

In this short note I give the proof of the term assignment formulation
of FILL first given in \cite{Hyland:1993}.

\section{The Fix}
\label{sec:the_fix}
Consider the $\FILLdrulename{Dparl}$ rule in the dependency-relation
formalization:
\begin{center}
  \begin{math}
    \begin{array}{lll}
      \FILLdruleDparl{}
      && Dep(\tau) = \{([[A (+) B]], [[A]]), ([[A (+) B]], [[B]])\}
      \star (Dep(\tau_1) \cup Dep(\tau_2))
    \end{array}
  \end{math}
\end{center}
If anything in $[[L1]]$ and $[[L2]]$ depend on $[[A]]$ or $[[B]]$ then
this will be accounted for in $Dep(\tau_1)$ and $Dep(\tau_2)$.  Thus,
in the term formalization when binding pattern variables across the
righthand side of the sequent we should do so if and only if there is
a dependency.  In fact, if a formula on the righthand side depends on
a formula in the lefthand side, then the variable associated with
that hypnosis must be free in the term associated with the formula on
the right.  This evidence suggests that to fix the term formalization
we must modify the $\FILLdrulename{Parl}$ rule.

The new $\FILLdrulename{Parl}$ rule as follows:
\[
\FILLdruleNParl{}
\]
The previous rule depends on a function which we define as follows:
\begin{center}
  \begin{math}
    \begin{array}{lll}      
      [[let-pat z (x (+) -) e]] = [[e]]\\
      \,\,\,\,\,\,\text{where } [[x]] \not\in \mathsf{FV}([[e]])\\
      & \\
      [[let-pat z (- (+) y) e]] = [[e]]\\
      \,\,\,\,\,\,\text{where } [[y]] \not\in \mathsf{FV}([[e]])\\
      & \\
      [[let-pat z p e]] = [[let z be p in e]]\\
    \end{array}
  \end{math}
\end{center}
Note that in the definition of $[[let-pat z p e]]$ the final case is a
catchall case.  Now the new $\FILLdrulename{Parl}$ rule only pattern
matches on $[[z]]$ in the righthand side if there is a dependency
between the variables in the pattern and the term in the pattern
match.  A similar rule to the above was proposed by Bellin in the
conclusion of \cite{Bierman:1996}.

This rule recovers from the counterexample.  The first derivation
given in the counter example above is unchanged, so we only give the
second:
\begin{center}
  \footnotesize
  \begin{math}    
    $$\mprset{flushleft}
    \inferrule* [right=\tiny Impr] {
      $$\mprset{flushleft}
      \inferrule* [right=\tiny Parr] {
        $$\mprset{flushleft}
        \inferrule* [right=\tiny Parl] {
          $$\mprset{flushleft}
          \inferrule* [right=\tiny Cut] {      
            $$\mprset{flushleft}
            \inferrule* [right=\tiny Pr] {
              $$\mprset{flushleft}
              \inferrule* [right=\tiny Ax] {
                \,
              }{[[v : A |- v : A]]}
            }{[[v : A |- v : A | o : _|_]]}
            \\
            $$\mprset{flushleft}
            \inferrule* [right=\tiny Tr] {
              $$\mprset{flushleft}
              \inferrule* [right=\tiny Ax] {
                \,
              }{[[x : A |- x : A]]}
              \\
              $$\mprset{flushleft}
              \inferrule* [right=\tiny Ax] {
                \,
              }{[[y : B |- y : B]]}
            }{[[x : A, y : B |- x (x) y : A (x) B]]}
          }{[[y:B, v : A |- v (x) y: A (x) B | o : _|_]]}          
          \\
          $$\mprset{flushleft}
          \inferrule* [right=\tiny Ax] {
            \,
          }{[[w : C |- w : C]]}
        }{[[v:A,cb{z}:B(+)C |- h(let cb{z} be y (+) - in h{v (x) y} : A (x) B | let cb{z} be - (+) w in w : C) | o : _|_]]}                   
      }{[[v:A,cb{z}:B(+)C |- {{let cb{z} be y (+) - in h{v (x) y}} (+) {let cb{z} be - (+) w in w}} : (A (x) B) (+) C | o : _|_]]}      
    }{[[v:A |- \z.{{let z be y (+) - in h{v (x) y}} (+) {let z be - (+) w in w}} : (B(+)C) -o ((A (x) B) (+) C) | o : _|_]]}
  \end{math}
\end{center}
This new derivation is now correct, and mirrors that of the
dependency-relation formalization.
% section the_fix (end)

\section{Basic Results}
\label{sec:basic_results}
\begin{lemma}[Substitution Distribution]
  \label{lemma:substitution_distribution}
  For any terms $[[t]]$, $[[t1]]$, and $[[t2]]$, $[[ [t1/x][t2/y]t]] = [[ [ [t1/x]t2/y][t2/x]t]]$.
\end{lemma}
\begin{proof}
  This proof holds by straightforward induction on the form of $t$.
\end{proof}

\begin{lemma}[Left and Right Exchange]
  \label{lemma:right_exchange}
  If $[[G,G' |- D,D']]$, then $[[G',G |- D',D]]$.
\end{lemma}
\begin{proof}
  This proof holds by straightforward induction on the assumed
  derivation.
\end{proof}

% section basic_results (end)


\section{Cut-elimination}
\label{sec:cut-elimination}

\subsection{Commuting conversion cut vs cut (first case)}
\label{subsec:commuting_conversion_cut_vs_cut_(first_case)}
The following proof
\begin{center}
  \begin{math}
    $$\mprset{flushleft}
    \inferrule* [right=Cut] {
      $$\mprset{flushleft}
      \inferrule* [right=] {
        \inferrule* [right=,vdots=1.5em,fraction=\,] {
          \,
        }{\pi_1}
      }{[[G |- t : A | L]]}
      \\
      $$\mprset{flushleft}
      \inferrule* [right=Cut] {
        $$\mprset{flushleft}
        \inferrule* [right=] {
          \inferrule* [right=,vdots=1.5em,fraction=\,] {
            \,
          }{\pi_2}
        }{[[G2,x : A,G3 |- t1 : B | L1]]}
        \\
        $$\mprset{flushleft}
        \inferrule* [right=] {
          \inferrule* [right=,vdots=1.5em,fraction=\,] {
            \,
          }{\pi_3}
        }{[[G1,y : B,G4 |- t2 : C | L2]]}
      }{[[G1,G2,x : A,G3,G4 |- h(L1 | [t1/y]t2 : C) | [t1/y]L2]]}
    }{[[G1,G2,G,G3,G4 |- h(h(L | [t/x]L1) | h{[t/x][t1/y]t2} : C) | [t/x]h([t1/y]L2)]]}
  \end{math}
\end{center}
is transformed into the proof
\begin{center}
  \begin{math}
    $$\mprset{flushleft}
    \inferrule* [right=Cut] {
      $$\mprset{flushleft}
      \inferrule* [right=] {
        $$\mprset{flushleft}
      \inferrule* [right=] {
        \inferrule* [right=,vdots=1.5em,fraction=\,] {
          \,
        }{\pi_1}               
      }{[[G |- t : A | L]]}
      \\
      $$\mprset{flushleft}
        \inferrule* [right=] {
          \inferrule* [right=,vdots=1.5em,fraction=\,] {
            \,
          }{\pi_2}
        }{[[G2,x : A,G3 |- t1 : B | L1]]}
      }{[[G2,G,G3 |- h{[t/x]t1} : B | [t/x]L1]]}
      \\
      $$\mprset{flushleft}
        \inferrule* [right=] {
          \inferrule* [right=,vdots=1.5em,fraction=\,] {
            \,
          }{\pi_3}
        }{[[G1,y : B,G4 |- t2 : C | L2]]}
    }{[[G1,G2,G,G3,G4 |- h(h(L | [t/x]L1) | {[ [t/x]t1/y]t2} : C) | [ [t/x]t1/y]L2]]}
  \end{math}
\end{center}
In order for the previous two proofs to be considered equal, we have
to show that the final terms in the conclusion of the above
derivations are equivalent. First, we know that the term $[[
[t/x][t1/y]t2]]$ in the first derivation above is equivalent to $[[ [
[t/x] t1/y][t/x] t2]]$ by Lemma~\ref{lemma:substitution_distribution}.
Furthermore, by inspecting the first derivation we can see that $[[x
nin FV(t2)]]$, and thus, $[[ [ [t/x] t1/y][t/x] t2 = [ [t/x] t1/y]
t2]]$.  This argument may be repeated for any term in $[[L2]]$, and
thus, we know $[[ [t/x][t1/y]L2 = [ [t/x]t1/y]L2]]$.
% subsection commuting_conversion_cut_vs_cut_(first_case) (end)

\subsection{Commuting conversion cut vs. cut (second case)}
\label{subsec:commuting_conversion_cut_vs._cut_(second_case)}
The second commuting conversion on cut begins with the proof
\begin{center}
  \begin{math}
    $$\mprset{flushleft}
    \inferrule* [right=Cut] {
      $$\mprset{flushleft}
      \inferrule* [right=] {
        \inferrule* [right=,vdots=1.5em,fraction=\,] {
          \,
        }{\pi_1}
      }{[[G |- t : A | L]]}
      \\
      $$\mprset{flushleft}
      \inferrule* [right=Cut] {
        $$\mprset{flushleft}
        \inferrule* [right=] {
          \inferrule* [right=,vdots=1.5em,fraction=\,] {
            \,
          }{\pi_2}
        }{[[G' |- t' : B | L']]}
        \\
        $$\mprset{flushleft}
        \inferrule* [right=] {
          \inferrule* [right=,vdots=1.5em,fraction=\,] {
            \,
          }{\pi_3}
        }{[[G1,x : A,G2,y:B,G3 |- t1 : C | L1]]}
      }{[[G1,x : A,G2,G',G3 |- h(L' | [t'/y]t1 : C) | [t'/y]L1]]}
    }{[[G1,G,G2,G',G3 |- h(h(L | [t/x]L') | h{[t/x][t'/y]t1} : C) | [t/x]h([t'/y]L1)]]}
  \end{math}
\end{center}
is transformed into the following proof:
\begin{center}
  \begin{math}
    $$\mprset{flushleft}
    \inferrule* [right=Cut] {
      $$\mprset{flushleft}
      \inferrule* [right=] {
        \inferrule* [right=,vdots=1.5em,fraction=\,] {
          \,
        }{\pi_2}
      }{[[G' |- t' : B | L']]}
      \\
      $$\mprset{flushleft}
      \inferrule* [right=Cut] {
        $$\mprset{flushleft}
        \inferrule* [right=] {
          \inferrule* [right=,vdots=1.5em,fraction=\,] {
            \,
          }{\pi_1}
        }{[[G |- t : A | L]]}      
        \\
        $$\mprset{flushleft}
        \inferrule* [right=] {
          \inferrule* [right=,vdots=1.5em,fraction=\,] {
            \,
          }{\pi_3}
        }{[[G1,x : A,G2,y:B,G3 |- t1 : C | L1]]}
      }{[[G1,G,G2,y:B,G3 |- h(L | [t/x]t1 : C) | [t/x]L1]]}
    }{[[G1,G,G2,G',G3 |- h(h(L' | [t'/y]L) | h{[t'/y][t/x]t1} : C) | [t'/y]h([t/x]L1)]]}
  \end{math}
\end{center}
Now, because we know $[[x,y nin FV(L)]]$ by inspection of the first
derivation, we know that $[[L = [t'/y]L]]$ and $[[L' = [t/x]L']]$.
Similarly, we know that $[[x,y nin FV(t)]]$ and $[[x,y nin FV(t')]]$.
Thus, by this fact and Lemma~\ref{lemma:substitution_distribution}, we
know that $[[ [t/x][t'/y]t1 = [ [t/x]t'/y][t/x]t1]] = [[ [t'/y][t/x]t1
]]$.  This argument can be repeated for any term in $[[L1]]$, hence,
$[[ [t/x][t'/y]L1 = [t'/y][t/x]L1]]$.  Finally, by
Lemma~\ref{lemma:right_exchange} we can exchange $[[L']]$ and $[[
[t'/y]L]]$ in the last proof to obtain the same order of terms on the
right in both derivations.  Therefore, both of the previous derivations
are equivalent.
% subsection commuting_conversion_cut_vs._cut_(second_case) (end)


% section cut-elimination (end)


\bibliographystyle{plain}
\bibliography{ref}

\appendix

\section{The full specification of FILL}
\label{sec:fill_specification}
\FILLall{}
% section the_full_fill_specification (end)


\end{document}

%%% Local Variables: 
%%% mode: latex
%%% TeX-master: t
%%% End: 