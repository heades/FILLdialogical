\documentclass{article}
\usepackage[utf8]{inputenc}
\usepackage{amssymb,amsmath,amsthm}
\usepackage{cmll}
\usepackage{stmaryrd}
\usepackage{todonotes}
\usepackage{mathpartir}
\usepackage{fullpage}

% Theorems
\newtheorem{theorem}{Theorem}
\newtheorem{lemma}[theorem]{Lemma}
\newtheorem{fact}[theorem]{Fact}
\newtheorem{corollary}[theorem]{Corollary}
\newtheorem{definition}[theorem]{Definition}
\newtheorem{remark}[theorem]{Remark}
\newtheorem{proposition}[theorem]{Proposition}
\newtheorem{notn}[theorem]{Notation}
\newtheorem{observation}[theorem]{Observation}

% Commands that are useful for writing about type theory and programming language design.
%% \newcommand{\case}[4]{\text{case}\ #1\ \text{of}\ #2\text{.}#3\text{,}#2\text{.}#4}
\newcommand{\interp}[1]{\llbracket #1 \rrbracket}
\newcommand{\normto}[0]{\rightsquigarrow^{!}}
\newcommand{\join}[0]{\downarrow}
\newcommand{\redto}[0]{\rightsquigarrow}
\newcommand{\nat}[0]{\mathbb{N}}
\newcommand{\fun}[2]{\lambda #1.#2}
\newcommand{\CRI}[0]{\text{CR-Norm}}
\newcommand{\CRII}[0]{\text{CR-Pres}}
\newcommand{\CRIII}[0]{\text{CR-Prog}}
\newcommand{\subexp}[0]{\sqsubseteq}
%% Must include \usepackage{mathrsfs} for this to work.
\newcommand{\powerset}[0]{\mathscr{P}}


\title{Cut-elimination of the term assignment formulation of FILL}
\author{Harley Eades III}
\date{September 2014}

%\usepackage{natbib}
%\usepackage{graphicx}

% Ott includes.
\usepackage{color}
\usepackage{supertabular}
\input{FILL-ott}

\begin{document}

\maketitle

In \cite{Hyland:1993} Martin Hyland and Veleria de Paiva give a term
formalization of Full Intuitionistic Linear Logic (FILL), but later
Bierman was able to give a counterexample to cut-elimination
\cite{Bierman:1996}.  As Bierman explains the problem was that the
left rule for par introduced a fresh variable into to many terms on
the right-side of the conclusion.  This resulted in a counterexample
where this fresh variable became bound in one term, but is left free
in another. This resulted from first doing a commuting conversion on
cut, and then $\lambda$-binding the fresh variable.  Thus,
cut-elmination failed.  In the conclusion of Bierman's paper he gives
an alternate left-par rule which he attributes to Bellin, and states
that this alternate rule should fix the problem with cut-elimination
\cite{Bierman:1996}.  In this note we adopt Bellin's rule, and then
show cut-elimination in Section~\ref{sec:cut-elimination}.

\section{Full Intuitionistic Linear Logic (FILL)}
\label{sec:full_intuitionistic_linear_logic_(fill)}

In this section we give a brief description of Full Intuitionistic
Linear Logic (FILL) in the style found in \cite{Hyland:1993}.
However, we use a slightly different presentation that we feel
provides a more elegant description of the logic.  We first give the
syntax of formulas, patterns, terms, and contexts.  Following the syntax we
define several meta-functions that will be used when defining the
inference rules of the logic.

\begin{definition}
  \label{def:syntax}
  The syntax for FILL is as follows:
  \begin{center}
    \begin{math}
      \begin{array}{cll}
        \text{(Formulas)}       & [[A]], [[B]], [[C]], [[D]], [[E]] ::= [[I]] \mid [[_|_]]
        \mid [[A -o B]] \mid [[A (x) B]] \mid [[A (+) B]] \\
        \text{(Patterns)} & [[p]] ::= [[stp]] \mid [[-]] \mid [[x]] \mid [[p1 (x)
        p2]] \mid [[p1 (+) p2]]\\
        \text{(Terms)}          & [[t]], [[e]] ::= [[x]] \mid [[*]] \mid [[o]] \mid
        [[t1 (x) t2]] \mid [[t1 (+) t2]] \mid [[\x.t]] \mid [[let t be p in e]] \mid [[t1 t2]]\\
        \text{(Left Contexts)}  & [[G]] ::= [[.]] \mid [[x : A]] \mid [[G1,G2]]\\
        \text{(Right Contexts)} & [[L]] ::= [[.]] \mid [[t : A]] \mid [[L1,L2]]\\
      \end{array}
    \end{math}
  \end{center}
\end{definition}

At this point we introduce some basic syntax and definitions to
facilitate readability, and presentation of the inference rules.
First, we will often write $[[L1 | L2]]$ as syntactic sugar for
$[[L1,L2]]$.  The former syntax should be read as ``$[[L1]]$ or
$[[L2]]$.'' This will help readability of the sequent we will
introduce below.  We denote the usual capture-avoiding substitution by
$[[ [t/x]t']]$.

\begin{definition}
  \label{def:delta-sub}
  We extend the capture-avoiding substitution function to right
  contexts as follows:
  \begin{center}
    \begin{math}
      \begin{array}{lll}
        [[ [t/x] .]] = [[.]]\\
        [[ [t/x] (t' : A)]] = [[{[t/x]t'} : A]]\\
        [[ [t/x] (L1 | L2)]] = [[([t/x]L1) | ([t/x]L2)]]\\
      \end{array}
    \end{math}
  \end{center}
\end{definition}
The previous extension will make conducting substitutions across a
sequence of terms in an inference rule easier.  Similarly, we find
it convenient to be able to do this style of extension for the
let-binding as well.
\begin{definition}
  \label{def:delta-let}
  We extend let-binding terms to right contexts as follows:
  \begin{center}
    \begin{math}
      \begin{array}{lll}
        [[ let t be p in .]] = [[.]]\\
        [[ let t be p in (t' : A)]] = [[{let t be p in t'} : A]]\\
        [[ let t be p in (L1 | L2)]] = [[(let t be p in L1) | (let t be p in L2)]]\\
      \end{array}
    \end{math}
  \end{center}
\end{definition}
We denote the usual function that computes the set of free variables
in a term by $\mathsf{FV}([[t]])$.
\begin{definition}
  \label{def:delta-FV}
  We extend the free-variable function on terms to right contexts as
  follows:
  \begin{center}
    \begin{math}
      \begin{array}{lll}
        [[FV(.)]] = \emptyset\\
        [[FV(t : A)]] = [[FV(t)]]\\
        [[FV(L1 | L2)]] = [[FV(L1)]] \cup [[FV(L2)]]\\
      \end{array}
    \end{math}
  \end{center}
\end{definition}

Finally, we arrive at the inference rules of FILL.
\begin{definition}
  \label{def:infr}
  The inference rules for derivability in FILL are as follows:
  \begin{center}
      \begin{mathpar}
        \FILLdruleAx{}    \and 
        \FILLdruleCut{}     \and 
        \FILLdruleIl{}            \and 
        \FILLdruleIr{}    \and 
        \FILLdruleTl{}    \and 
        \FILLdruleTr{}    \and 
        \FILLdrulePl{}    \and 
        \FILLdrulePr{}    \and 
        \FILLdruleParl{}    \and 
        \FILLdruleParr{}    \and 
        \FILLdruleImpl{}    \and 
        \FILLdruleImpr{}    \and 
        \FILLdruleExl{}    \and 
        \FILLdruleExr{}    
    \end{mathpar}
  \end{center}
\end{definition}

The $\FILLdrulename{Parl}$ rule depends on a function $[[let-pat z p
L]]$.  We define this function next.
\begin{definition}
  \label{def:let-pat-term}
  The function $[[let-pat z p t]]$ is defined as follows:
  \begin{center}
    \begin{math}
      \begin{array}{lll}      
        [[let-pat z (x (+) -) t]] = [[t]]\\
        \,\,\,\,\,\,\text{where } [[x]] \not\in \mathsf{FV}([[t]])\\
        & \\
        [[let-pat z (- (+) y) t]] = [[t]]\\
        \,\,\,\,\,\,\text{where } [[y]] \not\in \mathsf{FV}([[t]])\\
        & \\
        [[let-pat z p t]] = [[let z be p in t]]\\
      \end{array}
    \end{math}
  \end{center}
  We can then extend the previous definition to right-contexts as
  follows:
  \begin{center}
    \begin{math}
      \begin{array}{lll}      
        [[let-pat z p .]] = [[.]]\\                
        [[let-pat z p (t : A)]] = [[{let-pat z p t} : A]]\\
        [[let-pat z p (L1 | L2)]] = [[(let-pat z p L1) | (let-pat z p L2)]]\\
      \end{array}
    \end{math}
  \end{center}
\end{definition}
The motivation behind this function is that it only binds the pattern
variables in $[[p]]$ in a term if and only if those pattern variables
are free in the term.  This over comes the counterexample given by
Bierman in \cite{Bierman:1996}.  Throughout the sequel we will denote
derivations of the previous rules by $\pi$.
% section full_intuitionistic_linear_logic_(fill) (end)

\section{Basic Results}
\label{sec:basic_results}

In this section we simply list several basic results needed throughout
the sequel:

\begin{lemma}[Substitution Distribution]
  \label{lemma:substitution_distribution}
  For any terms $[[t]]$, $[[t1]]$, and $[[t2]]$, $[[ [t1/x][t2/y]t]] = [[ [ [t1/x]t2/y][t2/x]t]]$.
\end{lemma}
\begin{proof}
  This proof holds by straightforward induction on the form of $t$.
\end{proof}
% section basic_results (end)


\section{Cut-elimination}
\label{sec:cut-elimination}
The usual proof of cut-elimination for intuitionistic and classical
linear logic should suffice for FILL.  Thus, in this section we simply
give the cut-elimination procedure for FILL following the development
in \cite{Mellies:2009}.  However, there is one invariant that must be
verified across each derivation transformation.  The invariant is that
if a derivation $\pi$ is transformed into a derivation $\pi'$, then
the terms in the conclusion of the final rule applied in $\pi$ must be
equivalent to the terms in the conclusion of the final rule applied in
$\pi'$, but using what notion of equivalence?

\begin{definition}
  \label{def:FILL-eq}
  Equivalence on terms is defined as follows:
  \begin{center}
    \begin{mathpar}
      \FILLdruleEqXXAlpha{} \and
      \FILLdruleEqXXBeta{} \and
      \FILLdruleEqXXEta{} \and
      \FILLdruleEqXXI{} \and
      \FILLdruleEqXXStp{} \and
      \FILLdruleEqXXTOne{} \and
      \FILLdruleEqXXTTwo{} \and
      \FILLdruleEqXXPOne{} \and
      \FILLdruleEqXXPTwo{} \and
      \FILLdruleEqXXPThree{} \and
      \FILLdruleEqXXLam{} \and
      \FILLdruleEqXXAppOne{} \and
      \FILLdruleEqXXAppTwo{} \and
      \FILLdruleEqXXTenOne{} \and
      \FILLdruleEqXXTenTwo{} \and
      \FILLdruleEqXXParOne{} \and
      \FILLdruleEqXXParTwo{} \and
      \FILLdruleEqXXLetOne{} \and
      \FILLdruleEqXXLetTwo{} \and
      \FILLdruleEqXXRefl{} \and
      \FILLdruleEqXXSym{} \and
      \FILLdruleEqXXTrans{}
    \end{mathpar}
  \end{center}
\end{definition}

Throughout the remainder of this section we give each transformation
of derivations, and then prove that the terms maintain equivalence
across each transformation.
\subsection{Commuting conversion cut vs cut (first case)}
\label{subsec:commuting_conversion_cut_vs_cut_(first_case)}
The following proof
\begin{center}
  \begin{math}
    $$\mprset{flushleft}
    \inferrule* [right=Cut] {
      $$\mprset{flushleft}
      \inferrule* [right=] {
        \inferrule* [right=,vdots=1.5em,fraction=\,] {
          \,
        }{\pi_1}
      }{[[G |- t : A | L]]}
      \\
      $$\mprset{flushleft}
      \inferrule* [right=Cut] {
        $$\mprset{flushleft}
        \inferrule* [right=] {
          \inferrule* [right=,vdots=1.5em,fraction=\,] {
            \,
          }{\pi_2}
        }{[[G2,x : A,G3 |- t1 : B | L1]]}
        \\
        $$\mprset{flushleft}
        \inferrule* [right=] {
          \inferrule* [right=,vdots=1.5em,fraction=\,] {
            \,
          }{\pi_3}
        }{[[G1,y : B,G4 |- L2]]}
      }{[[G1,G2,x : A,G3,G4 |- h(L1 | [t1/y]L2)]]}
    }{[[G1,G2,G,G3,G4 |- h(L | [t/x]L1) | [t/x]h([t1/y]L2)]]}
  \end{math}
\end{center}
is transformed into the proof
\begin{center}
  \begin{math}
    $$\mprset{flushleft}
    \inferrule* [right=Cut] {
      $$\mprset{flushleft}
      \inferrule* [right=] {
        $$\mprset{flushleft}
      \inferrule* [right=] {
        \inferrule* [right=,vdots=1.5em,fraction=\,] {
          \,
        }{\pi_1}               
      }{[[G |- t : A | L]]}
      \\
      $$\mprset{flushleft}
        \inferrule* [right=] {
          \inferrule* [right=,vdots=1.5em,fraction=\,] {
            \,
          }{\pi_2}
        }{[[G2,x : A,G3 |- t1 : B | L1]]}
      }{[[G2,G,G3 |- h{[t/x]t1} : B | [t/x]L1]]}
      \\
      $$\mprset{flushleft}
        \inferrule* [right=] {
          \inferrule* [right=,vdots=1.5em,fraction=\,] {
            \,
          }{\pi_3}
        }{[[G1,y : B,G4 |- L2]]}
    }{[[G1,G2,G,G3,G4 |- h(L | [t/x]L1) | [ [t/x]t1/y]L2]]}
  \end{math}
\end{center}
First, if $[[L2]]$ is empty, then all the terms in the conclusion of
the previous two derivations are equivalent.  
So suppose $[[L2]] = [[t2 : C | L2']]$.  Then we know that the term
$[[ [t/x][t1/y]t2]]$ in the first derivation above is equivalent to
$[[ [ [t/x] t1/y][t/x] t2]]$ by
Lemma~\ref{lemma:substitution_distribution}.  Furthermore, by
inspecting the first derivation we can see that $[[x nin FV(t2)]]$,
and thus, $[[ [ [t/x] t1/y][t/x] t2 = [ [t/x] t1/y] t2]]$.  This
argument may be repeated for any term in $[[L2']]$, and thus, we know
$[[ [t/x][t1/y]L2 = [ [t/x]t1/y]L2]]$.
% subsection commuting_conversion_cut_vs_cut_(first_case) (end)

\subsection{Commuting conversion cut vs. cut (second case)}
\label{subsec:commuting_conversion_cut_vs._cut_(second_case)}
The second commuting conversion on cut begins with the proof
\begin{center}
  \begin{math}
    $$\mprset{flushleft}
    \inferrule* [right=Cut] {
      $$\mprset{flushleft}
      \inferrule* [right=] {
        \inferrule* [right=,vdots=1.5em,fraction=\,] {
          \,
        }{\pi_1}
      }{[[G |- t : A | L]]}
      \\
      $$\mprset{flushleft}
      \inferrule* [right=Cut] {
        $$\mprset{flushleft}
        \inferrule* [right=] {
          \inferrule* [right=,vdots=1.5em,fraction=\,] {
            \,
          }{\pi_2}
        }{[[G' |- t' : B | L']]}
        \\
        $$\mprset{flushleft}
        \inferrule* [right=] {
          \inferrule* [right=,vdots=1.5em,fraction=\,] {
            \,
          }{\pi_3}
        }{[[G1,x : A,G2,y:B,G3 |- L1]]}
      }{[[G1,x : A,G2,G',G3 |- L' | [t'/y]L1]]}
    }{[[G1,G,G2,G',G3 |- h(L | [t/x]L') | [t/x]h([t'/y]L1)]]}
  \end{math}
\end{center}
is transformed into the following proof:
\begin{center}
  \begin{math}
    $$\mprset{flushleft}
    \inferrule* [right=\text{\scriptsize Series of Exchanges}] {
      $$\mprset{flushleft}
    \inferrule* [right=Cut] {
      $$\mprset{flushleft}
      \inferrule* [right=] {
        \inferrule* [right=,vdots=1.5em,fraction=\,] {
          \,
        }{\pi_2}
      }{[[G' |- t' : B | L']]}
      \\
      $$\mprset{flushleft}
      \inferrule* [right=Cut] {
        $$\mprset{flushleft}
        \inferrule* [right=] {
          \inferrule* [right=,vdots=1.5em,fraction=\,] {
            \,
          }{\pi_1}
        }{[[G |- t : A | L]]}      
        \\
        $$\mprset{flushleft}
        \inferrule* [right=] {
          \inferrule* [right=,vdots=1.5em,fraction=\,] {
            \,
          }{\pi_3}
        }{[[G1,x : A,G2,y:B,G3 |- L1]]}
      }{[[G1,G,G2,y:B,G3 |- L | [t/x]L1]]}
    }{[[G1,G,G2,G',G3 |- h(L' | [t'/y]L) | [t'/y]h([t/x]L1)]]}
    }{[[G1,G,G2,G',G3 |- h(h([t'/y]L) | L') | [t'/y]h([t/x]L1)]]}
  \end{math}
\end{center}
Now, because we know $[[x,y nin FV(L)]]$ by inspection of the first
derivation, we know that $[[L = [t'/y]L]]$ and $[[L' = [t/x]L']]$.  If
$[[L1]]$ is empty, then we obtain our result, so suppose $[[L1]] =
[[t1 : C | L'1]]$. Then we know that $[[x,y nin FV(t)]]$ and $[[x,y
nin FV(t')]]$.  Thus, by this fact and
Lemma~\ref{lemma:substitution_distribution}, we know that $[[
[t/x][t'/y]t1 = [ [t/x]t'/y][t/x]t1]] = [[ [t'/y][t/x]t1 ]]$.  This
argument can be repeated for any term in $[[L'1]]$, hence, $[[
[t/x][t'/y]L1 = [t'/y][t/x]L1]]$.
% subsection commuting_conversion_cut_vs._cut_(second_case) (end)

\subsection{The $\eta$-expansion cases}
\label{subsec:eta-expansion_cases}

\subsubsection{Tensor}
\label{subsec:tensor}
The proof
\begin{center}
  \begin{math}
    $$\mprset{flushleft}
    \inferrule* [right=Ax] {
      \,
    }{[[x : A (x) B |- x : A (x) B]]}
  \end{math}
\end{center}
is transformed into the proof
\begin{center}
  \begin{math}
    $$\mprset{flushleft}
    \inferrule* [right=Tl] {
      $$\mprset{flushleft}
      \inferrule* [right=Tr] {
        $$\mprset{flushleft}
        \inferrule* [right=Ax] {
          \,
        }{[[y : A |- y : A]]}
        \\
        $$\mprset{flushleft}
        \inferrule* [right=Ax] {
          \,
        }{[[z : B |- z : B]]}
      }{[[y : A, z : B |- y (x) z : A (x) B]]}
    }{[[x : A (x) B |- h{let x be y (x) z in {y (x) z}} : A (x) B]]}
  \end{math}
\end{center}
Now by the rule $\FILLdrulename{EQ\_T2}$ we know $[[let x be y (x) z
in {y (x) z} = x]]$.
% subsubsection tensor (end)

\subsubsection{Par}
\label{subsec:par}
The proof
\begin{center}
  \begin{math}
    $$\mprset{flushleft}
    \inferrule* [right=Ax] {
      \,
    }{[[x : A (+) B |- x : A (+) B]]}
  \end{math}
\end{center}
is transformed into the proof
\begin{center}
  \begin{math}
    $$\mprset{flushleft}
    \inferrule* [right=Parr] {
      $$\mprset{flushleft}
      \inferrule* [right=Parl] {
        $$\mprset{flushleft}
        \inferrule* [right=Ax] {
          \,
        }{[[y : A |- y : A]]}
        \\
        $$\mprset{flushleft}
        \inferrule* [right=Ax] {
          \,
        }{[[z : B |- z : B]]}
      }{[[x : A (+) B |- h{let x be (y (+) -) in y} : A | h{let x be (- (+) z) in z} : B]]}
    }{[[x : A (+) B |- {let x be (y (+) -) in y} (+) {let x be (- (+) z) in z} : A (+) B]]}
  \end{math}
\end{center}
Just as we saw in the previous case by rule $\FILLdrulename{Eq\_P3}$
we know $[[{{let x be (y (+) -) in y} (+) {let x be (- (+) z) in z}} =
x]]$.
% subsubsection par (end)

\subsubsection{Implication}
\label{subsec:implication}
The proof
\begin{center}
  \begin{math}
    $$\mprset{flushleft}
    \inferrule* [right=Ax] {
      \,
    }{[[x : A -o B |- x : A -o B]]}
  \end{math}
\end{center}
transforms into the proof 
\begin{center}
  \begin{math}
    $$\mprset{flushleft}
    \inferrule* [right=ImpR] {
      $$\mprset{flushleft}
      \inferrule* [right=ImpL] {
        $$\mprset{flushleft}
        \inferrule* [right=Ax] {
          \,
        }{[[y : A |- y : A]]}
        \\
        $$\mprset{flushleft}
        \inferrule* [right=Ax] {
          \,
        }{z : B |- z : B}
      }{[[y : A, x : A -o B |- x y : B]]}
    }{[[x : A -o B |- \y.h{x y} : A -o B]]}
  \end{math}  
\end{center}
Finally, all terms in the two derivations are equivalent, because
$[[{\y.h{x y}} = x]]$ by the $\FILLdrulename{Eq\_Eta}$ rule.
% subsubsection implication (end)

\subsubsection{Tensor unit}
\label{subsec:tensor_unit}
The proof
\begin{center}
  \begin{math}
    $$\mprset{flushleft}
    \inferrule* [right=Ax] {
      \,
    }{[[x : I |- x : I]]}
  \end{math}
\end{center}
transforms into the proof
\begin{center}
  \begin{math}
    $$\mprset{flushleft}
    \inferrule* [right=Il] {
      $$\mprset{flushleft}
      \inferrule* [right=Ir] {
        \,
      }{[[. |- * : I]]}
    }{[[x : I |- h{let x be stp in *} : I]]}
  \end{math}
\end{center}
Lastly, we know $[[x = let x be stp in *]]$ by
$\FILLdrulename{Eq\_I}$.
% subsubsection tensor_unit (end)
% subsection eta-expansion_cases (end)

\section{The axiom steps}
\label{sec:the_axiom_steps}
\subsection{The axiom step}
\label{subsec:the_axiom_step}
The proof 
\begin{center}
  \begin{math}
    $$\mprset{flushleft}
    \inferrule* [right=Cut] {
      $$\mprset{flushleft}
      \inferrule* [right=Ax] {
        \,
      }{[[x : A |- x : A]]}
      \\
      $$\mprset{flushleft}
      \inferrule* [right=] {
        \inferrule* [right=,vdots=1.5em,fraction=\,] {
            \,
          }{\pi}          
      }{[[G1,y : A, G2 |- L]]}
    }{[[G1,x : A, G2 |- [x/y]L]]}
  \end{math}
\end{center}
transforms into the proof
\begin{center}
  \begin{math}
    $$\mprset{flushleft}
      \inferrule* [right=] {
        \inferrule* [right=,vdots=1.5em,fraction=\,] {
            \,
          }{\pi}          
      }{[[G1,y : A, G2 |- L]]}
  \end{math}
\end{center}
By $\FILLdrulename{Eq\_Alpha}$, we know, for any $[[t]]$ in
$[[L]]$, $[[t = [x/y]t]]$, and hence $[[L = [x/y]L]]$.
% subsection the_axiom_step (end)

\subsection{Conclusion vs. axom}
\label{subsec:conclusion_vs._axom}
The proof 
\begin{center}
  \begin{math}
    $$\mprset{flushleft}
    \inferrule* [right=Cut] {
      $$\mprset{flushleft}
      \inferrule* [right=] {
        \inferrule* [right=,vdots=1.5em,fraction=\,] {
            \,
          }{\pi}          
      }{[[G |- t : A | L]]}
      \,
      $$\mprset{flushleft}
      \inferrule* [right=Ax] {
        \,
      }{[[x : A |- x : A]]}
    }{[[G |- L | h{[t/x]x} : A]]}
  \end{math}
\end{center}
transforms into 
\begin{center}
  \begin{math}
    $$\mprset{flushleft}
    \inferrule* [right=\text{\tiny Series of Exchanges}] {
      \inferrule* [right=] {
        \inferrule* [right=,vdots=1.5em,fraction=\,] {
            \,
          }{\pi}          
      }{[[G |- t : A | L]]}
    }{[[G |- L | t : A ]]}
  \end{math}
\end{center}
By the definition of the substitution function we know $[[t =
[t/x]x]]$.
% subsection conclusion_vs._axom (end)
% section the_axiom_steps (end)

\subsection{The exchange steps}
\label{subsec:the_exchange_steps}
\subsubsection{Conclusion vs. left-exchange (the first case)}
\label{subsec:conclusion_vs._exchange_(the_first_case)}
The proof
\begin{center}
  \begin{math}
    $$\mprset{flushleft}
    \inferrule* [right=Cut] {
      \inferrule* [right=] {
        \inferrule* [right=,vdots=1.5em,fraction=\,] {
            \,
          }{\pi_1}          
      }{[[G |- t : A | L]]}
      \\
      $$\mprset{flushleft}
      \inferrule* [right=Exl] {        
        $$\mprset{flushleft}
        \inferrule* [right=] {
          \inferrule* [right=,vdots=1.5em,fraction=\,] {
            \,
          }{\pi_2}          
        }{[[G1,x : A, y : B, G2 |- L']]}        
      }{[[G1,y : B,x : A, G2 |- L']]}
    }{[[G1,y : B,G, G2 |- L | [t/x]L']]}
  \end{math}
\end{center}
transforms into the proof
\begin{center}
  \begin{math}
    $$\mprset{flushleft}
    \inferrule* [right=\text{Series of Exchanges}] {
      $$\mprset{flushleft}
      \inferrule* [right=Cut] {
        \inferrule* [right=] {
        \inferrule* [right=,vdots=1.5em,fraction=\,] {
            \,
          }{\pi_1}          
      }{[[G |- t : A | L]]}
      \\
        $$\mprset{flushleft}
        \inferrule* [right=] {
          \inferrule* [right=,vdots=1.5em,fraction=\,] {
            \,
          }{\pi_2}          
        }{[[G1,x : A, y : B, G2 |- L']]}        
      }{[[G1,G,y : B, G2 |- L | [t/x]L']]}
    }{[[G1,y : B,G, G2 |- L | [t/x]L']]}
  \end{math}
\end{center}
Clearly, all terms are equivalent.
% subsubsection conclusion_vs._exchange_(the_first_case) (end)

\subsubsection{Conclusion vs. left-exchange (the second case)}
\label{subsec:conclusion_vs._exchange_(the_second_case)}
The proof
\begin{center}
  \begin{math}
    $$\mprset{flushleft}
    \inferrule* [right=Cut] {
      \inferrule* [right=] {
        \inferrule* [right=,vdots=1.5em,fraction=\,] {
            \,
          }{\pi_1}          
      }{[[G |- t : B | L]]}
      \\
      $$\mprset{flushleft}
      \inferrule* [right=Exl] {        
        $$\mprset{flushleft}
        \inferrule* [right=] {
          \inferrule* [right=,vdots=1.5em,fraction=\,] {
            \,
          }{\pi_2}          
        }{[[G1,x : A, y : B, G2 |- L']]}        
      }{[[G1,y : B,x : A, G2 |- L']]}
    }{[[G1,G, x : A,G2 |- L | [t/y]L']]}
  \end{math}
\end{center}
transforms into the proof
\begin{center}
  \begin{math}
    $$\mprset{flushleft}
    \inferrule* [right=\text{Series of Exchanges}] {
      $$\mprset{flushleft}
      \inferrule* [right=Cut] {
        \inferrule* [right=] {
        \inferrule* [right=,vdots=1.5em,fraction=\,] {
            \,
          }{\pi_1}          
      }{[[G |- t : B | L]]}
      \\
        $$\mprset{flushleft}
        \inferrule* [right=] {
          \inferrule* [right=,vdots=1.5em,fraction=\,] {
            \,
          }{\pi_2}          
        }{[[G1,x : A, y : B, G2 |- L']]}        
      }{[[G1,x : A,G,G2 |- L | [t/y]L']]}
    }{[[G1,G,x : A,G2 |- L | [t/y]L']]}
  \end{math}
\end{center}
Clearly, all terms are equivalent.
% subsubsection conclusion_vs._exchange_(the_second_case) (end)

\subsubsection{Conclusion vs. right-exchange}
\label{subsec:conclusion_vs._right-exchange}
The proof
\begin{center}
  \begin{math}
    $$\mprset{flushleft}
    \inferrule* [right=Cut] {
        \inferrule* [right=] {
        \inferrule* [right=,vdots=1.5em,fraction=\,] {
            \,
          }{\pi_1}          
      }{[[G |- t : A | L]]}
      \\
      $$\mprset{flushleft}
      \inferrule* [right=Exr] {
        $$\mprset{flushleft}
        \inferrule* [right=] {
          \inferrule* [right=,vdots=1.5em,fraction=\,] {
            \,
          }{\pi_2}          
        }{[[G1,x : A, G2 |- h(h(L1 | t1 : B) | t2 : C) | L']]}        
      }{[[G1,x : A, G2 |- h(h(L1 | t2 : C) | t1 : B) | L']]}
    }{[[G1,G,G2 |- L | h(h(h(h([t/x]L1) | h{[t/x]t2} : C) | h{[t/x]t1} : B) | [t/x]L')]]}
  \end{math}
\end{center}
transforms into this proof
\begin{center}
  \begin{math}
    $$\mprset{flushleft}
    \inferrule* [right=Exr] {
      $$\mprset{flushleft}
      \inferrule* [right=Cut] {
        \inferrule* [right=] {
        \inferrule* [right=,vdots=1.5em,fraction=\,] {
            \,
          }{\pi_1}          
      }{[[G |- t : A | L]]}
      \\
      $$\mprset{flushleft}
        \inferrule* [right=] {
          \inferrule* [right=,vdots=1.5em,fraction=\,] {
            \,
          }{\pi_2}          
        }{[[G1,x : A, G2 |- h(h(L1 | t1 : B) | t2 : C) | L']]}        
      }{[[G1,G, G2 |- L | h(h(h(h([t/x]L1) | h{[t/x]t1} : B) | h{[t/x]t2} : C) | [t/x]L')]]}
    }{[[G1,G, G2 |- h(h(h([t/x]L1) | h{[t/x]t2} : C) | h{[t/x]t1} : B) | [t/x]L']]}
  \end{math}
\end{center}
% subsubsection conclusion_vs._right-exchange_(the_first_case) (end)
% subsection the_exchange_steps (end)

\subsection{Principle formula vs. principle formula}
\label{subsec:principle_formula_vs._principle_formula}

\subsubsection{Tensor}
\label{subsec:tensor}
The proof 
\begin{center}

  \begin{math}
    $$\mprset{flushleft}
    \inferrule* [right=\footnotesize Cut] {
      $$\mprset{flushleft}
      \inferrule* [right=\footnotesize Tr] {
        \inferrule* [right=] {
        \inferrule* [right=,vdots=1.5em,fraction=\,] {
            \,
          }{\pi_1}          
      }{[[G1 |- t1 : A | L1]]}
      \\
      \inferrule* [right=] {
        \inferrule* [right=,vdots=1.5em,fraction=\,] {
            \,
          }{\pi_2}          
      }{[[G2 |- t2 : B | L2]]}
      }{[[G1,G2 |- t1 (x) t2 : A (x) B | h(L1 | L2)]]}
      \\
      $$\mprset{flushleft}
      \inferrule* [right=\footnotesize Tl] {
        \inferrule* [right=] {
          \inferrule* [right=,vdots=1.5em,fraction=\,] {
            \,
          }{\pi_3}          
        }{[[G3,x : A, y : B,G4 |- L3]]}
      }{[[G3,z : A (x) B,G4 |- h(let z be x (x) y in L3)]]}
    }{[[G3,G1,G2,G4 |- h(L1 | L2) | [t1 (x) t2/z](h(let z be x (x) y in L3))]]}
  \end{math}
\end{center}
is transformed into the proof
\begin{center}
  \begin{math}
    $$\mprset{flushleft}
    \inferrule* [right=Cut] {
      \inferrule* [right=] {
        \inferrule* [right=,vdots=1.5em,fraction=\,] {
            \,
          }{\pi_1}          
      }{[[G1 |- t1 : A | L1]]}
      \\
      $$\mprset{flushleft}
      \inferrule* [right=Cut] {
        \inferrule* [right=] {
        \inferrule* [right=,vdots=1.5em,fraction=\,] {
            \,
          }{\pi_2}          
      }{[[G2 |- t2 : B | L2]]}
      \\
      \inferrule* [right=] {
          \inferrule* [right=,vdots=1.5em,fraction=\,] {
            \,
          }{\pi_3}          
        }{[[G3,x : A, y : B,G4 |- L3]]}
      }{[[G3,x : A, G2,G4 |- L2 | [t2/y]L3]]}
    }{[[G3,G1, G2,G4 |- h(L1 | L2) | [t1/x][t2/y]L3]]}
  \end{math}
\end{center}
If $[[L3]]$ is empty, then our result follows.  So suppose $[[L3 = t3
: C, L'3]]$.  We can see that $[[h{[t1 (x) t2/z]{let z be x (x) y in
    t3}} = let t1 (x) t2 be x (x) y in t3]]$ by the definition of
substitution, and by using the $\FILLdrulename{Eq\_T1}$ rule we obtain
$[[let t1 (x) t2 be x (x) y in t3 = [t1/x][t2/y]t3]]$.  This argument
can be repeated for any term in $[[ [t1 (x) t2/z](h(let z be x (x) y
in L'3))]]$, and thus, $[[ [t1 (x) t2/z](h(let z be x (x) y in L3)) =
[t1/x][t2/y]L3]]$.

Note that in the second derivation of the above transformation we
first cut on $[[B]]$, and then $[[A]]$, but we could have cut on
$[[A]]$ first, and then $[[B]]$, but this would yeild equivalent
derivations as above by using
Lemma~\ref{lemma:substitution_distribution}.
% subsubsection tensor (end)

\subsubsection{Par}
\label{subsec:par}
The proof
\begin{center}
  \scriptsize
  \begin{math}
    $$\mprset{flushleft}
\inferrule* [right=\scriptsize Cut] {
  $$\mprset{flushleft}
  \inferrule* [right=\scriptsize Parr] {
    \inferrule* [right=] {
        \inferrule* [right=,vdots=1.5em,fraction=\,] {
            \,
          }{\pi_1}          
      }{[[G1 |- h(L1 | t1 : A) | h(t2 : B | L2)]]}
    }{[[G1 |- h(L1 | t1 (+) t2 : A (+) B) | L2]]}
  \\
  $$\mprset{flushleft}
  \inferrule* [right=\scriptsize Parl] {
    \inferrule* [right=] {
        \inferrule* [right=,vdots=1.5em,fraction=\,] {
            \,
          }{\pi_2}          
      }{[[G2,x : A |- L3]]}
      \\
      \inferrule* [right=] {
        \inferrule* [right=,vdots=1.5em,fraction=\,] {
            \,
          }{\pi_3}          
      }{[[G3, y : B |- L4]]}
  }{[[G2,G3,z : A (+) B |- h(let-pat z (x (+) -) L3) | h(let-pat z (- (+) y) L4)]]}
}{[[G2,G3,G1 |- h(h(L1 | L2) | h([t1 (+) t2/z](let-pat z (x (+) -) L3))) | h([t1 (+) t2/z](let-pat z (x (+) -) L4))]]}
  \end{math}
\end{center}
is transformed into the proof
\begin{center}
  \begin{math}
    $$\mprset{flushleft}
    \inferrule* [right=\tiny Series of Exchanges] {
      $$\mprset{flushleft}
    \inferrule* [right=Cut] {
      $$\mprset{flushleft}
      \inferrule* [right=Cut] {
        \inferrule* [right=] {
          \inferrule* [right=,vdots=1.5em,fraction=\,] {
            \,
          }{\pi_1}          
        }{[[G1 |- h(L1 | t1 : A) | h(t2 : B | L2)]]}
        \\
        \inferrule* [right=] {
          \inferrule* [right=,vdots=1.5em,fraction=\,] {
            \,
          }{\pi_3}          
        }{[[G3, y : B |- L4]]}
      }{[[G3, G1 |- h(h(L1 | t1 : A) | L2) | [t2/y]L4]]}
      \\
      \inferrule* [right=] {
        \inferrule* [right=,vdots=1.5em,fraction=\,] {
          \,
        }{\pi_2}          
      }{[[G2,x : A |- L3]]}
    }{[[G2,G3,G1 |- h(h(L1 | L2) | [t2/y]L4) | [t1/x]L3]]}
    }{[[G2,G3,G1 |- h(L1 | L2) | h(h([t1/x]L3) | [t2/y]L4)]]}
  \end{math}
\end{center}
Consider the case when $[[L3]] = [[t3 : C_1 | L'3]]$ and $[[L4]] =
[[t4 : C_2 | L'4]]$.  All other cases are either trivial or similar. 
First, $[[ [t1 (+) t2/z]{let-pat z (x (+) -) t3}
= let-pat {t1 (+) t2} (x (+) -) t3]]$, and $[[let-pat {t1 (+) t2} (x
(+) -) t3 = [t1/x]t3]]$ if $[[x in FV(t3)]]$ or $[[let-pat {t1 (+) t2}
(x (+) -) t3 = t3]]$ otherwise.  In the latter case we can see that
$[[t3 = [t1/x]t3]]$, thus, in both cases $[[let-pat {t1 (+) t2} (x (+)
-) t3 = [t1/x]t3]]$.  This argument can be repeated for any terms in
$[[L'3]]$, and hence $[[ [t1 (+) t2/z](let-pat z (x (+) -) L3)]] =
[[let-pat {t1 (+) t2} (x (+) -) L3 = [t1/x]L3]]$.  We can apply a
similar argument for $[[ [t1 (+) t2/z]{let-pat z (- (+) y) t4}]]$ and
$[[ [t1 (+) t2/x](let-pat z (- (+) y) L4)]]$.

Note that just as we mentioned about tensor we could have first cut on
$[[A]]$, and then on $[[B]]$ in the second derivation, but we would
have arrived at the same result just with potentially more exchanges
on the right.
% subsubsection par (end)

\subsubsection{Implication}
\label{subsubsec:implication}
The proof
\begin{center}
  \begin{math}
    $$\mprset{flushleft}
    \inferrule* [right=Cut] {
      $$\mprset{flushleft}
      \inferrule* [right=Impr] {
        \inferrule* [right=] {
          \inferrule* [right=,vdots=1.5em,fraction=\,] {
            \,
          }{\pi_1}          
        }{[[G, x : A |- t : B | L]]}
        \\
        [[x nin FV(L)]]
      }{[[G |- \x.t : A -o B | L]]}
      \\
      $$\mprset{flushleft}
      \inferrule* [right=Impl] {
        \inferrule* [right=] {
          \inferrule* [right=,vdots=1.5em,fraction=\,] {
            \,
          }{\pi_2}          
        }{[[G1 |- t1 : A | L1]]}
        \\
        \inferrule* [right=] {
          \inferrule* [right=,vdots=1.5em,fraction=\,] {
            \,
          }{\pi_3}          
        }{[[G2, y : B |- L2]]}
      }{[[G1, z : A -o B,G2 |- L1 | [z t1/y]L2]]}
    }{[[G1,G,G2 |- L | h(h([\x.t/z]L1) | h([\x.t/z][z t1/y]L2))]]}
  \end{math}
\end{center}
transforms into the proof
\begin{center}
  \scriptsize
  \begin{math}
    $$\mprset{flushleft}
    \inferrule* [right=\tiny Series of Exchanges] {
      $$\mprset{flushleft}
    \inferrule* [right=\scriptsize Cut] {
      $$\mprset{flushleft}
      \inferrule* [right=\scriptsize Cut] {
        \inferrule* [right=] {
          \inferrule* [right=,vdots=1.5em,fraction=\,] {
            \,
          }{\pi_2}          
        }{[[G1 |- t1 : A | L1]]}
        \\
        \inferrule* [right=] {
          \inferrule* [right=,vdots=1.5em,fraction=\,] {
            \,
          }{\pi_1}          
        }{[[G, x : A |- t : B | L]]}
        \\
        [[x nin FV(L)]]
      }{[[G, G1 |- L1 | h(h{[t1/x]t} : B | [t1/x]L)]]}
      \\
      \inferrule* [right=] {
          \inferrule* [right=,vdots=1.5em,fraction=\,] {
            \,
          }{\pi_3}          
        }{[[G2, y : B |- L2]]}
      }{[[G2, G,G1 |- h(L1 | [t1/x]L) | [ [t1/x]t/y]L2]]}
    }{[[G1, G,G2 |- h([t1/x]L) | h(L1 | h([ [t1/x]t/y]L2))]]}
  \end{math}
\end{center}
Consider the case when $[[L2]] = [[t2 : C | L'2]]$. All other cases are either trivial or similar.
First, by hypothesis we know $[[x nin FV(L)]]$, and so we know $[[L = [t1/x]L]]$.  Now we can see
that $[[ [\x.t/z][z t1/y]t2 = [{\x.t} t1/y]t2]] = [[ [ [t1/x]t/y]t2]]$ by using the congruence rules
of equality and the rule $\FILLdrulename{Eq\_Beta}$.  This argument can be repeated for any term
in $[[ [\x.t/z][z t1/y]L'2]]$, and so $[[ [\x.t/z][z t1/y]L2 = [ [t1/x]t/y]L2]]$.  Finally, by inspecting the previous
derivations we can see that $[[z nin FV(L1)]]$, and thus, $[[L1 = [\x.t/z]L1]]$.  
% subsubsection implication (end)

\subsubsection{Tensors Unit}
\label{subsubsec:ten-unit}
The proof
\begin{center}
  \begin{math}
    $$\mprset{flushleft}
    \inferrule* [right=Cut] {
      $$\mprset{flushleft}
      \inferrule* [right=Ir] {
        \,
      }{[[. |- * : I]]}
      \\
      $$\mprset{flushleft}
      \inferrule* [right=Il] {
        \inferrule* [right=] {
          \inferrule* [right=,vdots=1.5em,fraction=\,] {
            \,
          }{\pi}          
        }{[[G |- L]]}
      }{[[G, x : I |- let x be stp in L]]}
    }{[[G |- [*/x](let x be stp in L)]]}
  \end{math}
\end{center}
is transformed into the proof
\begin{center}
  \begin{math}
    \inferrule* [right=] {
      \inferrule* [right=,vdots=1.5em,fraction=\,] {
        \,
      }{\pi}          
    }{[[G |- L]]}
  \end{math}
\end{center}

Suppose $[[L]] = [[t : A | L']]$.  All other cases are either similar
or trivial.  We can see that $[[ [*/x]{let x be stp in t} = let * be
stp in t]] = [[t]]$ by the definition of substitution and the
$\FILLdrulename{Eq\_I}$ rule.  This argument can be repeated for any
term in $[[ [*/x](let x be stp in L')]]$, and hence, $[[ [*/x](let x be
stp in L) = L]]$.
% subsubsection ten-unit (end)

\subsubsection{Pars Unit}
\label{subsec:pars_unit}
The proof
\begin{center}
  \begin{math}
    $$\mprset{flushleft}
    \inferrule* [right=Cut] {
      $$\mprset{flushleft}
      \inferrule* [right=Pr] {
        \inferrule* [right=] {
          \inferrule* [right=,vdots=1.5em,fraction=\,] {
            \,
          }{\pi}          
        }{[[G |- L]]}
      }{[[G |- o : _|_ | L]]}
      \\
      $$\mprset{flushleft}
      \inferrule* [right=Pl] {
        \,
      }{[[x : _|_ |- .]]}
    }{[[G |- L | [o/x].]]}
  \end{math}
\end{center}
transforms into the proof
\begin{center}
  \begin{math}
    \inferrule* [right=] {
      \inferrule* [right=,vdots=1.5em,fraction=\,] {
        \,
      }{\pi}          
    }{[[G |- L]]}
  \end{math}
\end{center}
Clearly, $[[ [o/x]. = .]]$.
% subsubsection pars_unit (end)
% subsection principle_formula_vs._principle_formula (end)

% section cut-elimination (end)


\bibliographystyle{plain}
\bibliography{ref}

\appendix

\section{The full specification of FILL}
\label{sec:fill_specification}
\FILLall{}
% section the_full_fill_specification (end)


\end{document}

%%% Local Variables: 
%%% mode: latex
%%% TeX-master: t
%%% End: 