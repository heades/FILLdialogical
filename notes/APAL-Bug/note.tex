\documentclass{article}
\usepackage[utf8]{inputenc}
\usepackage{amssymb,amsmath,amsthm}
\usepackage{cmll}
\usepackage{stmaryrd}
\usepackage{todonotes}
\usepackage{mathpartir}
\usepackage{fullpage}

\title{A short note explaining the bug in the APAL term assignment formulation of FILL}
\author{Harley Eades III}
\date{August 2014}

%\usepackage{natbib}
%\usepackage{graphicx}

% Ott includes.
\usepackage{color}
\usepackage{supertabular}
\input{FILL-ott}

\begin{document}

\maketitle

In this short note I give the details of Bierman's counterexample
\cite{Bierman:1996} to cut elimination of the term assignment
formulation of FILL first given in \cite{Hyland:1993}.  I first
reformulate his counterexample into our definition of FILL, and then
comment on the reason for the counterexample.  Following this I
reformulate the counterexample in the dependency tracking system
proposed by Bra\"uner and de Paiva in \cite{Brauner:1996} and revised
by the same authors in \cite{Brauner:1998}.  In this reformulation we
will see that the rule proposed by Bellin but communicated by Bierman
in \cite{Bierman:1996} is the proper left rule for par.

\section{The $\FILLdrulename{Parl}$ Inference Rule}
\label{sec:the_parl_inference_rule}
The existing $\FILLdrulename{Parl}$ inference rule is as follows:
\[
  \FILLdruleParl{}
\]
In the terms $[[let t1 be x (+) stp in t2]]$ and $[[let t1 be stp (+) y
in t2]]$ the variables $[[x]]$ and $[[y]]$ in the patterns are bound
in $[[t2]]$. So when applying the $\FILLdrulename{Parl}$ rule we bind
both the free variable $[[x]]$ in $[[di]]$, and the free variable
$[[y]]$ in each $[[fi]]$.  Now notice that we do this binding even
when the variables are not free in the respective terms.  Furthermore,
as a result of binding these pattern variables we carry along the
newly introduced free variable $[[z]]$.  It is this global binding
across the entire righthand size context along with introducing the
free variable $[[z]]$ in each term that results in the counterexample
of Bierman.
% section the_$\filldrulename{parl}$_inference_rule (end)


\section{Bierman's Counterexample}
\label{sec:gavin's_counterexample}
First lets recall the cut-elimination commuting conversion that is the
locus of the counterexample.  The following cut:
\begin{center}
  \begin{math}
    $$\mprset{flushleft}
    \inferrule* [right= Cut] {
      [[G |- d : C | gk : D]]
      \\
      $$\mprset{flushleft}
      \inferrule* [right= Parl] {
        [[x : C, w : A, G' |- ei : D']]
        \\
        [[z : B, G'' |- ti : D'']]
      }{[[x : C, y : A (+) B, G', G'' |- let y be w (+) stp in ei : D' | let y be stp (+) z in ti : D'']]}
    }{[[G, y:A (+) B, G',G'' |- h( [d/x]{let y be w (+) stp in ei}:D' | [d/x]{let y be stp (+) z in tj} : D'') | gk : D]]}
  \end{math}
\end{center}
Converts into the following:
\begin{center}
  \begin{math}
    $$\mprset{flushleft}
    \inferrule* [right=Parl] {
      $$\mprset{flushleft}
      \inferrule* [right= Cut] {
        [[G |- d : C | gk : D]]
        \\
        [[x : C, w : A, G' |- ei : D']]
      }{[[w : A, G' |- [d/x]ei : D']]}
      \\
      [[z : B, G'' |- ti : D'']]
    }{[[G, y:A (+) B, G',G'' |- h( let y be w (+) stp in [d/x]ei:D' | let y be stp (+) z in tj : D'') | let y be w (+) stp in gk : D]]}
  \end{math}
\end{center}
Notice that in the above cut, the \FILLdrulename{Parl} rule commutes
with \FILLdrulename{Cut}.  So again, we bind $[[w]]$ as a pattern
variable in each $[[ei]]$, and $[[z]]$ in each $[[ti]]$ regardless of
whether or not these are actually free in any of the terms.  In
addition, we introduce $[[z]]$ into each of these terms.

Next we give Bierman's counterexample.  The following uses the first rule given above.
\begin{center}
      \footnotesize
  \begin{math}
    $$\mprset{flushleft}
    \inferrule* [right=\tiny Impr] {
      $$\mprset{flushleft}
      \inferrule* [right=\tiny Parr] {
        $$\mprset{flushleft}
        \inferrule* [right=\tiny Cut] {
          $$\mprset{flushleft}
          \inferrule* [right=\tiny Pr] {
            $$\mprset{flushleft}
            \inferrule* [right=\tiny Ax] {
              \,
            }{[[v : A |- v : A]]}
          }{[[v : A |- v : A | o : _|_]]}
          \\
          $$\mprset{flushleft}
          \inferrule* [right=\tiny Parl] {
            $$\mprset{flushleft}
            \inferrule* [right=\tiny Tr] {
              $$\mprset{flushleft}
              \inferrule* [right=\tiny Ax] {
                \,
              }{[[x : A |- x : A]]}
              \\
              $$\mprset{flushleft}
              \inferrule* [right=\tiny Ax] {
                \,
              }{[[y : B |- y : B]]}
            }{[[x : A, y : B |- x (x) y : A (x) B]]}
            \\
            $$\mprset{flushleft}
            \inferrule* [right=\tiny Ax] {
              \,
            }{[[w : C |- w : C]]}
          }{[[x : A, z : B (+) C |- let z be y (+) - in h{v (x) y} : A (x) B | let z be - (+) w in w : C]]}
        }{[[v:A,z:B(+)C |- h(let z be y (+) - in h{v (x) y} : A (x) B | let z be - (+) w in w : C) | o : _|_]]}
      }{[[v:A,z:B(+)C |- {let z be y (+) - in h{v (x) y}}(+){let z be - (+) w in w} : ((A(x)B)(+)C) | o : _|_]]}
    }{[[v : A |- \z.{{let z be y (+) - in h{v (x) y}}(+){let z be - (+) w in w}}:(B(+)C) -o ((A(x)B)(+)C) | o : _|_]]}
  \end{math}
\end{center}
Next we use the second derived rule above to commute the cut in the previous
derivation past the $\FILLdrulename{Parl}$ rule:
\begin{center}
  \footnotesize
  \begin{math}    
    $$\mprset{flushleft}
    \inferrule* [right=\tiny Impr] {
      $$\mprset{flushleft}
      \inferrule* [right=\tiny Parr] {
        $$\mprset{flushleft}
        \inferrule* [right=\tiny Parl] {
          $$\mprset{flushleft}
          \inferrule* [right=\tiny Cut] {      
            $$\mprset{flushleft}
            \inferrule* [right=\tiny Pr] {
              $$\mprset{flushleft}
              \inferrule* [right=\tiny Ax] {
                \,
              }{[[v : A |- v : A]]}
            }{[[v : A |- v : A | o : _|_]]}
            \\
            $$\mprset{flushleft}
            \inferrule* [right=\tiny Tr] {
              $$\mprset{flushleft}
              \inferrule* [right=\tiny Ax] {
                \,
              }{[[x : A |- x : A]]}
              \\
              $$\mprset{flushleft}
              \inferrule* [right=\tiny Ax] {
                \,
              }{[[y : B |- y : B]]}
            }{[[x : A, y : B |- x (x) y : A (x) B]]}
          }{[[y:B, v : A |- v (x) y: A (x) B | o : _|_]]}          
          \\
          $$\mprset{flushleft}
          \inferrule* [right=\tiny Ax] {
            \,
          }{[[w : C |- w : C]]}
        }{[[v:A,cb{z}:B(+)C |- h(let cb{z} be y (+) - in h{v (x) y} : A (x) B | let cb{z} be - (+) w in w : C) | let cb{z} be y (+) - in o : _|_]]}                   
      }{[[v:A,cb{z}:B(+)C |- {{let cb{z} be y (+) - in h{v (x) y}} (+) {let cb{z} be - (+) w in w}} : (A (x) B) (+) C | let cb{z} be y (+) - in o : _|_]]}      
    }{[[v:A |- \z.{{let z be y (+) - in h{v (x) y}} (+) {let z be - (+) w in w}} : (B(+)C) -o ((A (x) B) (+) C) | let cb{z} be y (+) - in o : _|_]]}
  \end{math}
\end{center}
Now notice that as a result of the rule $\FILLdrulename{Parl}$ rule a
fresh free variable $[[z]]$ -- colored blue when it is considered free
-- is introduced, and then let-bound in every term in the righthand
side context.  Furthermore, we bind $[[y]]$ and $[[w]]$ in terms which
do not depend on them, for example, we bind $[[y]]$ in $[[o]]$.
Furthermore, we introduce $[[z]]$ into each of these terms,
especially, the rightmost term.  Thus, the application of the
$\FILLdrulename{Impr}$ rule is in error, because $[[z]]$ occurs in the
right most term.
% section gavin's_counterexample (end)

\section{Bierman's Counterexample in the Dependency-Relation Formalization}
\label{sec:bierman's_counterexample_in_the_dependency-relation_formalization}
Next we give Bierman's counterexample in the dependency-relation
formalization.  To obtain the derivations we simply erase all the
terms:
\begin{center}
  \footnotesize
  \begin{math}
    $$\mprset{flushleft}
    \inferrule* [right=\tiny Dimpr] {
      $$\mprset{flushleft}
      \inferrule* [right=\tiny Dparr] {
        $$\mprset{flushleft}
        \inferrule* [right=\tiny Dcut] {
          $$\mprset{flushleft}
          \inferrule* [right=\tiny Dpr] {
            $$\mprset{flushleft}
            \inferrule* [right=\tiny Dax] {
              \,
            }{[[A |- A]]}
          }{[[A |- A | _|_]]}
          \\
          $$\mprset{flushleft}
          \inferrule* [right=\tiny Dparl] {
            $$\mprset{flushleft}
            \inferrule* [right=\tiny Dtr] {
              $$\mprset{flushleft}
              \inferrule* [right=\tiny Dax] {
                \,
              }{[[A |- A]]}
              \\
              $$\mprset{flushleft}
              \inferrule* [right=\tiny Dax] {
                \,
              }{[[B |- B]]}
            }{[[A, B |- A (x) B]]}
            \\
            $$\mprset{flushleft}
            \inferrule* [right=\tiny Dax] {
              \,
            }{[[C |- C]]}
          }{[[A, (B (+) C) |- (A (x) B) | C]]}
        }{[[A,B(+)C |- h( A (x) B | C ) | _|_]]}
      }{[[A,B(+)C |- ((A(x)B)(+)C) | _|_]]}
    }{[[A |- (B(+)C) -o ((A(x)B)(+)C) | _|_]]}
  \end{math}
\end{center}
Now if we compute the dependencies of the previous derivation up to
the application of the $\FILLdrulename{Dimpr}$ rule we will see that
the only formula on the righthand side of the premise of the
$\FILLdrulename{Dimpr}$ rule with dependencies is
$[[((A(x)B)(+)C)]]$.  The set of dependencies is as follows:
\[ \{ ([[B (+) C]] , [[(A (x) B) (+) C]]), 
      ([[A]] , [[(A (x) B) (+) C]]),
      ([[A]]' , [[A]]''), 
      ([[B']] , [[B'']]), 
      ([[C']] , [[C'']]) \} \]

We can easily see that $[[_|_]]$ has no dependencies and
rightfully so.  If we commute the cut just as before we obtain the
following derivation:
\begin{center}
  \footnotesize
  \begin{math}    
    $$\mprset{flushleft}
    \inferrule* [right=\tiny Dimpr] {
      $$\mprset{flushleft}
      \inferrule* [right=\tiny Dparr] {
        $$\mprset{flushleft}
        \inferrule* [right=\tiny Dparl] {
          $$\mprset{flushleft}
          \inferrule* [right=\tiny Dcut] {      
            $$\mprset{flushleft}
            \inferrule* [right=\tiny Dpr] {
              $$\mprset{flushleft}
              \inferrule* [right=\tiny Dax] {
                \,
              }{[[A |- A]]}
            }{[[A |- A | _|_]]}
            \\
            $$\mprset{flushleft}
            \inferrule* [right=\tiny Dtr] {
              $$\mprset{flushleft}
              \inferrule* [right=\tiny Dax] {
                \,
              }{[[A |- A]]}
              \\
              $$\mprset{flushleft}
              \inferrule* [right=\tiny Dax] {
                \,
              }{[[B |- B]]}
            }{[[A, B |- A (x) B]]}
          }{[[B, A |- A (x) B | _|_]]}          
          \\
          $$\mprset{flushleft}
          \inferrule* [right=\tiny Dax] {
            \,
          }{[[C |- C]]}
        }{[[A,B(+)C |- h(A (x) B | C) | _|_]]}                   
      }{[[A,B(+)C |- (A (x) B) (+) C | _|_]]}      
    }{[[A |- (B(+)C) -o ((A (x) B) (+) C) | _|_]]}
  \end{math}
\end{center}
As we can see without the terms, this derivation is nearly identical
to the previous, and hence, we obtain the exact same dependency set.  
However, we can use the dependency-relation formalization as a guiding
principle in fixing the term formalization.
% section bierman's_counterexample_in_the_dependency-relation_formalization (end)

\section{The Fix}
\label{sec:the_fix}
Consider the $\FILLdrulename{Dparl}$ rule in the dependency-relation
formalization:
\begin{center}
  \begin{math}
    \begin{array}{lll}
      \FILLdruleDparl{}
      && Dep(\tau) = \{([[A (+) B]], [[A]]), ([[A (+) B]], [[B]])\}
      \star (Dep(\tau_1) \cup Dep(\tau_2))
    \end{array}
  \end{math}
\end{center}
If anything in $[[L1]]$ and $[[L2]]$ depend on $[[A]]$ or $[[B]]$ then
this will be accounted for in $Dep(\tau_1)$ and $Dep(\tau_2)$.  Thus,
in the term formalization when binding pattern variables across the
righthand side of the sequent we should do so if and only if there is
a dependency.  In fact, if a formula on the righthand side depends on
a formula in the lefthand side, then the variable associated with
that hypnosis must be free in the term associated with the formula on
the right.  This evidence suggests that to fix the term formalization
we must modify the $\FILLdrulename{Parl}$ rule.

The new $\FILLdrulename{Parl}$ rule as follows:
\[
\FILLdruleNParl{}
\]
The previous rule depends on a function which we define as follows:
\begin{center}
  \begin{math}
    \begin{array}{lll}      
      [[let-pat z (x (+) -) e]] = [[e]]\\
      \,\,\,\,\,\,\text{where } [[x]] \not\in \mathsf{FV}([[e]])\\
      & \\
      [[let-pat z (- (+) y) e]] = [[e]]\\
      \,\,\,\,\,\,\text{where } [[y]] \not\in \mathsf{FV}([[e]])\\
      & \\
      [[let-pat z p e]] = [[let z be p in e]]\\
    \end{array}
  \end{math}
\end{center}
Note that in the definition of $[[let-pat z p e]]$ the final case is a
catchall case.  Now the new $\FILLdrulename{Parl}$ rule only pattern
matches on $[[z]]$ in the righthand side if there is a dependency
between the variables in the pattern and the term in the pattern
match.  A similar rule to the above was proposed by Bellin in the
conclusion of \cite{Bierman:1996}.

This rule recovers from the counterexample.  The first derivation
given in the counter example above is unchanged, so we only give the
second:
\begin{center}
  \footnotesize
  \begin{math}    
    $$\mprset{flushleft}
    \inferrule* [right=\tiny Impr] {
      $$\mprset{flushleft}
      \inferrule* [right=\tiny Parr] {
        $$\mprset{flushleft}
        \inferrule* [right=\tiny Parl] {
          $$\mprset{flushleft}
          \inferrule* [right=\tiny Cut] {      
            $$\mprset{flushleft}
            \inferrule* [right=\tiny Pr] {
              $$\mprset{flushleft}
              \inferrule* [right=\tiny Ax] {
                \,
              }{[[v : A |- v : A]]}
            }{[[v : A |- v : A | o : _|_]]}
            \\
            $$\mprset{flushleft}
            \inferrule* [right=\tiny Tr] {
              $$\mprset{flushleft}
              \inferrule* [right=\tiny Ax] {
                \,
              }{[[x : A |- x : A]]}
              \\
              $$\mprset{flushleft}
              \inferrule* [right=\tiny Ax] {
                \,
              }{[[y : B |- y : B]]}
            }{[[x : A, y : B |- x (x) y : A (x) B]]}
          }{[[y:B, v : A |- v (x) y: A (x) B | o : _|_]]}          
          \\
          $$\mprset{flushleft}
          \inferrule* [right=\tiny Ax] {
            \,
          }{[[w : C |- w : C]]}
        }{[[v:A,cb{z}:B(+)C |- h(let cb{z} be y (+) - in h{v (x) y} : A (x) B | let cb{z} be - (+) w in w : C) | o : _|_]]}                   
      }{[[v:A,cb{z}:B(+)C |- {{let cb{z} be y (+) - in h{v (x) y}} (+) {let cb{z} be - (+) w in w}} : (A (x) B) (+) C | o : _|_]]}      
    }{[[v:A |- \z.{{let z be y (+) - in h{v (x) y}} (+) {let z be - (+) w in w}} : (B(+)C) -o ((A (x) B) (+) C) | o : _|_]]}
  \end{math}
\end{center}
This new derivation is now correct, and mirrors that of the
dependency-relation formalization.
% section the_fix (end)

\bibliographystyle{plain}
\bibliography{ref}

\appendix

\section{The full specification of FILL}
\label{sec:fill_specification}
\FILLall{}
% section the_full_fill_specification (end)


\end{document}

