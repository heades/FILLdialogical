\documentclass{article}
\usepackage[utf8]{inputenc}
\usepackage{amssymb,amsmath,amsthm}
\usepackage{cmll}
\usepackage{stmaryrd}
\usepackage{todonotes}
\usepackage{mathpartir}
\usepackage{fullpage}

\title{A short note explaining the bug in the APAL term assignment formulation of FILL}
\author{Harley Eades III}
\date{August 2014}

%\usepackage{natbib}
%\usepackage{graphicx}

% Ott includes.
\usepackage{color}
\usepackage{supertabular}
\input{FILL-ott}

\begin{document}

\maketitle

In this short note I give the details of Bierman's counterexample
\cite{Bierman:1996} to cut elimination of the term assignment
formulation of FILL first given in \cite{Hyland:1993}.  I first
reformulate his counterexample into our definition of FILL, and then
comment on the reason for the counterexample.  Following this I
reformulate the counterexample in the dependency tracking system
proposed by Bra\"uner and de Paiva in \cite{Brauner:1996} and revised
by the same authors in \cite{Brauner:1998}.  In this reformulation we
\todo{Conjecture: Revise if incorrect} will see that the rule proposed
by Bellin but communicated by Bierman in \cite{Bierman:1996} is the
proper left rule for par.

\section{The $\FILLdrulename{Parl}$ Inference Rule}
\label{sec:the_parl_inference_rule}
The existing $\FILLdrulename{Parl}$ inference rule is as follows:
\[
  \FILLdruleParl{}
\]
In the terms $[[let t1 be x (+) stp in t2]]$ and $[[let t1 be stp (+) y
in t2]]$ the variables $[[x]]$ and $[[y]]$ in the patterns are bound
in $[[t2]]$. So when applying the $\FILLdrulename{Parl}$ rule we bind
both the free variable $[[x]]$ in $[[di]]$, and the free variable
$[[y]]$ in each $[[fi]]$.  Now notice that we do this binding even
when the variables are not free in the respective terms.  Furthermore,
as a result of binding these pattern variables we carry along the
newly introduced free variable $[[z]]$.  It is this global binding
across the entire righthand size context along with introducing the
free variable $[[z]]$ in each term that results in the counterexample
of Bierman.
% section the_$\filldrulename{parl}$_inference_rule (end)


\section{Bierman's Counterexample}
\label{sec:gavin's_counterexample}
First lets recall the cut-elimination commuting conversion that is the
locus of the counterexample.  The following cut:
\begin{center}
  \begin{math}
    $$\mprset{flushleft}
    \inferrule* [right= Cut] {
      [[G |- d : C | gk : D]]
      \\
      $$\mprset{flushleft}
      \inferrule* [right= Parl] {
        [[x : C, w : A, G' |- ei : D']]
        \\
        [[z : B, G'' |- ti : D'']]
      }{[[x : C, y : A (+) B, G', G'' |- let y be w (+) stp in ei : D' | let y be stp (+) z in ti : D'']]}
    }{[[G, y:A (+) B, G',G'' |- h( [d/x]{let y be w (+) stp in ei}:D' | [d/x]{let y be stp (+) z in tj} : D'') | gk : D]]}
  \end{math}
\end{center}
Converts into the following:
\begin{center}
  \begin{math}
    $$\mprset{flushleft}
    \inferrule* [right=Parl] {
      $$\mprset{flushleft}
      \inferrule* [right= Cut] {
        [[G |- d : C | gk : D]]
        \\
        [[x : C, w : A, G' |- ei : D']]
      }{[[w : A, G' |- [d/x]ei : D']]}
      \\
      [[z : B, G'' |- ti : D'']]
    }{[[G, y:A (+) B, G',G'' |- h( let y be w (+) stp in [d/x]ei:D' | let y be stp (+) z in tj : D'') | let y be w (+) stp in gk : D]]}
  \end{math}
\end{center}
Notice that in the above cut, the \FILLdrulename{Parl} rule commutes
with \FILLdrulename{Cut}.  So again, we bind $[[w]]$ as a pattern
variable in each $[[ei]]$, and $[[z]]$ in each $[[ti]]$ regardless of
whether or not these are actually free in any of the terms.  In
addition, we introduce $[[z]]$ into each of these terms.

Next we give Bierman's counterexample.  The following uses the first rule given above.
\begin{center}
      \footnotesize
  \begin{math}
    $$\mprset{flushleft}
    \inferrule* [right=\tiny Impr] {
      $$\mprset{flushleft}
      \inferrule* [right=\tiny Parr] {
        $$\mprset{flushleft}
        \inferrule* [right=\tiny Cut] {
          $$\mprset{flushleft}
          \inferrule* [right=\tiny Pr] {
            $$\mprset{flushleft}
            \inferrule* [right=\tiny Ax] {
              \,
            }{[[v : A |- v : A]]}
          }{[[v : A |- v : A | o : _|_]]}
          \\
          $$\mprset{flushleft}
          \inferrule* [right=\tiny Parl] {
            $$\mprset{flushleft}
            \inferrule* [right=\tiny Tr] {
              $$\mprset{flushleft}
              \inferrule* [right=\tiny Ax] {
                \,
              }{[[x : A |- x : A]]}
              \\
              $$\mprset{flushleft}
              \inferrule* [right=\tiny Ax] {
                \,
              }{[[y : B |- y : B]]}
            }{[[x : A, y : B |- x (x) y : A (x) B]]}
            \\
            $$\mprset{flushleft}
            \inferrule* [right=\tiny Ax] {
              \,
            }{[[w : C |- w : C]]}
          }{[[x : A, z : B (+) C |- let z be y (+) stp in h{v (x) y} : A (x) B | let z be stp (+) w in w : C]]}
        }{[[v:A,z:B(+)C |- h(let z be y (+) stp in h{v (x) y} : A (x) B | let z be stp (+) w in w : C) | o : _|_]]}
      }{[[v:A,z:B(+)C |- {let z be y (+) stp in h{v (x) y}}(+){let z be stp (+) w in w} : ((A(x)B)(+)C) | o : _|_]]}
    }{[[v : A |- \z.{{let z be y (+) stp in h{v (x) y}}(+){let z be stp (+) w in w}}:(B(+)C) -o ((A(x)B)(+)C) | o : _|_]]}
  \end{math}
\end{center}
Next we use the second derived rule above to commute the cut in the previous
derivation past the $\FILLdrulename{Parl}$ rule:
\begin{center}
  \footnotesize
  \begin{math}    
    $$\mprset{flushleft}
    \inferrule* [right=\tiny Impr] {
      $$\mprset{flushleft}
      \inferrule* [right=\tiny Parr] {
        $$\mprset{flushleft}
        \inferrule* [right=\tiny Parl] {
          $$\mprset{flushleft}
          \inferrule* [right=\tiny Cut] {      
            $$\mprset{flushleft}
            \inferrule* [right=\tiny Pr] {
              $$\mprset{flushleft}
              \inferrule* [right=\tiny Ax] {
                \,
              }{[[v : A |- v : A]]}
            }{[[v : A |- v : A | o : _|_]]}
            \\
            $$\mprset{flushleft}
            \inferrule* [right=\tiny Tr] {
              $$\mprset{flushleft}
              \inferrule* [right=\tiny Ax] {
                \,
              }{[[x : A |- x : A]]}
              \\
              $$\mprset{flushleft}
              \inferrule* [right=\tiny Ax] {
                \,
              }{[[y : B |- y : B]]}
            }{[[x : A, y : B |- x (x) y : A (x) B]]}
          }{[[y:B, v : A |- v (x) y: A (x) B | o : _|_]]}          
          \\
          $$\mprset{flushleft}
          \inferrule* [right=\tiny Ax] {
            \,
          }{[[w : C |- w : C]]}
        }{[[v:A,cb{z}:B(+)C |- h(let cb{z} be y (+) stp in h{v (x) y} : A (x) B | let cb{z} be stp (+) w in w : C) | let cb{z} be y (+) stp in o : _|_]]}                   
      }{[[v:A,cb{z}:B(+)C |- {{let cb{z} be y (+) stp in h{v (x) y}} (+) {let cb{z} be stp (+) w in w}} : (A (x) B) (+) C | let cb{z} be y (+) stp in o : _|_]]}      
    }{[[v:A |- \z.{{let z be y (+) stp in h{v (x) y}} (+) {let z be stp (+) w in w}} : (B(+)C) -o ((A (x) B) (+) C) | let cb{z} be y (+) stp in o : _|_]]}
  \end{math}
\end{center}
Now notice that as a result of the rule $\FILLdrulename{Parl}$ rule a
fresh free variable $[[z]]$ -- colored blue when it is considered free
-- is introduced, and then let-bound in every term in the righthand
side context.  Furthermore, we bind $[[y]]$ and $[[w]]$ in terms which
do not depend on them, for example, we bind $[[y]]$ in $[[o]]$.
Furthermore, we introduce $[[z]]$ into each of these terms,
especially, the rightmost term.  Thus, the application of the
$\FILLdrulename{Impr}$ rule is in error, because $[[z]]$ occurs in the
right most term.
% section gavin's_counterexample (end)

\section{Bierman's Counterexample in the Dependency-Relation Formalization}
\label{sec:bierman's_counterexample_in_the_dependency-relation_formalization}
Next we give Bierman's counterexample in the dependency-relation
formalization.  To obtain the derivations we simply erase all the
terms:
\begin{center}
      \footnotesize
  \begin{math}
    $$\mprset{flushleft}
    \inferrule* [right=\tiny Impr] {
      $$\mprset{flushleft}
      \inferrule* [right=\tiny Parr] {
        $$\mprset{flushleft}
        \inferrule* [right=\tiny Cut] {
          $$\mprset{flushleft}
          \inferrule* [right=\tiny Pr] {
            $$\mprset{flushleft}
            \inferrule* [right=\tiny Ax] {
              \,
            }{[[v : A |- v : A]]}
          }{[[v : A |- v : A | o : _|_]]}
          \\
          $$\mprset{flushleft}
          \inferrule* [right=\tiny Parl] {
            $$\mprset{flushleft}
            \inferrule* [right=\tiny Tr] {
              $$\mprset{flushleft}
              \inferrule* [right=\tiny Ax] {
                \,
              }{[[x : A |- x : A]]}
              \\
              $$\mprset{flushleft}
              \inferrule* [right=\tiny Ax] {
                \,
              }{[[y : B |- y : B]]}
            }{[[x : A, y : B |- x (x) y : A (x) B]]}
            \\
            $$\mprset{flushleft}
            \inferrule* [right=\tiny Ax] {
              \,
            }{[[w : C |- w : C]]}
          }{[[x : A, z : B (+) C |- let z be y (+) stp in h{v (x) y} : A (x) B | let z be stp (+) w in w : C]]}
        }{[[v:A,z:B(+)C |- h(let z be y (+) stp in h{v (x) y} : A (x) B | let z be stp (+) w in w : C) | o : _|_]]}
      }{[[v:A,z:B(+)C |- {let z be y (+) stp in h{v (x) y}}(+){let z be stp (+) w in w} : ((A(x)B)(+)C) | o : _|_]]}
    }{[[v : A |- \z.{{let z be y (+) stp in h{v (x) y}}(+){let z be stp (+) w in w}}:(B(+)C) -o ((A(x)B)(+)C) | o : _|_]]}
  \end{math}
\end{center}
Next is the derivation after the commute:
\begin{center}
  \footnotesize
  \begin{math}    
    $$\mprset{flushleft}
    \inferrule* [right=\tiny Impr] {
      $$\mprset{flushleft}
      \inferrule* [right=\tiny Parr] {
        $$\mprset{flushleft}
        \inferrule* [right=\tiny Parl] {
          $$\mprset{flushleft}
          \inferrule* [right=\tiny Cut] {      
            $$\mprset{flushleft}
            \inferrule* [right=\tiny Pr] {
              $$\mprset{flushleft}
              \inferrule* [right=\tiny Ax] {
                \,
              }{[[v : A |- v : A]]}
            }{[[v : A |- v : A | o : _|_]]}
            \\
            $$\mprset{flushleft}
            \inferrule* [right=\tiny Tr] {
              $$\mprset{flushleft}
              \inferrule* [right=\tiny Ax] {
                \,
              }{[[x : A |- x : A]]}
              \\
              $$\mprset{flushleft}
              \inferrule* [right=\tiny Ax] {
                \,
              }{[[y : B |- y : B]]}
            }{[[x : A, y : B |- x (x) y : A (x) B]]}
          }{[[y:B, v : A |- v (x) y: A (x) B | o : _|_]]}          
          \\
          $$\mprset{flushleft}
          \inferrule* [right=\tiny Ax] {
            \,
          }{[[w : C |- w : C]]}
        }{[[v:A,cb{z}:B(+)C |- h(let cb{z} be y (+) stp in h{v (x) y} : A (x) B | let cb{z} be stp (+) w in w : C) | let cb{z} be y (+) stp in o : _|_]]}                   
      }{[[v:A,cb{z}:B(+)C |- {{let cb{z} be y (+) stp in h{v (x) y}} (+) {let cb{z} be stp (+) w in w}} : (A (x) B) (+) C | let cb{z} be y (+) stp in o : _|_]]}      
    }{[[v:A |- \z.{{let z be y (+) stp in h{v (x) y}} (+) {let z be stp (+) w in w}} : (B(+)C) -o ((A (x) B) (+) C) | let cb{z} be y (+) stp in o : _|_]]}
  \end{math}
\end{center}
% section bierman's_counterexample_in_the_dependency-relation_formalization (end)


\bibliographystyle{plain}
\bibliography{ref}

\appendix

\section{The full specification of FILL}
\label{sec:fill_specification}
\FILLall{}
% section the_full_fill_specification (end)


\end{document}

